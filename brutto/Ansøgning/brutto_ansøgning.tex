\documentclass{my_application}
\usepackage[danish]{babel} 
\usepackage[utf8]{inputenc}


%\usepackage[applemac]{inputenc}

%
\begin{document}
%
\firstpagestyle{}
%
\begin{figure}[!ht]
\begin{minipage}[t][1 pt]{.69\textwidth}%
	\Large{Søger stilling som Hardware ingeniør.}
\end{minipage}
\hfill%
\begin{minipage}[t][1 pt][b]{.29\textwidth}
	\vfill
	\hfill Dato \today
	%\hfill Dato 22/01/2019
\end{minipage}%
\end{figure}
%
\section{Indledende og hvad jeg kan tilbyde}
Jeg kan bidrage med design og viden indenfor Switch mode power supply, hvor jeg har arbejdet med Power factor korrigering i flyback konvertere på LINAK A/S. Derudover har jeg taget et kursus på EUC syd i forskellige SMPS-kredsløbsmetoder, og haft undervisning i SMPS på Aalborg Universitet (AAU). På LINAK A/S arbejdede jeg med test og validering af high efficiency SMPS med aktiv load, for optimering af standby forbruget i strømforsyningen. Jeg har også repareret flyback konvertere i mine egne aktive højtaler, som jeg bruger i mit hobby firma.

Jeg har viden inden for low noise klasse H forstærkere, hvor jeg har arbejdet med strømforsyninger, som tracker outputsignalet og samtidig har lav egen-støj. Jeg har arbejdet med klasse H trin, som kan gå tættere på rail spændingen end darlingtonkoblingen tillader ved brug af complementary koblinger. Derudover har jeg også arbejdet med både klasse D high power PA forstærkere, og almindelige klasse AB HIFI forstærkere. Ved at benytte en klasse H forstærker kan der opnås samme lyd, som med en klasse AB forstærker, men der opnås høj effektivitet. 

Jeg har viden inden for psykoakustisk. Eksempelvis inden for design af filtrer ved gain justering, således at justeringen ikke er frekvensmæssig lineær, men er opbygget efter den måde øret opfatter lyden på. Jeg har arbejdet med dette i mit hobby firma de seneste 6 år, for at skabe en bedre lydoplevelse blandt publikum, hvilket har givet mig meget positivt feedback. Disse metoder mener jeg, kan bruges indenfor HIFI branchen og særligt i aktive højtalere. Derudover kan min akustiske viden bruges under design af hardware, da jeg har en forståelse af akustikken ved højtalerdesign. Jeg har viden inden for optimering af IIR og FIR filtrer, således at frekvens og fase responset, kan korrigeres med en simpel DSP i assembler. 

Jeg har en stor passion indenfor lyd, som har gjorde, at jeg i 2009 startede min egen hobby virksomhed med lydoptimering. Jeg benytter firmaet til at teste og validere min viden og ideer inden for lydbranchen. Jeg ligger en stor ære i at designe mine egne test og validerings programmer, således at jeg ved nøjagtig, hvad der bliver beregnet og målt. 

Mit hobby firma har givet mig mange kompetencer indenfor kommunikation med slutkunderne, og tilfredsstillelse af deres behov ved hjælp af kunde specificeret løsninger. Ofte er det de små forskelle som gør den store forskel.

Jeg har arbejdet både i teams og selvstændigt. På AAU har jeg på alle semester ud over masteren arbejdet i udviklingsgrupper, for at udvikle mine kompetencer inden for gruppe-projektstyring og samarbejde. Samtidig har jeg arbejdet selvstændigt i mit hobby firma, hvor tidspresset er stort for at nå målet inden for deadline. 
\skipline
%
Mine kompetencer:
\begin{table}[!ht]
	\begin{tabular}{l c}
		Hardwaredesign & \faCheck \\
		Elektroakustik & \faCheck \\
		Mekanisk akustik & \faCheck\\
		Psykoakustics & \faCheck
	\end{tabular}
\end{table}
%
\skipline
%
Jeg håber det har skabt interesse, og ser frem til at høre fra Jer. 
\newline
\newline
I er meget velkommen til at kontakte mig, hvis I ønsker yderligere information.
\skipline
%	
\contact{Jonas Flensborg Riis Buchholdt}{privat@jossound.dk}{40158236}
\end{document}