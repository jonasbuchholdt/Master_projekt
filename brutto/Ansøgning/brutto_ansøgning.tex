\documentclass{my_application}
\usepackage[danish]{babel} 
\usepackage[utf8]{inputenc}


%\usepackage[applemac]{inputenc}

%
\begin{document}
%
\firstpagestyle{}
%
\begin{figure}[!ht]
\begin{minipage}[t][1 pt]{.69\textwidth}%
	\Large{Søger stilling som Hardware ingeniør.}
\end{minipage}
\hfill%
\begin{minipage}[t][1 pt][b]{.29\textwidth}
	\vfill
	\hfill Dato \today
	%\hfill Dato 22/01/2019
\end{minipage}%
\end{figure}
%
\section{Indledende og hvad jeg kan tilbyde}
Jeg kan bidrage med design og viden indenfor Switch mode power supply, hvor jeg har arbejdet med Power factor korrigering i flyback konvertere på LINAK A/S. Derudover har jeg taget et kursus på EUC syd i forskellige SMPS-kredsløbsmetoder, og haft undervisning i SMPS på Aalborg Universitet (AAU). På LINAK A/S arbejdede jeg med test og validering af high efficiency SMPS med aktiv load, for optimering af standby forbruget i strømforsyningen. Jeg har også repareret flyback konvertere i mine egne aktive højttaler, som jeg bruger i mit hobbyfirma.

Jeg har viden inden for low noise klasse H forstærkere, hvor jeg har arbejdet med strømforsyninger, som tracker outputsignalet og samtidig har lav egen-støj. Jeg har arbejdet med klasse H trin, som kan gå tættere på rail spændingen end darlingtonkoblingen tillader ved brug af complementary koblinger. Derudover har jeg også arbejdet med både klasse D high power PA forstærkere, og almindelige klasse AB HIFI forstærkere.

Jeg har viden indenfor psykoakustisk. Eksempelvis indenfor design af filtrer ved gain justering, således at justeringen ikke er frekvensmæssig lineær, men er opbygget efter den måde øret opfatter lyden på. Jeg har arbejdet med dette i mit hobbyfirma de seneste 6 år, for at skabe en bedre lydoplevelse blandt publikum, hvilket har givet mig meget positiv respons. Disse metoder mener jeg, kan bruges indenfor HIFI branchen, og særligt i aktive højttalere. Derudover kan min akustiske viden bruges under design af hardware, da jeg har en forståelse af akustikken ved højttalerdesign. Jeg har viden indenfor optimering af IIR og FIR filtrer og programering af DSP i assembler og C. 

Jeg har en stor passion indenfor lyd, som resulterede i, at jeg i år 2009 startede min egen hobby-virksomhed med lydudlejning og lydoptimering. Jeg benytter firmaet til at teste og validere min viden og ideer indenfor lydbranchen. Jeg ligger en stor ære i at designe mine egne tests og validerings programmer, således at jeg nøjagtigt ved, hvad der bliver beregnet og målt. 

Mit hobbyfirma har givet mig mange kompetencer indenfor kommunikation med slutkunder, og tilfredsstillelse af deres behov ved hjælp af kunde-specificeret løsninger - ofte er det de små forskelle, som gør den store forskel.

Jeg har arbejdet både i teams- og selvstændigt. På AAU har jeg på alle semestre pånær marsterprojektet, arbejdet i grupper, for at udvikle mine kompetencer indenfor gruppe-projektstyring og samarbejde. Samtidig har jeg arbejdet selvstændigt i mit hobbyfirma, hvor tidspresset er stort for at nå målet indenfor deadline. 
\skipline
%
Mine kernekompetencer er blandt andet:
\begin{table}[!ht]
	\begin{tabular}{l c}
		Hardwaredesign & \faCheck \\
		Elektroakustik & \faCheck \\
		Mekanisk akustik & \faCheck\\
		Psykoakustics & \faCheck
	\end{tabular}
\end{table}
%
\skipline
%
Jeg håber det har skabt interesse, og ser frem til at høre fra Jer. 
\newline
\newline
I er meget velkommen til at kontakte mig, hvis I ønsker yderligere information.
\skipline
%	
\contact{Jonas Flensborg Riis Buchholdt}{privat@jossound.dk}{+45 40 15 82 36}
\end{document}