\newcommand{\projectAbstract}{
%The paper deals with the creation of different sound effects for an electric guitar on the a Digital Signal Processor. Some of these effects are the reverb, the flanger and the equalizer. 
%The report includes a thorough explanation of each of the effects followed by the used design approach. Simulations on MATLAB were done to verify the design. All the effects have been coded in assembly for the DSP implementation.  The Assembly code works with the TMS320C5515 DSP from Texas Instruments. 
%In order to make the DSP usable on a variety of electric guitars, a preamplifier was built. All details relating to the design and the implementation of this component are included in the paper as well. 


%The scope of the project at hand is investigating the speech intelligibility, that is achieve via bone-conducted (BC) sound compared to that of airbone sound.
%After extensive literature research, a perceptual is designed. A loudness matching routine is developed, speech intelligibility is tested with a procedure, that is a modified version of the Danish Hearing In Noise Test (HINT). The Bone-conduction Intelligibility Evaluation Routine (BIER) is executed with ten normal hearing subjects. The analysis of the results hints towards a worse conduction of the bone conduction tranducer in terms of intelligibility, that cannot be shown to be statistically significant.


%This project deals with low/mid frequency directivity control.
%The directional characteristics of a single loudspeaker and position of its acoustic center are characterised. An analytical model of the loudspeaker is established to model a beamforming array. After pointing out some properties of the commonly known first order gradient source, a three speaker array is designed. Three speakers are chosen to overcome the disadvantage of the first order gradient speaker not being able to freely control its main lobe direction. To predict the behaviour of the speaker array in a non-free-field environment, a numerical model is set up.  A genetic algorithm is chosen to optimize gain and phase for the individual loudspeakers. A suitable positioning scheme for the speakers is established. Required signal processing parameters are implemented as filters. The measured directional characteristics of the array are compared with the analytical and numerical models and compared with the directional characteristics of a commercial product.
com

}

\newcommand{\projectSynopsis}{
Synopsis
}