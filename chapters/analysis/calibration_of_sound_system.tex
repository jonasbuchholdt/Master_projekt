%\section{Calibration of sound system}
%This section analyses the calibration method, which is used by a selection of some Danish sound company. By experience of the author, the hypothesis is that the sound system is calibrated in one point and the microphone is placed just in front of the \gls{foh}. The \gls{foh} is often a little tent, where the sound engineer controls the sound system. The tent is only open in the direction of the stage and reflection might occur from the tent ceiling to the calibration microphone. 

%\section{Sound pressure level measurement doing the concert}

%\section{sound pressure level doing a concert}
%In Denmark, there is no law limiting the \gls{spl} doing a concert. The only restriction there might be of \gls{spl} is area dependent. In a city the local community has limited the total \gls{spl} average over \SI{15}{\minute} of any event. In the countryside, the sound engineer can decide by himself, and the often used limit is A-weighted \dB{102} average over \SI{15}{\minute}. 


%The standard ?? for long term exposure of high \gls{spl} limits the \gls{spl} for A-weighted \dB{94} average over a maximum of \SI{1}{\hour}, then the ear needs to have a break to ensure no damage of the hearing. A concert i often more than \SI{1}{\hour} with A-weighted \dB{102} average. This is at least \dB{8} A-weighed more than the regulation recommends. It shall here be understood that the \gls{spl} measurement is done in the \gls{foh}, and the exposed \gls{spl} is higher for the audience closer to the stage.   