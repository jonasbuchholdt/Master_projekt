\section{Live venue sound challenges}
This section explores the challenges of producing sound in an outdoor environment. The challenge of producing a good sound experience for the audience highly depend on the calibration method and the atmosphere condition. It is well known that acoustically wave propagation is strongly affected by the inhomogeneous atmosphere doing the outdoor sound propagation. This inhomogeneous atmosphere shifts the calibration of the sound system which affects the intelligibility. In \autoref{sec:ana:aco_liv_ven} an overview of high \gls{spl} \gls{pa} system is discussed.



\subsection{Acoustics as live venue}\label{sec:ana:aco_liv_ven}
An outdoor \gls{pa} system is an essential sound reinforcement concept today. It is used to address information, music or just entertainment where the number of audiences is large, sometimes more than 10.000 audiences. The number of the audience makes it difficult to address the information to a large number of the audience without the reinforcement of the information. The reinforcement is nearly always done from a stage with a sizeable \gls{pa} system and sometimes delay unit in the middle of the audience area. The stage lifts the artist while the \gls{pa} system is designed to cover the audience area with sound. The optimal \gls{pa} system covers the area with a linear frequency spectrum in the audible frequency range with a homogeneous \gls{spl}. Today, the used speaker is a line source array flown in both side of the stage and is therefore only close to the audience in front of the stage. The line source array is an array of small identically wide speakers attached to each other, to form a vertical line of speakers. An example of a line source array is shown in \autoref{fig:ana:flo_kud}

\fig{flown_kudo}{The figure shows an illustration of a KUDO line source array from L-Acoustics \citep{KUDO_manual}}{fig:ana:flo_kud}{0.4}

Every speaker or a small group of the line source array can be controlled individually, both in sound coverage area angle and \gls{spl}. The benefit of using the line source array design is that the coupling between the speaker makes a line acting source. With an optimised control system of the line source array, the audience area can is covered with sound such that all audience can hear the information without damage the ear of the frontal audience. An optimised line source array has, for example, an optimised main lobe such that the lower part of the main lobe lays flat along the audience area.
The following \autoref{fig:ana:aco:liv_ven} shows a graphical illustration of the outdoor \gls{pa} venue concept.

\xfig{analysis/concept_outdoor.pdf_t}{The figure illustrate the concept of outdoor \gls{pa} venue}{fig:ana:aco:liv_ven}{0.65}

As shown in \autoref{fig:ana:aco:liv_ven}, the distances from one element in the line source array to the receiving audience dependent on the audience position. The distance indicates that the signal to every line source element has to be set individually to cover the audience area wit ha homogeneous \gls{spl}. The individual control of the source is necessary because of the wave amplitude decay with distances. This phenomena is addressed in \autoref{sec:ana:geo_spr_los}. The adjustment is not as simple as just supply the upper speaker with more power. A sound wave is a mechanical movement of the particle in the air, which condensate and compression the air molecule, then low pressure and high pressure respectively. The movement of the molecule depends on the medium, and in this thesis, the medium is limited to air. The \gls{spl} is the pressure divination of the instantiates atmospheric pressure. The atmospheric pressure, therefore, set a lower bound on the condensation while very high pressure changes the speed of sound and distort the wave as it propagates. To ensuring that the information is communicated to the audience without distortion, the limitation is addressed in \autoref{sub:sec:pre_imp}.  The medium in the air is not constant and varies over time regarding pressure, wind, humanity and temperature. The analysis starts with the experience for live concert of the author in \autoref{sec:ana:aut_exp_con}, next \autoref{sec:ana:hom_ats_con} address the impact of homogeneous atmospheric effect on sound propagation. Then \autoref{sec:ana:hom_ats_con} address the impact of  inhomogeneous atmospheric effect on sound propagation. 


\subsection{Author experience of live concert}\label{sec:ana:aut_exp_con}
The Author of the thesis has experience with live concert both as an audience and as a sound engineer. The aspect of being the sound engineer and an audience to a live concert is very different. As a sound engineer, the area for controlling the sound is a secured area with a tent as protection. The tent roof often shadows for the high frequency, and the walls make standing waves of the low frequency because the distance between parallel tent walls fits with the wavelength for the low frequency. The sound engineer control area is defined as the \gls{foh}. The \gls{foh} is often equipped with an additional speaker, and the sound engineer does not fully know how it sounds outside the \gls{foh}, but base there mixes on experience. The aspect of being an audience depends on where the audience is regarding the stage. In close hand to the stage the \gls{spl} is high and often to high especially in the low frequency. The low frequency is often made as a vertical array at the ground or two end-fire arrays and shall be able to exhibit all audience by an audible low frequency spectrum typically from \Hz{25} but one company extends down to \Hz{13}. Therefore the \gls{spl} just in front of the subwoofer has a very high \gls{spl}. This position is not comfortable to be at in longer period, and the high \gls{spl} mask the higher frequency. The optimal audience position is in the centre of the stage and not as long from the stage as the delay towers. The average \gls{spl} is often less than \dB{102} since the sound engineer try to keep a maximum average \gls{spl} at \dB{102} just in front of the \gls{foh}. Moreover, it is the stereo sweet spot. This position is the only position where the stereo image is optimal. The stereo perspective problem is a hot topic nowadays, both L-Acoustics \citep{l_acoustics_l_isa} and D\&B Audiotechnik \citep{dbsoundscape} have made there own solution to the problem. The idea is to fly many small line source array above the stage and assign every musician to there own line source array. The concept minimises the interference between two line source array playing the same mono signal. 
Near the delay towers or approximately \SI{50}{\meters} from the main stage, the low frequency spectrum is still sharp and audible but something happens to the high frequency. Often the high frequency disappears for a few seconds and gets back. This phenomenon altering through the full concert. Behind the delay towers, the line source array in the delay tower reproduces the sound such that the audience in the back also gets the high frequency spectrum. The question is why does the high frequency disappear for a short period when the low frequency does not? This analysis focus on finding the atmospheric condition which cause the phenomena.

 
