\section{Live venue sound challenges}
This section explore the challenges of producing sound in an outdoor environmental. The challenge of producing a good sound experience for the audience highly depend on the calibration method and the atmosphere condition. It is well known that acoustically wave propagation is strongly affected by the inhomogeneous atmosphere doing the outdoor sound propagation. This inhomogeneous atmosphere shifts the calibration of the sound system which affect the intelligibility. \autoref{sec:ana:aco_liv_ven} gives an overview of the need of controled acoustics at live venue.


\subsection{Acoustics as live venue}\label{sec:ana:aco_liv_ven}
A outdoor \gls{pa} venue is an important concept today. It is used to address thougth, music or just entertainment where the number of audience is large. Sometimes more that 10.000 audience. The number of audience makes it difficult to address the opinion in an indoor venue and without speakers. Therefore an outdoor \gls{pa} venue is a large area with a stage and speaker, which is designed to cover the audience area with a frequency spectrum that is linear in the audiable frequency range. The speaker is often hanged in the side of the stage and is therefore only close to the audience in the front. Today the used speaker is called a line source array. It is an array of small wide speaker attached to each other to form a vertical line. Every speaker or a small group of the line source array can be controlled individual, both in sound coverage area angle and \gls{spl}. The benefit of the line source array modul design is that every speaker can be angled such that they cover it oven part of the audience area. The following \autoref{fig:ana:aco:liv_ven} shows a graphical illustration of the outdoor \gls{pa} venue concept.

\xfig{analysis/concept_outdoor.pdf_t}{The figure illustrate the concept of outdoor \gls{pa} venue}{fig:ana:aco:liv_ven}{0.65}

As it can be seen in the \autoref{fig:ana:aco:liv_ven} the distances from the one element in the line source array to the receiver is dependent on the receiver position. This indicate that the signal to every line source element have to be adjusted with respect to the coverage area. This is necessary because the wave amplitude decay with distances and is exaplained in \autoref{sec:ana:geo_spr_los} in an ideal atmospheric condition. The adjustment is not as simple as just supply the upper speaker with more power. A sound wave is a mechanical movement of the particle in the air, which condensate and compression the air molecule so low pressure and high pressure respectavly. The movement of of the molecule depends on the medium, and in this report the medium is limited to air. The sound pressure is a local pressure divination of the instantiates atmospheric pressure. The atmospheric pressure therefore set a lover bound on the condensation but in theory there is no upper bound. To communicate the purpose without introducing distortion by the lower limit, the maximum amplification is therefore limited by the lover bound of the atmospheric pressure. Luckily the pressure near the ground is typically \SI{101.325}{\kilo\pascal} or a maximum pressure of \dB{194}. The movement of the particle in the air depends on the medium in the air. The medium in the air is not constand and varie over time with respect to wind, humanity and temperature.   \autoref{sec:ana:hom_ats_con} address the impact of homogeneous atmospheric effect on sound propagation. \autoref{sec:ana:hom_ats_con} address the impact of  inhomogeneous atmospheric effect on sound propagation. 