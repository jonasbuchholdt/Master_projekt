
This chapter analyses the known sound wave effect in the sound path. The analysis is addressed as following  

\begin{enumerate}
\item In \autoref{sec:ana:geo_spr_los} the distance dependency \gls{spl} loss from a line source array is analysed. 
\item In \autoref{sec:ana:hom_ats_con} the homogeneous atmospheric effect on sound propagation is analysed.
\item In  \autoref{sec:ana:inhom_ats_con} the  impact of  inhomogeneous atmospheric effect on sound propagation is analysed.
\end{enumerate}



\section{Ideal geometric spreading loss}\label{sec:ana:geo_spr_los}
When a line source generates a sound wave, the wave field exhibits two fundamental difference spatially directive regions, near-field and far-field. In the near-field, the wave propagates as a cylindrical wave wherein the far-field the wave propagates as a spherical wave. When the wave propagates as a cylindrical wave, the wave propagates only in the horizontal plane, and therefore the attenuation is \dB{3} per doubling of distance. For a spherical wave propagation, the wave propagates in both the horizontal direction and the vertical direction. Therefore the attenuation is \dB{6} per doubling of distance. The near-field and far-field attenuation are based on non-absorption homogeneous atmospheric conditions. The border between the near-field and far-field depends on the hight of the line source array and the wavelength. The distance is calculated with Fresnel formula \autoref{eq:ana:near_field}, where the wavelength $\lambda$ is approximated to $\frac{1}{3f}$ \citep{bauman2001wavefront}

\begin{equation}\label{eq:ana:near_field}
d_{B} = \frac{3}{2}f \cdot H^{2}\sqrt{1-\frac{1}{(3f \cdot H)}}
\end{equation}

\startexplain
\explain{d_{B}}{is the distance from the line source array to the end of the near-field}{\meter}
\explain{f}{is the frequency}{\kilo\hertz}
\explain{H}{is the hight of the line source array}{\meter}
\stopexplain

In equation \autoref{eq:ana:near_field} it can be calculated that less than \Hz{80} radiate directly intro spherical wave on the exit of the line source array no matter the hight. The following \autoref{fig:ana:near_far_field} shows a horizontal cut of the near-field, far-field from a line source array. 

\fig{near_far_field}{The figure shows a horizontal cut of a sound wave radiation pattern of a line source array \citep{bauman2001wavefront}.}{fig:ana:near_far_field}{0.80}

As seen in \autoref{fig:ana:near_far_field}, the wave propagate as planar wave in the near-field. In the vertical domain, the plane wave propagates as a cylindrical wave in the near-field, where the coverage area for every double of distance is twice as big. Since the coverage area is twice as big, the \gls{spl} is \dB{-3} for the doubled distance. When the wave excites distance $d_B$, the wave propagates intro far-field where the coverage area is four times higher while travelling the double of distance and therefore the \gls{spl} is \dB{-6}. In far-field, the wave propagates as a point source. The following \autoref{fig:ana:KUDO_nearfield_limit} gives two examples of the $d_B$ limit with different line source array hight.

\plot{plot/KUDO_nearfield_limit}{The figure shows two hight example calculated from \autoref{eq:ana:near_field}.}{fig:ana:KUDO_nearfield_limit}

As seen in \autoref{fig:ana:KUDO_nearfield_limit}, while the hight is the double, the far-field is moved four times as far back.


\section{Homogeneous atmospheric conditions}\label{sec:ana:hom_ats_con}
This section aims to analyse the sound wave propagation in homogeneous atmospheric conditions. It is well known that the sound wave propagation is highly depending on the atmospheric conditions. The propagation depends on the atmospheric pressure, wind, temperature and humidity, where the two latter moreover is frequency dependent. The following sections introduce a brief discussion of homogeneous atmospheric conditions effect on sound propagation.


\subsection{Humidity and temperature impact}\label{sec:ana:hu_temp}
The temperature and humidity have three impacts on sound wave propagation from a line source array, directionality of the line source array, the speed of sound and a lowpass effect. The following description starts with the latter. 

\paragraph{Lowpass effect} The effect of humidity and temperature on sound wave propagation act as a lowpass filter while the sound wave propagates. The low frequency remains without any additional attenuation where the high frequency highly depends on the temperature and humidity. In other words, attenuation in the high frequency range does not only depends on the spreading loss but also temperature and humidity. Therefore, for long distance, the atmospheric conditions have a high influence on the frequency spectrum delivered to the audience. Humidity and temperature attenuation are already well studied and standardised. Standard \citep{iso_9613-1} gives an overview of calculating the \gls{spl}  attenuation concerning the frequency, distance, temperature and humidity. The article \citep{corteel2017large} gives some examples of attenuation at a distance of \SI{100}{\meter}.  The following \autoref{fig:ana:temp_eg} shows the worst-case scenario from \citep{corteel2017large}.
\fig{temp_worst_case}{The graph shows the attenuation in \db with respect to frequency, humanity and temperature \citep{corteel2017large}.}{fig:ana:temp_eg}{0.8}

The article shows that if humidity increases proportionally to the temperature, the lowpass effect is small. If the change in temperature and humidity is the opposite of each other, for example, high temperature but dry, the attenuation in high frequency is significant. As shown in \autoref{fig:ana:temp_eg} the attenuation in the high frequency is significant and excite \dB{30} within the audible frequency range. The attenuation is such markedly that applying more power does not cover the attenuation without introducing high distortion as is explained in \autoref{sub:sec:pre_imp}

\paragraph{Speed of sound} The second consequence is the speed of sound. At temperature range from \SI{0}{\celsius} to \SI{40}{\celsius} the speed of sound with respect to humidity change is sparse and mostly only depend on temperature. At \SI{0}{\percent} humidity, the speed of sound increases with \SI{0.6}{\meter\per\second} for every increasing degree \si{\celsius}. At humidity higher that \SI{0}{\percent} the speed of sound increase with humidity, depends on temperature. The wave propagation speed start at \SI{331.5}{\meter\per\second} at \SI{0}{\celsius} and \SI{0}{\percent} humidity. The following \autoref{fig:ana:hu_speed} shows the speed of sound with respect to humidity and temperature. 

\plot{plot/speed_hum_temp}{The figure shows the increase of sound speed with respect to humidity and temperature \citep{bohn1987environmental} \citep{humanity_effect_on_speed} }{fig:ana:hu_speed}

As seen in \autoref{fig:ana:hu_speed}, the effect of humidity is negligible compared to the effect of temperature changes, but as the temperature increases the relative humidity gets significant. At a temperature of \SI{40}{\celsius} the speed of sound is changed \SI{4}{\meter\per\second} from \SI{0}{\percent} humidity to \SI{100}{\percent}. The humidity is relative because it depends on the temperature. The higher the temperature is, more water can be contained intro the air molecule. The relative in relative humidity is assumed in the rest of the thesis and is not written. 


\paragraph{Directivity} The directivity of a line source array in the mid and high frequency is always controlled mechanically by a horn because the wavelength is short compared to the size of the line source array. At low frequency, the wavelength is too long to be controlled mechanically by a horn. Therefore the directional pattern is controlled via cancellation from a backwards pointing speaker. The directivity of both the low frequency and the high frequency driver sufferers from temperature increased. At the high frequency, the main lobe gets narrower when the mechanical horn gets warmer, and the effect is notable when the sun directly heats the horn. When the surface of the horn heats up by the sun, the temperature is able to get much warmer in the horn that the air temperature. Therefore the surface of the horn affects the directivity of the high frequency by radiate warm air from the surface. The resend that main lope gets narrower is that the wavelength gets shorter at higher temperature \citep{levine2018influence}. The directivity of the low frequency is affected as in the high frequency with the temperature increase. The difference is not as significant as in the high frequency since the wavelength is longer than the speaker cabinet, and the surface is not heated as much by the sun. The directivity is then nearly not affected due to the sunlight, but mostly by the temperature increase and decrease. As in the high frequency temperature, differences change the wavelength, and then the length between the speaker in a cardioid low frequency does not match the optimised distance between the speaker more. 



\subsection{Wind impact}
The wind influence is depending on the angle of the wind direction with respect to the direction of sound propagation. A homogeneous wind is a laminar wind flown with the same homogeneous speed. The following analysis assumes homogeneous laminar wind flow from one direction. The analysis is of both oblique wind and parallel wind with respect to the frontal direction of the line source array. The analysis starts with the latter. 

\paragraph{Parallel to sound propagation} When the wind flows in the same direction as the sound wave propagation, the wind flow in \si{\meter\per\second} is an addition to the speed of sound. When the wind flows in the opposite direction, it is a negative addition.  In other cases, the influence is complicated since the wind deflect the sound waves.


\paragraph{oblique- and crosswind} The effect of homogeneous oblique- and crosswind on sound propagation from a line source array is rarely studied. One author has addressed the problem in a simulation of a low frequency source \citep{crosswind_simulation} where the author of  \citep{BALLOU2008xi} have practical experience with high power sound system and indicate that crosswind effect might be frequency dependent.   The author indicates that the frequency dependency might be due to the directionality of the high frequency drivers. The author of \citep{crosswind_simulation} has simulated a homogeneous crosswind effect on an omnidirectional source at \Hz{100}. The author of \citep{ray_tracing} implemented a ray tracing method with a vector based interpolation as shown in \autoref{fig:ana:cal_wav_dir}.



\xfig{analysis/cal_of_crosseffect.pdf_t}{The figure shows a geometrical ray tracing calculation scheme of calculate the resulting wave direction at crosswind \citep{ray_tracing}, \citep{crosswind_simulation}}{fig:ana:cal_wav_dir}{0.5}
\startexplain
\explain{c}{is the speed of sound}{\meter\per\second}
\explain{n}{is the normal unit vector}{\meter}
\explain{v}{is the speed of wind}{\meter\per\second}
\explain{v_{ray}}{is the resulting sound ray}{\meter}
\stopexplain

As seen in \autoref{fig:ana:cal_wav_dir}, the ray vector $v_{ray}$ is an addition of the sound speed vector $c \cdot n$ and the speed of wind $v$. The wave speed and wavelength, therefore, depend on the speed of the wind and the angle between the wind and the sound propagation. The following \autoref{eq:ana:wind_angle} calculate the speed of sound in the $v_{ray}$ direction with respect to the wind speed and angle.



\begin{equation}\label{eq:ana:wind_angle}
c_r = c+||v||_2 \cdot sin(\theta) = ||c \cdot n + v||_2 = ||v_{ray}||_2
\end{equation}

\startexplain
\explain{\theta}{is the angle of the wave with respect to the wind}{\degree}
\explain{c_r}{is the resulting speed of sound}{\meter\per\second}
\stopexplain



As the wave propagates, the resulting $v_{ray}$ increases in the direction of the wind. The article \citep{crosswind_simulation} simulates the effect of crosswind in a \gls{fdtd} simulation with a wind speed of \SI{102.9}{\meter\per\second}. For the acceptable condition to a concert, the wind speed is less than \SI{20}{\meter\per\second}. Otherwise, the audience is escorted from the stage to the exit, and the line source array system is taken down to ensure safety. The following \autoref{fig:ana:cro_win_sim} shows a simulation result from \citep{crosswind_simulation}, where the source is an omnidirectional \Hz{100} spherical source while the wind has a constant uniform wind speed from left. The simulation is done in two dimensions.



\fig{crosswind_simu.pdf}{The figure shows a simulation of a \Hz{100} omnidirectional source with a uniform constant wind speed from left with wind speed of \SI{102.9}{\meter\per\second} \citep{crosswind_simulation}. }{fig:ana:cro_win_sim}{0.8}

As seen in \autoref{fig:ana:cro_win_sim}, the homogeneous crosswind does not affect the direction of the wave from a low frequency spherical source. It only affects the time of arrival to the audience.

\subsection{Pressure impact}\label{sub:sec:pre_imp}
The influence of atmospheric pressure change is low compared to the effect of wind, humidity and temperature. The average atmospherical absorption from \Hz{4000} to \Hz{16000} with fixed temperature and variable humidity, increases with \dB{2} while going from \SI{101.33}{\kilo\pascal} to \SI{54.02}{\kilo\pascal}. The atmospheric pressure then only have a negligibility influence on sound propagation and is generally not frequency dependent. 

Besides the small impact of pressure difference in the atmosphere, the high pressure generated by the line source array does have a tremendous influence on the sound propagation. There are three states in the propagation way that is producing distortion concerning the pressure. The design of the high frequency horn \citep{czerwinski1999air}, the port design of the low frequency driver \citep{vanderkooy1998nonlinearities} and the influence of the sound path. The following description starts with the latte.


\paragraph{Sound path} In the sound path, two factors distort the wave doing propagating in air. As described in \autoref{sec:ana:aco_liv_ven}, a sound wave is condensation and compresses of the air particle. The air medium, therefore, has a lower limit that cannot be less than vacuum. The higher bound of \gls{spl} is then depending on the atmospheric pressure. As an example, at \SI{54.02}{\kilo\pascal} the highest \gls{spl} before distortion caused be vacuum is \dB{188.6} and at \SI{101.33}{\kilo\pascal} the highest \gls{spl} before distortion caused be vacuum is \dB{194.1}. 

There is, therefore, a higher limit determined by the atmospheric pressure to vacuum,  but distortion occurs much before the limit of vacuum.  High pressure in the compression also distorts the sound because of the lack of linear dependency between the particle velocity and stiffness in the sound wave. The stiffness or density increases while the air particle is closer to each other. Therefore \gls{spl} increases more than the density of the sound wave, which causes the compression of the sound wave to be stiffer and therefore propagates faster than in the condensation of the wave. This speed differences, therefore, produce harmonic distortion, and is even present in \gls{spl} less than \dB{120} \citep{czerwinski1999air}. The speed differences transform the sinusoid intro a sawtooth as it propagates which transfer energy to the harmonic of the propagation frequency. The distortion is not only \gls{spl} dependent, but also depend on the frequency. The higher the frequency is, the faster the sinusoid transformers intro a sawtooth, therefore, the distortion increases with frequency for constant \gls{spl}. The harmonic frequency is higher than the fundamental frequency and therefore, as explained in \autoref{sec:ana:hu_temp}, the harmonic has higher attenuation with respect to the viscosity. In most cases, the attention is not as high as the increase of the harmonic distortion, and therefore, the distortion of the wave propagation is not fully compensated by the viscous losses in the air. \citep{czerwinski1999air}. 
The distortion made by air propagation is much less than the distortion in the mouth of the speaker, which leads to the next distortion problem produced by high-pressure \citep{czerwinski1999air}.

\paragraph{Driver throat and mouth design} High pressure in both horn phase plug, sealed enclosures, vented enclosures and reflex enclosures for low frequency driver cabinet produce distortion as they act as nonlinear components. The latter produce distortion because high pressure makes air turbulence in the vent. In the optimal design, the distortion of air turbulent is low but is always present in high-pressure \citep{roozen1998reduction}. The air turbulence is not only caused in the vent of the low frequency driver, but it also occurs in the phase plug of the compression driver if the \gls{spl} is high \citep{czerwinski1999air}. The distortion depends on the moving mass, the stiffness and the viscous losses in the air on the diagram displacement and the \gls{spl}. As the air in the high frequency driver compress, it becomes denser, stiffer and thicker, which make nonlinear wave propagation. It typically occurs when the compression chamber exceeds approximately \dB{170}. At a higher level, the particle velocity resistance to the air flow increases and the laminar air flow turns intro turbulent air flow. The distortion is also depending on the length of the horn and the expansion rate of the horn flare. To keep the distortion as low as possible for the high frequency driver, the displacement of the diaphragm should be kept significantly lower than the hight of the compression chamber \citep{voishvillo2004comparative}. Therefore, to keep the displacement of the high frequency driver as low as possible, the frequency range should be limited.



\subsection{Ground absorption and reflection}\label{ana:ground_ref}
In a concert area, ground absorption and reflection is complicated because there are two very different situations. Before the concert, the area is a local plan area often with mown grass and with ground reflection.  An example of a frequency response over mown grass where the measuring hight of the microphone is in the hight of the ear is given in \citep{review_of_sound}. The measurement shows that the ground reflection affects the frequency response with high interference. 
A measurement in \autoref{ch:ap:measurement_one} is performed where the ground reflection has a significant influence on the received frequency response. In this measurement, inhomogeneous airflow is present, but the interference is similar in homogenous airflow \citep{review_of_sound}. Doing the concert the exciting part is not such ground reflection effect but the audience reflection or absorption.  The area along the concert is packed by the audience and therefore, the reflection is not easy to calculate. The absorption and reflection in an outside concert area with a group of audience is rarely studied, but absorption for the audience inside a concert hall is highly studied \citep{audience_abso}. The absorption of the audience is founded to be high in all measured concert hall from \Hz{1000} octave band to \Hz{4000} octave band \citep{audience_abso}. The average absorption $a_{sabine}$ coefficient is calculated to be above 0.80. The method and result can be founded in \citep{audience_abso}. The reflection in the high frequency in the audience area doing concert is therefore assumed to be low. At low frequency, the article \citep{audience_abso} indicate that the absorption decay with frequency beneath  \Hz{250}, but the octave band for low frequency driver, which is \SI{31.5}{\hertz}, is not measured by \citep{audience_abso}. The low frequency absorption at \SI{31.5}{\hertz} octave band is therefore assumed to low. The low frequency driver is mostly located in front of the stage on a line or in end-fire settings, often with a maximum distance of half the wavelength from acoustical centre to acoustical centre. The distance between the low frequency driver is determined by the half wavelength of the highest frequency, such that the wave radiates as a plan wave \citep{bauman2001wavefront}. A higher distance between the acoustical centre causes interference in the low frequency in the audience area. 



 
\section{Inhomogeneous atmospheric conditions}\label{sec:ana:inhom_ats_con}
This section aims to analyse the sound wave propagation in inhomogeneous atmospheric conditions. In an inhomogeneous atmosphere, the pressure and speed is a function of position. By this fact, the modelling of a sound wave is very complex and depend on various variables such as temperature, humidity and wind speed. The following sections give a short introduction to the effect of inhomogeneous atmospheric conditions. 
 
 
\subsection{Atmospheric refraction} \label{sec:ana:atm_ref}

 %and the effect of differences between low frequency and high frequency seems to be unclear.

When the wind speed, the temperature and humidity is assumed to be homogeneous in the sound field, the sound is travelling in a straight not refracting wave. Often this is not true, the wind speed increases logarithmically with the hight from the ground to the geostrophic wind \citep{asmos_acous_2016} in the free troposphere \citep{spr_hand_book}, and the temperature and humidity are inhomogeneous. The geostrophic wind in the free troposphere is located in a hight from approximately \SI{1}{\kilo\meter} above the ground \citep{spr_hand_book}, \citep{geostrophic_wind}. The inhomogeneous atmospheric condition makes the speed of sound to depend on the hight from the ground. This inhomogeneous atmospheric condition results in a curved sound path and is defined as atmospheric refraction. For small distances, the atmospheric refraction has a spars effect on the sound travelling path, because the speed of sound is much faster than the speed of the wind and the temperature change. Generally distance up to \SI{50}{\meter} is often assumed to have no significant refraction effect \citep{effect_of_wind}. For distances larger than \SI{50}{\meter} the refraction is assumed to have a significant influence, especially when the sound source and the receiver are close to the ground. Refraction is frequency and distance dependent and is measured in \si{\decibel} excess attenuation. The means of excess attenuation is that only the effect of wind or temperature is considered, all other atmospherical effect is excluded. A measurement is given in \citep{review_of_sound} for a point source where the wind speed is \SI{5}{\meter\per\second} and the wind direction is parallel to the sound path. At a distance of \SI{110}{\meter}, it is observed that frequency above \Hz{400} is refracting where frequency below is rarely effected of refraction. Moreover, at a distance of \SI{615}{\meter}, the refraction is present in the full measured frequency range from \Hz{50} to \Hz{3200}. In the perspective of a live concert, the interesting distance is the \SI{110}{\meter} from the line source array to the audience rather than the \SI{615}{\meter}. Both the downwards and upwards refraction is interesting. In the upwards refraction, the audience might be in the shadow zone where for the downwards refraction the high frequency reflection from the ground is assumed to be low when the concert area is full of audience. Therefore the high frequency is refracted down intro the frontal audience, and only sparse reflection of the high frequency propagate to the back part of the audience. The following \autoref{fig:ana:shadow_zone} display the phenomena of upwards refraction.


\xfig{analysis/shadow_zone.pdf_t}{The figure illustrates that the shadow zone occurs from an upwards refraction. A line source array contains many couplet point sources. Every lowest sound path dashed line indicates the lower directional angle of one point source in the line source array.}{fig:ana:shadow_zone}{0.73}


The following description is based on the distance of \SI{110}{\meter} and upwards refraction. As explained in \citep{review_of_sound} the refraction at a distance of \SI{110}{\meter} is highly frequency dependent. At a frequency below \Hz{400} the effect is sparse, but above the effect is high and may result in \dB{20} attenuation at the audience. The reason that the refraction is frequency dependent is that the scale of the wind gradient and temperature gradient close to the ground is small compared to the wavelength of the low frequency \citep{review_of_sound}. This theory does not follow the shell's law of refraction. Shell's law describes the refraction as a layer change in the medium of propagation. Shell's law of refraction is defined as \autoref{eq:ana:shell_law}

\begin{equation}\label{eq:ana:shell_law}
\frac{cos(a_1)}{c_1} = \frac{cos(a_2)}{c_2}
\end{equation}

\startexplain
\explain{a_1}{is the input angle in the horizontal plan}{\degree}
\explain{c_1}{is the sound of speed in the medium of arrival}{\meter\per\second}
\explain{a_2}{is the output angle in the horizontal plan}{\degree}
\explain{c_2}{is the sound of speed in the medium of destination}{\meter\per\second}
\stopexplain

As shown in shell's law \autoref{eq:ana:shell_law}, the frequency dependency is not a factor. The article \citep{review_of_sound} only explorer frequency up to \Hz{3200}, but since the refraction depends on the wavelength, the distance of refraction wave might be smaller for higher frequency. The attenuation with respect to refraction seems to have a saddle attenuation at \dB{20}. A measurement in \citep{review_of_sound} shows the attenuation for the centre frequency of \Hz{1200} with \octave{3} band filtered aircraft noise over mown grass. The measurement is interesting with respect to a concert area and is therefore shown in \autoref{fig:ana:excess} 


\fig{excess}{Excess attenuation measured for aircraft noise in the \Hz{1200} \octave{3} band for the ground-to-ground configuration. The vector component of the wind velocity in the direction of propagation for $\blacktriangle$ is \SI{5}{\meter\per\second}, $\square $ is \SI{0}{\meter\per\second}, and $\triangledown$ is \SI{-5}{\meter\per\second}. The temperature profile is neutral. $F_s$ is the shielding factor, B is the shadow boundary \citep{review_of_sound}}{fig:ana:excess}{0.7}

The following two paragraphs explain the difference between wind refraction and temperature refraction.

\paragraph{Temperature} Temperature decreases with respect to the hight at day time and increases at the night time. The increase or decrease is usually approximated as a logarithmic function. In the day time, the sun heats the ground even on a cloudy day, and the concert area is full of audience. Therefore, the eath and audience radiate warm air, which makes the temperature at a low hight warmer than the temperature at higher hight. These phenomena are named lapse, where the opposite is defined as inversion. As explained in \autoref{sec:ana:hu_temp}, the speed of sound depends on the temperature. Therefore, at day time, the speed of sound in this situation decay with respect to hight. The speed change is modelled as a change of layer for a plane wave. The output angle of the layer change follows the shell's law when the frequency dependency is excluded. Therefore when the temperature profile is logarithmic, the layer change is a function of hight and change the wave direction. The wave direction of the descript weather condition results in an upwards refraction. Since the temperature is a scalar quantity uniformly over a large area and a function of hight, the same temperature profile is applicable all around an omnidirectional sound source. Therefore the upwards refraction is uniform all around the line source array. The following \autoref{fig:ana:temp_ref} illustrate the phenomena where the temperature decay concerning the hight and the line source array is omnidirectional. The omnidirectionality of the line source array is only present in the low frequency and typically below \Hz{200}.

\xfig{analysis/refraction_temp.pdf_t}{Wave refraction of a horizontal omnidirectional line source array in inhomogeneous temperature with lapse profile}{fig:ana:temp_ref}{0.9}

When the temperature profile is reversed, the refraction is downwards. 
 

\paragraph{Wind} With respect to the wind speed, a concert area is often a protected area with, for example, barrier, stage and building. This blockage and the ground friction slows down the wind speed near the ground and cause turbulence. Moreover, from nature itself, the wind speed is often logarithmically increased with respect to the hight. When the wave is propagation in the same direction as the wind, the atmospheric refraction refracts the sound wave downwards. When the wave propagates against the wind, the atmospheric refraction refracts the sound wave upwards. The following \autoref{fig:ana:wind_ref}  illustrate the phenomena with a logarithmic increasing wind from left, and where the line source array is omnidirectional.

\xfig{analysis/refraction.pdf_t}{Wave refraction of a horizontal omnidirectional line source array in inhomogeneous logarithmically increasing wind profile where the wind gradient points towards left}{fig:ana:wind_ref}{0.9}

As shown in \autoref{fig:ana:shadow_zone}, the refraction is upwards when the wind flows in the opposite direction as the wave propagation. Behind the line source array, the refraction is downwards and is therefore different than for temperature refraction. The refraction of wind is the most dominant at a distance of \SI{110}{\meter}. The following \autoref{fig:ana:fre_dep_exc} shows an excess attenuation plot of both inhomogeneous wind and lapse temperature profile. 

\fig{frequency_dep_excess}{Observed attenuation of aircraft noise in a ground-to-ground configuration under a variety of weather conditions. Calculated losses from atmospheric absorption and spherical spreading have been subtracted from the attenuation measured in \octave{3} bands for distances of \SI{110}{\meter} and \SI{615}{\meter}. The numbers on the curves indicate the vector component of the wind velocity in the direction of propagation in \si{\meter\per\second}. All curves are for neutral conditions of temperature except for those marked L, which are for the lapse. \citep{review_of_sound}}{fig:ana:fre_dep_exc}{0.7}

It is seen in \autoref{fig:ana:fre_dep_exc} that the refraction effect at a distance of \SI{110}{\meter} starts at \Hz{400}. The reason that sound enters the shadow zone is not fully understood, but one theory is that the shadow boundary wave is diffuse and therefore a significant amount of sound energy enters the shadow zone by turbulent air flow. In a non-turbulent atmosphere condition the \gls{spl} inside the shadow zone is attenuated well more than \dB{30}. Close to the ground, the atmosphere condition is always turbulent because of ground friction. The turbulence wind diffuses the sound wave and changes the direction of propagation. The wave that enters the shadow zone is considered as a creeping wave while turbulent air flow is present. The creeping wave will by them self also be refracted and therefore, parallel to the other refraction waves. \citep{tur_on_sound}




\paragraph{Oblique- and crosswind} The effect of oblique- and crosswind on acoustical wave propagation in inhomogeneous atmospheric conditions is studied by the author in \citep{review_of_sound}. The author explains that the refraction is directly zero when only crosswind is present, and increase progressively as the direction of propagation deviate from the direction of crosswind. The author of \citep{no_refraction_1998} support this theory for the inhomogeneous atmospheric condition.

Since the effect of oblique wind on a line source array is rarely studied, a measurement in a windy condition is performed. The measurement is performed over mown grass in a large open area used for football. The used measurement technique is done according to \citep{gunness2001loudspeaker} where more than one impulse response is measured, and average by alining the impulse response. The wind is considered as healthy for an outdoor concert. The wind speed is measured to  \SI{14}{\meter\per\second} doing the full measurement. The measurement is done with a four element line source array \SI{1.1}{\meter} above the ground. There are used two microphones, where both are situated \SI{25}{\meter} from the line source array in the first measurements and \SI{23}{\meter} from the line source array in the last measurements. While changing the distance, the angle to the line source array is changed.  The frontal direction of the line source array is placed orthogonal to the wind direction, and the microphone is placed on both side of the line source array as shown in \autoref{fig:ana:position}

\xfig{analysis/measurement_one_sec.pdf_t}{The figure shows the microphone position versus the position of the line source array}{fig:ana:position}{1}   


The measurement is done with sine swept and according to the description in \autoref{ch:ap:measurement_one}. The measurement is performed with two microphone positions, two measurements where the microphone are within the line source array high frequency directional angle and three measurements outside the line source array high frequency directional angle. The first measurement is shown in \autoref{fig:ana:mea_two_one}. The other four measurement result is founded in \autoref{ch:ap:measurement_one}. They show the same tendency, but the difference between the measurements are more drastically in the measurement where the microphone is situated outside the high frequency directional angle. 

%\plot{plot/measurement_one_one_ana}{The graph shows the average of the first transfer function measurement routine https://asa-scitation-org.zorac.aub.aau.dk/doi/pdf/10.1121/1.381455?class=pdf}{fig:ana:mea_one_one}

\plot{plot/measurement_two_one_ana}{The graph shows the first transfer function measurement within the high frequency directional angle. The $L_{eq}$ \gls{spl} different between the microphones is \dB{4.41} (IR_3)}{fig:ana:mea_two_one}

It is seen in \autoref{fig:ana:mea_two_one} that the general \gls{spl} is higher for microphone 1. Furthermore, both measurements show ground reflection in the frequency response. Besides the ground reflection, the general level is higher in the measurement measured by microphone 1. This difference indicates refraction upwards in the direction of microphone 2.  The resulting $L_{eq}$ \gls{spl} difference for all measurement is shown in \autoref{ta:ana:spl_dif}.

\begin{table}[H]
\centering
\caption{The table shows the measured $L_{eq}$ \gls{spl} for all measurement and the difference between the microphone}
\begin{tabular}{l|l|l|l}
Measurement number &  Mic 1 $L_{eq}$ & Mic 2 $L_{eq}$ & Difference\\ \hline
        \measurement{fig:ap:mea_one_one}{1}        &  \dB{71.82}     &  \dB{66.33} & \dB{5.49} \Tstrut \\
         \measurement{fig:ap:mea_one_two}{2}      &  \dB{69.09}      &  \dB{64.69} & \dB{4.40} \\
        \measurement{fig:ap:mea_one_thr}{3}         &  \dB{67.67}     &  \dB{63.44} & \dB{4.23} \\
         \measurement{fig:ap:mea_two_one}{4}       &  \dB{68.10}      &  \dB{63.69} & \dB{4.41} \\
         \measurement{fig:ap:mea_two_two}{5}       &  \dB{68.44}      &  \dB{63.62} & \dB{4.81} \\ 
 Average &   \dB{69.02} &   \dB{64.35} &   \dB{4.67} 
\end{tabular}
\label{ta:ana:spl_dif}
\end{table}

As it is shown in \autoref{ta:ana:spl_dif}, the $L_{eq}$ \gls{spl} is higher for microphone 1 in all measurement. Moreover the average $L_{eq}$ \gls{spl} difference is \dB{4.67}, where for A-weighted $L_{Aeq}$ \gls{spl} the average difference is \dB{6.17}. With respect to the intelligibility frequency range, a weighting filter is designed to observe the \gls{spl} differences in the critical intelligibility frequency range. The filter is based on the founded intelligibility frequency range in \citep{arl_us_army}. It is shown in \citep{arl_us_army} that the critical intelligibility frequency range lays between \Hz{1000} and \Hz{4000}. The designed intelligibility weighting filter is an 8\textsuperscript{th} order bandpass filter with lower crossover frequency at \Hz{1000} and higher crossover frequency at \Hz{4000}. The resulting average difference is \dB{7.88} and the maximum difference is \dB{9.95}.



\paragraph{Turbulent} Turbulence is an atmospheric condition where the wind eddies. It often starts with large eddies and progressively brakes down like a cascade effect to smaller and smaller eddies, which only depend on the local region. When the eddies are as small as \SI{1}{\milli\meter} the energy disappears in viscosity and thermal conduction. A statistical distribution of the eddies is defined as turbulence. The turbulence wind flow is, therefore, a chaotic and stochastic process by nature and is present all the time. It occurs because of change in landscape, ground friction, stage and blockage, but also be a process of the flow speed increase in the wind, which makes the wind to refract on itself. Turbulence is often high on a windy afternoon day and low under the inverse of lapse. Turbulence often occurs near the ground because the ground surface slows down the speed of wind by the friction to the ground. The effect of turbulence on sound is known to make phase and amplitude fluctuation of pure tone. The fluctuation increases with distance until the standard divination of the phase fluctuation is comparable to \SI{90}{\degree} \citep{review_of_sound}. At this point, the phase correlation for each sound path is uncorrelated.



%\citep{lemke2017adjoint}






