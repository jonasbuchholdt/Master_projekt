\section{Ideal geometric spreading loss}\label{sec:ana:geo_spr_los}
When a line source generates a sound wave, the wave field exhibits two fundamental difference spatially directive regions, near-field and far-field. In near-field, the wave propagates as a cylindrical wave wherein the far-field the wave propagates as a spherical wave. When the wave propagates as a cylindrical wave, the wave propagates only in the horizontal plane and therefore the attenuation is \dB{3} per doubling of distance. For a spherical wave propagation, the wave propagates in all direction, therefore the attenuation is \dB{6} per doubling of distance. The near field and far-field attenuation are based on non-absorption homogeneous atmospheric conditions. The border between the near-field and far-field depends on the hight of the array and the frequency. The distance can be calculated with Fresnel formula \autoref{eq:ana:near_field}, where the wavelength $\lambda$ is approximated to $\frac{1}{3f}$ \citep{bauman2001wavefront}

\begin{equation}\label{eq:ana:near_field}
d_{B} = \frac{3}{2}f \cdot H^{2}\sqrt{1-\frac{1}{(3f \cdot H)}}
\end{equation}

\startexplain
\explain{d_{B}}{is is the distance from the array to the end of near field}{\meter}
\explain{f}{is the frequency}{\kilo\hertz}
\explain{H}{is the hight of the array}{\meter}
\stopexplain

In equation \autoref{eq:ana:near_field} it can be calculated that less than \Hz{80} radiate directly intro spherical wave on the exit of the speaker. The following \autoref{fig:ana:near_far_field} shows a horizontal cut of the near-field, far-field from a line source array. 

\fig{near_far_field}{The figure shows horizontal cut of a \gls{spl} radiation pattern of a line source array \citep{bauman2001wavefront}.}{fig:ana:near_far_field}{0.8}

\section{Homogeneous atmospheric conditions}\label{sec:ana:hom_ats_con}
The aim of this section is to analyse the sound wave propagation in homogeneous atmospheric conditions. It is well known that the sound wave propagation is highly depending on the atmospheric conditions. The propagation depends on the atmospheric pressure, wind, temperature and humanity, where the two latter moreover is frequency dependent. The attenuation difference in frequency for temperature and humanity can be above \dB{80} \citep{corteel2017large}. The following sections introduce a brief discussion of homogeneous atmospheric conditions effect on sound propagation.


\subsection{Humanity and temperature impact}\label{sec:ana:hu_temp}
The temperature and humanity have two impacts on wave propagation, speed of sound and a lowpass effect. The following description starts with the latter. 

\paragraph{Lowpass effect} The effect of humanity and temperature act as a lowpass filter, where the low frequency remains without any additional attenuation. In other words, attenuation in the high frequency range per doubling of distance depends not only on the spreading loss but also on temperature and humanity. Therefore, for long distance, the atmospheric conditions have a high influence on the frequency spectrum delivered to the audience. Humanity and temperature attenuation are already well studied and standardised. Standard \citep{iso_9613-1} gives an overview of calculating the frequency attenuation with respect to the distance, temperature and humanity. The article \citep{corteel2017large} gives some examples of attenuation at a distance of \SI{100}{\meter}. The article shows, if humanity increases proportional to the temperature, the lowpass effect is small. If the change in temperature and humanity is the opposite of each other, for example, the high temperature but dry, the attenuation in high frequency is significant. The following \autoref{fig:ana:temp_eg} shows the worst-case scenario from \citep{corteel2017large}.

\fig{temp_worst_case}{The graph shows the attenuation in \db with respect to frequency, humanity and temperature \citep{corteel2017large}.}{fig:ana:temp_eg}{0.9}
 
\paragraph{Speed of sound} The second consequence is the speed of sound. At temperature range from \SI{0}{\celsius} to \SI{40}{\celsius} the speed of sound with respect to humanity change is sparse and mostly depend on temperature change. At \SI{0}{\percent} humanity, the speed of sound only depends on the temperature. At humanity higher that \SI{0}{\percent} the sound speed increase with respect to humanity, depends on temperature. At \SI{0}{\celsius} the speed of sound increases with approximately \SI{0.8}{\meter\per\second} while for \SI{30}{\celsius} speed of sound increases with approximately \SI{2.7}{\meter\per\second} \citep{humanity_effect_on_speed} \citep{bohn1987environmental}. For only temperature differences, the speed of sound increases approximately by \SI{0.6}{\meter\per\second} for every increasing degree Celsius. The wave propagation speed start at \SI{331.5}{\meter\per\second} at \SI{0}{\celsius} and \SI{0}{\percent} humanity. The following \autoref{fig:ana:hu_speed} shows the speed of sound with respect to humanity and temperature. 

\plot{code/speed_hum_temp}{The figure shows the increese of sound speed with respect to humanity and temperature \citep{bohn1987environmental}}{fig:ana:hu_speed}


\subsection{Wind impact}
The wind impact is complex and is not homogeneous with respect to sound source. The impact is depending on the angle of the wind direction with respect to the direction of sound propagation. 


\paragraph{Parallel to sound propagation} When the wind flows in the same direction as the sound wave propagation, the wind flow in \si{\meter\per\second} is an addition to the speed of sound. When the wind flows in the opposite direction it is a negative addition.  In other cases, the influence is complex since the wind deflect the sound waves.

\paragraph{oblique- and crosswind} The effect of oblique- and crosswind on sound wave propagation is rarely studied, and the effect seems to be unclear. Few author have addressed the problem in a simulation of traffic noise and by practical experience \citep{effect_of_wind}, \citep{crosswind_effect_2016}, \citep{BALLOU2008xi}. They claim that the crosswind effect refracts the wave in the wind direction. Furthermore, they claim that the effect is not linear in frequency. The author of \citep{BALLOU2008xi} indicates that the frequency dependency might be due to the directionality of the high frequency drivers. Since the study of crosswind effects is rare, a geometric calculation of the effect is developed. The geometric calculation will only be done for crosswind which mean that the wind is orthogonal the the frontal direction of the speaker. It will be based on the directional speaker with \SI{90}{\degree} angle, so \SI{45}{\degree} angle from frontal direction as the \autoref{fig:ana:spe_ang} shows.


\xfig{analysis/speaker_angle.pdf_t}{The figure shows a geometrical interpolation of a wave moved by the wind}{fig:ana:spe_ang}{0.65}


The calculation will calculate the refraction of the two outer wave angle in \autoref{fig:ana:spe_ang}. For simplicity the outer wave angle direction is defined as wave vector. It is assumed that the speed of sound is \SI{343}{\meter\per\second} with \SI{20}{\celsius}, dry and no influence of wind. A interpolation of the wave vector refraction with present of crosswind is that the time the wave uses for travelling \SI{1}{\meter} can be used to find the how much the wind have mowed the air particle in the wind direction. The following \autoref{eq:ana:wind_turn} calculated the cross movement of the air particle as the wave have travelled  \SI{1}{\meter}.

\begin{equation}\label{eq:ana:wind_turn}
d = v \cdot \frac{1}{c}
\end{equation} 

\startexplain
\explain{d}{is the cross distance the wave have moved in the wind direction after the wave have travelled \SI{1}{\meter}}{\meter}
\explain{v}{is the speed of wind}{\meter\per\second}
\explain{c}{is the speed of sound}{\meter\per\second}
\explain{1}{is the travelling distance of the wave}{\meter}
\stopexplain


The cross distance is defined as the wind vector. The wave speed depend on the angle the the wind, therefore the wave speed will change with the refraction. The following \autoref{eq:ana:wind_angle} calculate the speed of sound with respect to the angle to the wind 


\begin{equation}\label{eq:ana:wind_angle}
c = c_r+v \cdot sin(\theta)
\end{equation}

\startexplain
\explain{\theta}{is the angle of the wave with respect to the wind}{\degree}
\explain{c_r}{is the reference speed of sound with \SI{20}{\celsius}, dry and no influence of wind}{\meter\per\second}
\stopexplain

The wind gradient with present of wind after the wave have travelled \SI{1}{\meter} became as following {eq:ana:wind_turn}.

\begin{equation}\label{eq:ana:wind_turn}
d = v \cdot \frac{1}{c_r+v \cdot sin(\theta)}
\end{equation} 

As the wave refract in the direction of the wind, the wave speed increases. Therefore the travelling time for \SI{1}{\meter} decay which do that the travelling distance of the wind vector decays. In the mean time the frontal direction will also decay.  A discrete geometrical interpolation of of the descried phenomena is shown in  \autoref{eq:ana:wind_turn}.


\xfig{analysis/geo_of_crosswind.pdf_t}{The figure shows a geometrical interpolation of a wave moved by the wind}{fig:ana:geo_cross}{0.62}


To calculate the resulting wave vector, the following \autoref{fig:ana:cal_wav_dir} illustration of the wave refraction is used.


\xfig{analysis/cal_of_crosswind.pdf_t}{The figure shows a geometrical calculation scheme of calculate the resulting wave direction at crosswind}{fig:ana:cal_wav_dir}{0.5}
\startexplain
\explain{d}{is the resulting wind vector with a wave length of \SI{1}{\meter}}{\meter}
\explain{w}{is the resulting wave vector}{\meter}
\explain{a}{is the wave vector without wind}{\meter}
\explain{\textcolor{red}{p}}{is the orthogonal projection of the wind vector intro $a$ }{\meter}
\explain{e}{is the projection \textcolor{red}{$p$} added with the wind vector $d$ }{\meter}
\explain{\phi}{is the angle between $a$ and $w$ }{\degree}
\stopexplain


The procedure of calculating the resulting wave vector $w$, is to project the wind vector $d$ intro the direction of the wave vector $a$ without wind. With the projector \textcolor{red}{$p$} the orthogonal $e$ in \autoref{fig:ana:cal_wav_dir} can be calculated with simple addition. The length of $a$ is unknown but can be calculated within the unit circle, since the length of the resulting wave vector always is \SI{1}{\meter}. By taking $argsin$ to the length of $e$, the length of $a$ is calculated. The resulting wave direction is then an addition between $a$ and $e$. 

The used calculation script is as \autoref{code:ana:spe_cov}.

\includeCode{crosswind_geo.m}{matlab}{26}{33}{The speaker coverage area simulation code}{code:ana:spe_cov}{./code/}


A simulation of the speaker coverage area is shown in \autoref{plot:ana:spe_cov_area}

\plot{plot/speaker_cov}{The graph shows the coverage area of two speaker with \SI{90}{\degree} angle and a crosswind from left with a speed of \SI{10}{\meter\per\second} and a wave propagation time of \SI{189.3}{\milli\second}}{plot:ana:spe_cov_area}


It can be seen in \autoref{plot:ana:spe_cov_area} that the coverage area distance is reduces to \SI{53}{\meter} when the speaker have an angle of \SI{90}{\degree} and the the wind speed is \SI{10}{\meter\per\second}. Another important aspect which is changed is the wave travelling distance from the speaker to the audience. Without wind the distance along the first ariavle wave is  \SI{53}{\meter}. When the wind is present, the sound ariving to the audiance at \SI{53}{\meter} is the wave vector, which have traveled longer that the wavev arriving to the audiance without wind. The wave traveling is increased by \SI{6}{\meter}. This increase in length introduce attenuation in the \gls{spl} arrival to the audience.

%refraction that's a direct result of Snell's law

\subsection{Pressure impact}
The influence of atmospheric pressure change is low compared to the effect of wind, humanity and temperature. The average attenuation from \Hz{4000} to \Hz{16000} with fixed temperature was \dB{2} while going from \SI{54.02}{\kilo\pascal} to \SI{101.33}{\kilo\pascal}. The atmospheric pressure then only have a negligibility influence on sound propagation and is generally not frequency dependent. 
 
 
\subsection{Ground absorption} 
In a concert area, ground absorption is complex. The concert area doing the concert is packed by audience, and is therefore not easy to calculate. On the other hand, the line source array is flown and therefore positioneded higher than the audience. The reflection of the high frequency is assumed to be neglible and therefore full absorbtion of the audience. The low frequency driver, also called subwoofer is posisioned in front of the stage on a line, often with a maximum distance of \SI{2.8}{\meter} from acoustical center to acoustical center. The distance between the low frequency driver is determinde by the half wavelength of the highest frequency, such that they radiate a plan wave \citep{bauman2001wavefront}. Higher distance between acoustical center will cause interference in the low frequency in the audience area. The absorption from the audience in the low frequency is assumed to be low since the size of the audience is much smaller than the wave length.   


 
 \subsection{Homogeneous speed equation}\label{sec:ana:inhom_ats_con}
 The following \autoref{eq:ana:wind} calculate the speed of sound based on homogeneous temperature and wind speed.

\begin{equation}\label{eq:ana:wind}
c =  c_0 \sqrt{1+t/t_0} + u \cdot sin(\theta)
\end{equation}  

\startexplain
\explain{c}{is the resulting speed of sound}{\meter\per\second}
\explain{u}{is the speed of wind}{\meter\per\second}
\explain{c_0}{is the speed of sound at \SI{0}{\celsius}}{\meter\per\second}
\explain{t}{is the temperature}{\celsius}
\explain{t_0}{is the temperature at \SI{0}{\celsius} (273.15)}{\kelvin}
\explain{\theta}{is the angle of wind with respect to the wave propagation}{\degree}
\stopexplain


 
\section{Inhomogeneous atmospheric conditions} 
The aim of this section is to analyse the sound wave propagation in inhomogeneous atmospheric conditions. In an inhomogeneous atmosphere, the pressure and speed is a function of position. By this fact, the modelling of a sound wave is very complex and depend on various variables such as temperature and wind speed. The following sections give a short introduction to the effect of inhomogeneous atmospheric conditions. 
 
 
\subsection{Atmospheric refraction}
When the speed of the wind, the temperature and humanity is assumed to be homogeneous in the sound field, the sound is travelling in a straight path. Often this is not the case, the wind speed increases logarithmically with the hight from the ground to the geostrophic wind \citep{asmos_acous_2016} and the temperature and humanity are inhomogeneous. The geostrophic wind in founded from approximately \SI{1}{\kilo\meter} above the ground \citep{geostrophic_wind}.  In such a situation the change of sound wave propagation is directly caused by the atmospheric temperature or wind. This often results in a curved path of the sound wave and is defined as atmospheric refraction. For small distances, the atmospheric refraction has a spars effect on the sound travelling path, because the speed of sound is much faster than the speed of the wind and the temperature. Generally distance up to \SI{100}{\meter} is often assumed to have no significant refraction effect \citep{effect_of_wind}. For distances larger than \SI{100}{\meter} the refraction is assumed to have a significant influence, especially when the sound source and the receiver are close to the ground.  


\paragraph{Temperature} The refraction occur because of the temperature change with respect to hight along the day. The sun heats the ground, the and the concert area is full of audience. Therefore, the eath and audience radiate warm air, which makes the temperature at a low hight warmer than the temperature at higher hight. As explained in \autoref{sec:ana:hu_temp}, the speed of sound depends on the temperature and therefore, the speed of sound in this situation decay with respect to hight and result in an upwards refraction. The following \autoref{fig:ana:temp_ref} illustrate the phenomena where the temperature decay with respect to the hight.

\xfig{analysis/refraction_temp.pdf_t}{Wave refraction in inhomogeneous temperature}{fig:ana:temp_ref}{1}

The sound refraction will be identically all around the source for a omnidirectional source with respect to temperature.


\paragraph{Wind} With respect to the wind speed, a concert area is often a protected area with for example barrier, stage and building. This blockage slows down the wind speed close to the ground. Moreover, from nature itself, the wind speed is often logarithmically increased with respect to the hight. When the wave is propagation in the same direction as the wind, the atmospheric refraction refracts the sound wave downwards. When the wave propagates against the wind, the atmospheric refraction refracts the sound wave upwards. The following \autoref{fig:ana:wind_ref} shows the phenomena when the wave propagates against the wind.

\xfig{analysis/refraction.pdf_t}{Wave refraction in inhomogeneous wind}{fig:ana:wind_ref}{1}

Due to the wind refraction effect, the sound propagates faster at the ground under the described condition. The consequence is a change of wave direction. This upwards refraction creates a shadow zone in the audience area \citep{asmos_acous_2016}. In this shadow zone, the \gls{spl} is very low and the audience intelligibility is dramatically decreased. The following \autoref{fig:ana:shadow_zone} shows the phenomena.


\xfig{analysis/shadow_zone.pdf_t}{The figure illustrate the shadow zone ocourse from a upwards refraction. A line source speaker array contains of many couplet point sources. Every lowest sound path dashed line indicate the lower directional angle of one poins source in the line source array.}{fig:ana:shadow_zone}{0.73}

As shown in \autoref{fig:ana:shadow_zone} the refraction is upwards when the wind flows in the opposite direction as the wave propagation. Behind the line array source, the refraction is downwards and is therefore different than for temperature refraction.

\paragraph{Oblique- and crosswind} The effect of oblique- and crosswind on acoustical wave propagation in inhomogeneous atmospheric conditions are rarly studied.  The author was not able to find any relevant paper on the subject.

\paragraph{Turbulent} Turbulence is a atmospheric condition where the wind does not flow continuesly from one direction, but fluctuates from all directions. Fluctuation often occur in scales of hour, minutes and second where the latter is defined as turbulence. The turbulence wind flow is a chaotic and stochastic process by the nature. It can occur because of change in landscape for example bulding stage and blockage, but can also be a process of flow speed increase in the wind, which make the wind to refract on itself. Turbulence occur often near the ground because the ground surface slow down the speed of wind by the resistance to the ground. This ground resistance also force the air to be turbulent.




%https://www.sweetwater.com/insync/effects-of-wind-live-sound/

%https://fenix.tecnico.ulisboa.pt/downloadFile/395144345754/dissertacao.pdf

%http://www.hk-phy.org/iq/sound_night/sound_night_e.html

%https://physics.stackexchange.com/questions/341704/why-does-wind-direction-significantly-affect-sound-propagation

%\citep{lemke2017adjoint}






