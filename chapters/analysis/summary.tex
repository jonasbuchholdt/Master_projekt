The analysis started addressing the generally used method for a live concert. It is founded that live concert today use line source array system to cover the audience area with sound. The line source array is flown above the audience at the main stage, and the delay speaker covers the back audience at a large concert. The line source array is constructed of many identical speakers attached to each other in a vertical line. Moreover, the distance from the speaker to the individual audience depends on the audience position. The analysis founded that a homogeneous \gls{spl} among all audience might not possible but the \gls{spl} among all audience can be optimised by knowledge of the condition of the atmosphere and gain up for the spreading lose. The author observes that the wind does have a frequency and distance-dependent effect on sound propagation, for example at high frequency the high frequency attenuate audibly in the crosswind. The high frequency blows away for periods and comes back again. 
The analysis of sound from a line source array started by the ideal geometric spreading loss. Here it is founded that the sound propagation of the line source array highly depends on the hight of the source. The line source array propagates differently with respect to frequency. At a certain hight of the line source array the propagation is a cylindrical propagation until a certain distance from the source where it starts propagating as a spherical source. In the cylindrical propagation, the sound field is defined as near-field while in the spherical propagation the sound field is defined as far-field. In the non-ideal scenario, the line source array propagates in inhomogeneous atmospherical condition. To cover the inhomogeneous atmospherical condition, the local homogeneous atmospherical condition is analysed. In the homogeneous atmospherical condition, it is founded that the temperature, humidity, pressure and wind influence the sound field. The effect of temperature and humidity is close coupled on sound propagation. When the temperature is high, and the humidity is low the air has a significant high frequency absorption whereas when the temperature and humidity follow each other, the absorption is less. The second effect the temperature and humidity have on sound propagation is the speed of sound. The higher the temperature is, the higher the sound of speed. The humidity affects the speed of sound the same way as the temparature, but the increase is negligible compared to the temperature. The effect of wind seems to have a sparse effect on the sound propagation when and only when the wind is homogeneous. It is founded that the speed of wind affects the speed of sound. If the wind moves in the direction of the sound propagation the wind speed is an addition to the speed of sound. In the opposite wind case, the speed of sound is lowered. In the case of oblique- or crosswind, the effect seems to be unclear for high frequencies. One author has simulated a low frequency spherical source and founded that the only effect is the time of arrival to the audience.  The impact of the atmospheric pressure is small, and the pressure close to the ground is so high that other limitations of wave propagation limit the \gls{spl} before the negative amplitude riches vacuum in the condensation. When the wave compresses the air, the wave travels faster such that the received wave at the audience is a sawtooth wave. The effect produces harmonic distortion where some of the harmonic energy is attuned be the viscous losses. The harmonic distortion is present in \gls{spl} lower than \dB{120} but is not as critical as the distortion created by the construction of the speaker enclosure. 
The audience area is assumed to have high absorption in frequency above \Hz{1000}, while frequency in octave band \SI{31.5}{\hertz} is assumed to have low absorption of the audience. 


In the inhomogeneous atmospherical condition, it is founded that refraction of the sound wave is one of the biggest challenges for an outside sound concert. The refraction occurs because of inhomogeneous speed which is present in both inhomogeneous wind and temperature. It is further founted that the refraction is frequency dependent and distance dependent. The effect, however, is low at a distance lower than \SI{50}{\meter} with a wind speed of \SI{5}{\meter\per\second}. Depending on the atmospheric condition two kinds of refraction was founded, upwards and downwards. Upwards refraction produces a shadow zone where turbulent atmospheric condition makes creeping wave intro the shadow zone. For the case of oblique and crosswind the effect of high frequency, the refraction might be zero at direct crosswind but increases progressively as the direction of propagation deviate from crosswind. One measurement was done to research the effect of crosswind on a line source array. It was founded that the average $L_{Aeq,5}$ \gls{spl} at microphone 1 was \dB{6.17} higher than microphone 2. Therefore it can be concluded that the crosswind with respect to the speaker coverage area does affect. 
 

  

