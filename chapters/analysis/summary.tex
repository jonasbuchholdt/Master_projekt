
The analysis started addressing the general used method for live concert. It is found out that love concert today use line source array system to cover the audience area with sound. It is observed that a line source array is flown above the audience at the main stage and at a large concert delay tower line source array is used and placed in the middle of the audience. It is founded that a line source array is constructed of many identically speaker attached to each other in a vertical line. Based on the speaker set up the distance from every audience to the speaker depends on the position in the concert area. Therefore the distance from the speaker and the audience depends on the audience and therefore the sound system have to be optimized such that the \gls{spl} coverage is equaly in all position. The following analysis in ... founded that a homogenious \gls{spl} among all audience is an imposible senario but the \gls{spl} among all audience might be possible to optimized by knowledge of the condition of the atmosphere and make up for the spredding lose. 


The experence of the author is included as knowledge for the analysis. It is observed by the author that the wind condition at a concert has a large influence of the experience of the concert. At the low frequency the influence is sparse where for high frequency the effect of the atmospherical condition have a big influence especially when the distance from the speaker is increesed. The figh frequency blowes away for periods ad comes back again. 

A short introduction to the stereo sweet spot problem is also discused. It is a hot topic today among some of the large speaker factorys but the solution require heavily amount of speakers and flying tools. 


The analysis of sound from a line source array started by the ideal geometric spreading loss. Here it is founded that the sound propagation of the line source array highly depend on the hight of the source. The line source array propagate differently with respect to frequency. At a surtain hight of the line source array the propagation is a cylendrical propagation until a surtain distance from the sorce where it starts propagating as a espherical source. In the cylendrical propagation the sound filed as defined as near-field while in the spherical propagation the sound field is defined as far field. 

In the non ideal senario the line source array propagate in ether a homogeneous atmospherical condition or in an inhomogeneous atmospherical condition. 


In the homogeneous atmospherical condition it is founded that the temperature, humidity, pressure and wind influence the sound field. The effect of temperature and humidity is close cubled on sound propagation. When the tamperature is high and the and the humidity is low the air have an large high frequency absorbsion whereas when the temperature and humidity both is eather high or low the  absorbsion is much less. The second effect the temperature and humidity have on sound propagation is the speed of sound. The higher the temperature is the higher the sound of speed is. At \SI{0}{\percent} humidity the speed of sound incerasses with approximatli \SI{0.6}{\meter\per\second} for every \si{\celsius} increase. The effect of humidity also effect the speed of sound by increaseing speed of sound by incerasing humidity but the speed incerse is neglible compare to the temperature effect. Thudermore effect of speed difference in sound propagation change the wavelendth and therefore the directivity of the speaker is changed, but the change is minimal. 

The effect of wind seems to have a sparse effect on the sound propagation when and only when the wind is homogeneous. It is founded that the speed of wind effect the speed of sound. if the wind mowes in the direction of the sound propagation the wind speed is and addition to to speed of sound to find the resulting propagation of speed of sound. In the uppesite cace the speed of sound is lowered. In the case of oblique- or crosswind the effect seems to be unclear for high frequencies. One author has simulated a low frequency spherical source and founded that the only effect is the time of ariavle to the audience.   


The impact of the atmospheric pressure is small, and the pressure close to the ground is so high that other limitation of wave propagation limite the \gls{spl} before the \gls{spl} ritches vacuum in the condensation. When the wave compress the air the wave travelse faster such that the resived wave at the audience is transformed from a sinosoid to a sawtooth wave. The effect produce harmonic distortion where some of the hamonic energy is attunatied be the viscus losses. The distortion is present in \gls{spl} lower than \dB{120} but is not as critical as the distortion created by the construction of the horn itself. The construction of the horn leave very litllel air space within the airgab between the diaphram and the pahse plug. At high level at the phase plug at approximatly \dB{170} the air gets turbulent and the soundwave therefore gets distorted.

The audience area is assumed to have high absorbtion in frequency above \Hz{1000} and the while frequency in octave band \SI{31.5}{\hertz} is assumed to have low absorbtion of audience.  


In the inhomogeneous atmospherical condition it is founded that refraction of sound wave is one of the biggest challenge for outside sound concert. The refraction orcore because of inhomogeneous speed which is pressent in both inhomogeneous wind and temperature. it is forther founted that the the refraction is frequency dependent and distance dependent. The effect, however, is low at distance lower than \SI{50}{\meter}. Depending on the atmospehric condition two kind of refraction was founded, upwards and downwards. Upwards refraction produce a shadow zone where turbulent atmospheric condition makes creeping wave intro the shadow zone. 


For the case of oblique and crosswind the effect of high frequency, The refraction might be zero at direct crosswind but increases prograsiv as the direction of propagation diviate from crosswind. A measurement was done to resurch the effect of crosswind on a line source array. It was founded that the average $L_{Aeq,5}$ \gls{spl} at microphone 1 was \dB{6.17} higher than microphone 2. Therefore it can be concluded that the crosswind with respect to the speaker corverage area does have an effect. 
 

  

(remember to write short about interference while playing mono from stereo setup)    


The effect of 

Three effect of atmospheric conditions have been observed on the analysis, pure attenuation, lowpass effect and refraction effect
