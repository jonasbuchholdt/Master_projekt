\chapter{Test a guitars frequency area}
A measurement was made to measure the transfer function differences in two point in cross wind.


\section*{Materials and setup}
To measure the .transfer function in windy weather, the following materials are used:

\startequipment
\equipment{PC}{Macbook}{W89242W966H}{-}
\equipment{Audio interface}{RME Fireface UCX}{23811948}{108230}
\equipment{Microphone}{GRAS 26CC}{78189}{75583}
\equipment{Preamp}{GRAS 40 AZ}{100268}{75551}
\equipment{Microphone}{GRAS 26CC}{78186}{75582}
\equipment{Preamp}{GRAS 40 AZ}{100267}{75550}
\equipment{dB technologies}{T4}{-}{-}
\equipment{flying tools}{-}{-}{-}
\stopequipment

\xfig{measurement_one.pdf_t}{The figure shows the microphone position vurses the position of the line source}{fig:ap:position}{1}

\section*{Test procedure}


\begin{enumerate}
\item The materials are set up as in \autoref{fig:ap:position} where the microphone and speaker is connected to the audio interface.
\item 10 sine sweep is performed with a length of \SI{5}{\second} each 
\item  The correlation is calculated for each impulse response to the first impulse response for time alignment.  
\item  
\item 
\item 
\end{enumerate}

\section*{Results}

\begin{figure}[htbp!]
	\centering
		\includegraphics[width=1\textwidth]{guitar_low_E_neck.pdf}
		\caption{Measurement of the low E note on the neck pickup.}
		\label{fig:appendix:low_E_neck}
\end{figure}

On  \autoref{fig:appendix:low_E_neck} it is seen that the lowest significant frequency is around \SI{80}{\hertz} and the highest significant frequency is around \SI{400}{\hertz}, when playing the low E note on the guitar, using the neck pickup.

