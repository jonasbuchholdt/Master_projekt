\chapter{Wind noise attenuation of windscreen}\label{ap:wind_noise_att}
A measurement is made to measure the wind attenuation of difference windscreen configuration. All configuration includes the GRAS AM0069 windscreen with an additional wind stopper surface all around the microphone except the frontal direction. The measurement is done as a preliminary test with low wind speed to test the concept. 


\section*{Materials and setup}
To measure the wind attenuation of the windscreen configuration the following materials are used:

\startequipment
\equipment{PC}{Macbook}{W89242W966H}{-}
\equipment{Audio interface}{RME Fireface UCX}{23811948}{108230}
\equipment{Microphone}{GRAS 26CC}{78189}{75583}
\equipment{Preamp}{GRAS 40 AZ}{100268}{75551}
\equipment{Microphone}{GRAS 26CC}{78186}{75582}
\equipment{Preamp}{GRAS 40 AZ}{100267}{75550}
\stopequipment
\startequipment
\equipment{Fan}{IMPEGA}{-}{-}
\equipment{Fan}{IMPEGA}{-}{-}
\equipment{Windscreen}{GRAS AM0069}{-}{-}
\equipment{Windscreen}{GRAS AM0069}{-}{-}
\equipment{large foam wedge}{-}{-}{-}
\equipment{Small foam wedge}{-}{-}{-}
\equipment{Rockwool bat}{-}{-}{-}
\stopequipment

\xfig{appendix/frequency_response_in_windscreen.pdf_t}{The figure shows the measurement setup for the wind noise measurement in the microphone position and outside the windscreen. The two microphone seems to lay onto each other but it shall show that one is inside the windscreen and one is outside the windscreen}{fig:ap:wind_meas_setup}{1}

\fig{IMG_1153}{The picture shows the measurement set up}{fig:ap:pre_att_setup_test}{0.5}

\section*{Test procedure}


\begin{enumerate}
\item The materials are set up as in \autoref{fig:ap:wind_meas_setup} where the two microphone connected to the audio interface.
\item Both microphones are calibrated.
\item Both fans is activated 
\item A \SI{7}{\second} time signal is measured three times synchronise on both microphones.
\item The frequency content is calculated by \texttt{fft} on all six measured time signal.
\item The average of the frequency response for each microphone is calculated
\item The difference between the microphone is calculated to find the attenuation of the windscreen configuration
\item The procedure is done for all windscreen configuration and one where no additionally wind stopper is added around the microphone. This last configuration is defined as the reference configuration.
\item A no wind measurement is measured the same way just without the fan activated and only with GRAS AM0069 windscreen in the end.
\item The wind speed is measured.
\end{enumerate}

\section*{Measurement area}
To be able to generate a controlled wind flow, the hallway in Fredrick Bajers vej 7B5, 9200 Aalborg is used. The following \autoref{ap:wind:flow_slow} shows a drawing of the area and the position of the fan and windscreen.

\fig{wind_meas_pos_slow}{The picture illustrate the area, where the wind flow is measured}{ap:wind:flow_slow}{0.7}

\section*{Results}
The following graphs show the result of the measurements. The wind speed is measured to be \SI{2.5}{\meter\per\second}. 


\plot{plot/with_ball_no_wind}{The graph shows the frequency content without the fan activated}{fig:ap:att_wit_bal_no_vin}

The \autoref{fig:ap:att_wit_bal_no_vin} shows the frequency content in the measuring area without the fan activated for both microphone and the reference windscreen configuration.

\plot{plot/with_ball_with_wind}{The graph shows the frequency content with the fan activated}{fig:ap:att_wit_bal_an_vin}
The \autoref{fig:ap:att_wit_bal_an_vin} shows the frequency content in the measuring area with the fan activated for both microphone and the reference windscreen configuration. It is seen that the highest attenuation is at \Hz{900} but the general attenuation is 



\fig{IMG_1150}{The picture shows the microphone covered with windscreen and the Small foam wedge windscreen configuration. This configuration is defined as configuration one}{fig:ap:pre_att_setup}{0.5}
\plot{plot/with_ball_with_small}{The graph shows the frequency content of the measurement with configuration one}{fig:ap:att_wit_bal_sma_kil}
\plot{plot/time_with_ball_small_kile}{The graph shows one of the time measurement with configuration one}{fig:ap:tim_wit_bal_sma_kil}
The \autoref{fig:ap:att_wit_bal_sma_kil} and \autoref{fig:ap:tim_wit_bal_sma_kil} shows the measurement with configuration one in frequency and time domain respectively. It can be seen that the general windscreen attenuation is not lowered compared to the reference windscreen measurement.



\fig{IMG_1156}{The picture shows the microphone covered with windscreen and the large foam wedge windscreen. This configuration is defined as configuration two.}{fig:ap:pre_att_setup_two}{0.5}
\plot{plot/with_ball_with_large}{The graph shows the frequency content of the measurement with configuration two}{fig:ap:att_wit_bal_lar_kil}
\plot{plot/time_with_ball_large_kile}{The graph shows one of the time measurement with configuration two}{fig:ap:tim_wit_bal_lar_kil}
The \autoref{fig:ap:att_wit_bal_lar_kil} and \autoref{fig:ap:tim_wit_bal_lar_kil} shows the measurement with configuration two in frequency and time domain respectively. The measurement shows that the windscreen attenuation does have an effect compare to the reference windscreen measurement. The attenuation is nearly greater for all frequency especially in the low and high frequency range.


\fig{IMG_1155}{The picture shows the microphone covered with windscreen and the Rockwool wind stopper. This configuration is defined as configuration three.}{fig:ap:pre_att_setup_three}{0.5}
\plot{plot/with_ball_with_rock}{The graph shows the frequency content of the measurement with configuration three}{fig:ap:att_wit_bal_roc_kil}
\plot{plot/time_with_ball_rock_kile}{The graph shows one of the time measurement with configuration three}{fig:ap:tim_wit_bal_roc_kil}
The \autoref{fig:ap:att_wit_bal_roc_kil} and \autoref{fig:ap:tim_wit_bal_roc_kil} shows the measurement with configuration three in frequency and time domain respectively. The measurement shows that the windscreen attenuation does affect compare to the reference windscreen measurement, but the attenuation is not as good as in configuration two. There is better attenuation is the low frequency compared to the reference windscreen measurement, but in the high frequency, the attenuation is worse than the reference windscreen measurement, that might be due to reflection on the surface of the Rockwool.
