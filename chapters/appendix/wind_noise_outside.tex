\chapter{Wind noise in designed windscreen}\label{ap:wind_noise_in_design}
A measurement was made to measure the wind attenuation of the designed windscreen. All measurement include the modified GRAS AM0069 windscreen. The measurement is done in real senario outside on a flat area with above \SI{5}{\meter\per\second} wind speed. The measurement is done to ensure that the measured wind does not overload the preamp of the microphone at the measured wind speed.  


\section*{Materials and setup}
To measure the wind attenuation of the windscreen configuration the following materials are used:

\startequipment
\equipment{PC}{Macbook}{W89242W966H}{-}
\equipment{Audio interface}{RME Fireface UCX}{23811948}{108230}
\equipment{Microphone}{GRAS 26CC}{78189}{75583}
\equipment{Preamp}{GRAS 40 AZ}{100268}{75551}
\equipment{Windscreen}{GRAS AM0069}{-}{-}
\equipment{large foam wedge}{-}{-}{-}
\stopequipment
\startequipment
\stopequipment

\xfig{appendix/frequency_response_in_windscreen.pdf_t}{The figure shows the measurement setup for the wind noise measurement in the microphone position and outside the windscreen. The two microphone seems to lay onto each other but it shall show that one is inside the windscreen and one is outside the windscreen}{fig:ap:win_noi_out_set}{1}

\fig{IMG_0138}{The picture shows the measurement set up}{fig:ap:win_noi_out_pic_set}{0.5}

\section*{Test procedure}


\begin{enumerate}
\item The materials are set up as in \autoref{fig:ap:win_noi_out_set} where the two microphone connected to the audio interface.
\item The microphone is calibrated.
\item A \SI{7}{\second} time signal is measured.
\item The frequency content is calculated by \texttt{fft} on all six measured time signal.
\item The average of the frequency response for each microphone is calculated
\item The difference between the microphone is calculated to find the attenuation of the windscreen configuration
\item The procedure is done for all windscreen configuration and one where no additionally wind stopper is added around the microphone. This last configuration is defined as reference configuration.
\item A no wind measurement is measured the same way just without the fan activated and only with GRAS AM0069 windscreen in the end.
\item The wind speed is measured.
\end{enumerate}

\section*{Measurement area}
To be able to measure in a windy area, the football stadium at Fredrick Alfred Nobels Vej 7, 9220 Aalborg is used. The following \autoref{ap:wind:flow_mes_are} shows a picture of the area and the approximate position of the speaker and microphone.

\fig{wind_nois_area}{The picture illustrate the area, where the measured is done}{ap:wind:flow_mes_are}{0.7}

\section*{Results}

The following graphs shows the result of the measurement. 

\plot{plot/windnoise_without_screen}{The graph shows the frequency content of the measurement without the windscreen}{fig:ap:windnoise_without_screen}




\plot{plot/windnoise_with_screen}{The graph shows the frequency content of the measurement without the windscreen}{fig:ap:windnoise_with_screen}

Based on the observed ground reflection shown in \autoref{} and the high differences founded in the above measurement, only the measurement on the ground is shown in the rest of the measurement. The measurement in ear hight is chosen to be irrelevant since the measurement for the final test is done with microphone on the ground. The measurement shown next is also done in the hight of the ear and can be founded in the attached file.



\plot{plot/windnoise_with_screen_on_ground}{The graph shows the frequency content of the measurement without the windscreen}{fig:ap:windnoise_with_screen_on_ground}

\plot{plot/windnoise_with_screen_rotation}{The graph shows the frequency content of the measurement without the windscreen}{fig:ap:windnoise_with_screen_rotation}


%\fig{IMG_1150}{The picture shows the microphone covered with windscreen and the Small foam wedge windscreen configuration. This configuration is defined as configuration one}{fig:ap:pre_att_setup}{0.5}

%\plot{plot/time_with_ball_small_kile}{The graph shows one of the time measurement with configuration one}{fig:ap:tim_wit_bal_sma_kil}
%The \autoref{fig:ap:att_wit_bal_sma_kil} and \autoref{fig:ap:tim_wit_bal_sma_kil} shows the measurement with configuration one in frequency and time domain respectively. It can be seen that the general windscreen attenuation is not lowered compared to the reference windscreen measurement.



%\fig{IMG_1156}{The picture shows the microphone covered with windscreen and the large foam wedge windscreen. This configuration is defined as configuration two.}{fig:ap:pre_att_setup_two}{0.5}
%\plot{plot/with_ball_with_large}{The graph shows the frequency content of the measurement with configuration two}{fig:ap:att_wit_bal_lar_kil}
%\plot{plot/time_with_ball_large_kile}{The graph shows one of the time measurement with configuration two}{fig:ap:tim_wit_bal_lar_kil}
%The \autoref{fig:ap:att_wit_bal_lar_kil} and \autoref{fig:ap:tim_wit_bal_lar_kil} shows the measurement with configuration two in frequency and time domain respectively. The measurement shows that the windscreen attenuation does have an effect compare to the reference windscreen measurement. The attenuation is nearly greater for all frequency especially in the low and high frequency range.




\section*{Summary}
