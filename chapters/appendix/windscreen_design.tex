\chapter{Design of windscreen}\label{ap:desig_screen}
The idea of an additional windscreen is to stop the wind in just at the microphone position with a blocking and non-reflecting surface. The surface shall, therefore, be able to lower the wind speed at the microphone position and have less reflection as possible. The original windscreen is kept on the microphone.

The first two windscreen concept is very identical but just with different size of the material. The idea for the first windscreen is to seal the microphone with foam all around except at the frontal direction. The frontal direction include both \SI{180}{\degree} angle in the vertical direction and \SI{90}{\degree} in the horizontal direction.  The reason to have a narrow horizontal opening is to be able to get sound inside the opening but still, have a wind-stopping effect. The following \autoref{fig:td:mes_foa_con} illustrate both windscreen configuration one and windscreen configuration two, just with one size foam wedge.
    
\xfig{design/measure_foam_concept.pdf_t}{The figure shows the foam wedge concept. The concept is covering over to different foam wedge, ether two small or too large. The small concept is defined as windscreen configuration one, where the large concept is defined as windscreen configuration two.}{fig:td:mes_foa_con}{1} 

The next concept build on the concept in \autoref{fig:td:mes_foa_con} just with plan surfaces rockwool plates. The opening is also \SI{180}{\degree} angle in the vertical direction and \SI{90}{\degree} in the horizontal direction. The concept is defined as windscreen configuration three. The following \autoref{fig:td:mes_roc_con} illustrate the concept.

The next concept builds on minimizing the reflection from the additional windscreen by only placing the microphone close against one surface, which covers for the wind noise. The concept is defined as windscreen configuration four. The following \autoref{fig:td:mes_roc_sin_con} illustrate the concept.




\begin{figure}[H]
    \centering
     \captionsetup{width=1\linewidth}
    \begin{minipage}{0.46\textwidth}
        %\centering
         \captionsetup{width=0.90\linewidth}
       \input{figures/design/measure_rock_consept.pdf_t}
        \caption{The figure shows the rockwool concept. This concept is defined as windscreen configuration three.}
        \label{fig:td:mes_roc_con}
    \end{minipage}%
    \begin{minipage}{0.46\textwidth}
        \centering
         \captionsetup{width=0.90\linewidth}
        \input{figures/design/measure_rock_single_consept.pdf_t}
        \caption{The figure shows the single rockwool concept. This concept is defined as windscreen configuration four.}
        \label{fig:td:mes_roc_sin_con}
    \end{minipage}
\end{figure}


 

Before the optimal windscreen configuration is founded, an optimality criterion is defined, and a test is designed. The optimal criteria for the windscreen are as low wind noise as possible at the microphone position and low sound reflection. To find the windscreen configuration which meets the criteria best, three tests are made on the windscreen configuration. First, the wind speed attenuation of the windscreen configuration is measured to ensure that the windscreen configuration concept does affect the wind speed. Secondly, the frequency response of the windscreen has to be founded to ensure that the windscreen configuration does not have a large influence on the frequency measurement response of the speaker. To test these criteria, the frequency response of a speaker is measured in the anechoic chamber without any windscreen configuration and without the original windscreen. This measurement is compared with the frequency response of the speaker with the windscreen configuration. Finally, the wind noise is measured. To measure the wind noise two low speed and low noise fan is generating \SI{2.5}{\meter\per\second} at the microphone position. The wind noise is measured without any windscreen configuration and the original windscreen and compared with the wind noise in the microphone position in the windscreen configuration. To ensure that the background noise is identically on the wind noise measurement with and without the windscreen configuration two microphones are used and recorded simultaneously. Both the time signal and the frequency content is analysed. The wind speed attenuation is founded in \autoref{ap:wind_speed_att_first}. The wind noise attenuation is founded in \autoref{ap:wind_noise_att}. The frequency response is founded in \autoref{ap:wind_screen_freq_res}


The result for all configuration is as follows.

\paragraph{Configuration one} is the one with the smallest foam wedge and size of the wedge is measured to have the worst wind attenuation. The wind attenuation shows that the wind speed is lowered from \SI{8}{\meter\per\second} to \SI{2}{\meter\per\second}. However, the directional turbulence in the wind is more stable in this configuration compare the configuration three and above. The frequency response of the windscreen configuration is the one that has the lowest effect. At low frequency, up to \Hz{100}, the windscreen does not affect the measurement. The frequency above \Hz{100} gets off with about \dBr{2} compare with only the original windscreen. The measured wind noise attenuation is equal to zero. In the measurement, the wind noise is a bit worse compared to only the original windscreen. The attenuation is both approximately \dBr{10} for both with only the original windscreen and the windscreen configuration in the low frequency below \Hz{10}, but at some frequency, the attenuation is lower that \dBr{5} for the windscreen configuration. For frequency above \Hz{10}, the windscreen configuration does not affect.

\paragraph{Configuration two} is the one with the largest foam wedge and is measured to have one of the best wind speed and noise attenuations. The wind attenuation shows that the wind speed is lowered from \SI{8}{\meter\per\second} to \SI{1}{\meter\per\second} and have less peek in the wind speed compare to the windscreen with Rockwool. The directional turbulence in the wind is more stable in this configuration compare the configuration three and above, but little less stable compared to configuration one. The frequency response of the windscreen configuration has an amplification in the low frequency range from \Hz{80} to \Hz{600} of  \dBr{2}. From \Hz{1000} and above the frequency response is very similar compared to only the original windscreen. At low frequency up to \Hz{80}, the windscreen does not affect the measurement much. The windscreen attenuates the wind noise \dBr{10} more than only the original windscreen from \Hz{30} and downwards to the measured limit at \Hz{2}. The frequency range between \Hz{30} and \Hz{600} have the same attenuation as the original windscreen, and the frequency above have further \dBr{10} more attenuation than the original windscreen.

\paragraph{Configuration three} is the one with two Rockwool bat formed as an arrow and is measured to have wind attenuation between the small wedge and large wedge. The frequency response of the windscreen configuration is the worst. It alternates between $\pm$\dB{6}. At the low frequency range from \Hz{80} to \Hz{600} the amplification goes from \dBr{2} at \Hz{80} to \dBr{6.2} at \Hz{250} and then back to  \dBr{0} at \Hz{700}. At  \Hz{1000} the attenuation is at \dB{6} and above the frequency response alternate around the frequency response of the original windscreen. The windscreen attenuates the wind noise \dBr{10} more than only the original windscreen from \Hz{30} and downwards to the measured limit at \Hz{2}. The frequency range between \Hz{30} and \Hz{600} have the same attenuation as the original windscreen, and the frequency above have further \dBr{5} to \dBr{10} more attenuation than the original windscreen. Based on that the frequency response and the wind noise attenuation is worse than configuration two, the configuration is excluded.

\paragraph{Configuration four} is the one with only one Rockwool bat where the microphone is situated close to the side of the windscreen and is measured to have one of the best wind speed attenuations. The wind attenuation shows that the average wind speed is lowered from \SI{8}{\meter\per\second} to \SI{1}{\meter\per\second}, but the directional and wind speed turbulence is less stable compared to the configuration the windscreen with a foam wedge. The wind speed turbulence circulate from \SI{0}{\meter\per\second} to \SI{2}{\meter\per\second}. The frequency response of the windscreen configuration does not change more than approximately $\pm$\dBr{2} in the low and high frequency range. The noise attenuation is not measured in this configuration since the mechanical stability is founded to be poor in wind speed above \SI{5}{\meter\per\second}.




Based on the finding above, the final windscreen concept is designed. The design of the final windscreen concept combines configuration two to and configuration four, where the stability problem is solved. 

As the first test of the final windscreen solution, a preliminary setup is done with the available material in the acoustics lab. The preliminary setup is defined as windscreen configuration five. The following \autoref{fig:td:mes_roc_foa_con} illustrate the concept.

\xfig{design/measure_rock_foam_concept.pdf_t}{The figure shows the final windscreen concept. This concept is defined as windscreen configuration five.}{fig:td:mes_roc_foa_con}{1}  

This configuration is measured to have more wind speed attenuation that this combination apart. The wind speed attenuation shows that the mean wind speed is lowered from \SI{8}{\meter\per\second} to \SI{0.8}{\meter\per\second}, but the directional turbulence is less stable compared to the configuration the windscreen with only foam wedge. The frequency response of the windscreen configuration is as configuration two but with a little closer fit to without windscreen in the high frequency.