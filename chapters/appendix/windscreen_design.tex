\chapter{Design of windscreen}


The first two windscreen concept is very identical but just with difference size of material. The idea for the first windscreen is to seal the microphone with foam all around except at the frontal direction. The frontal direction include both \SI{180}{\degree} angle in the vertical direction and \SI{90}{\degree} in the horizontal direction. The resan to chose that high degree vertical opening is that no sound is from ground reflection is effected and no sound from upwards is stopped by the foam. The reson to have a narrow horizontal  opening is to be able to get sound inside the opening but still have a wind stopping effect. The following \autoref{fig:td:mes_foa_con} illustrate both windscreen configuration one and windscreen configuration two, just with one size foam wedge.
    
\xfig{design/measure_foam_concept.pdf_t}{The figure shows the foam wedge concept. The concept is is covering over to differend foam wedge, ether two small or two large. The small concept is defined as windscreen configuration one, where the large concept is defined as windscreen configuration two.}{fig:td:mes_foa_con}{1} 

The next concept build on the concept in \autoref{fig:td:mes_foa_con} just with plan surfaces rockwool plates. The opening is also \SI{180}{\degree} angle in the vertical direction and \SI{90}{\degree} in the horizontal direction. The concept is defined as windscreen configuration three. The following \autoref{fig:td:mes_roc_con} illustrate the concept.

The next concept build on minimizing the reflection from the additional windscreen by only placing the microphone close agenst one surface, which cover for the wind noise. The concept is definde as windscreen configuration four. The following \autoref{fig:td:mes_roc_sin_con} illustrate the concept.



\begin{figure}[H]
    \centering
     \captionsetup{width=1\linewidth}
    \begin{minipage}{0.46\textwidth}
        %\centering
         \captionsetup{width=0.90\linewidth}
       \input{figures/design/measure_rock_consept.pdf_t}
        \caption{The figure shows the rockwool concept. This concept is defined as windscreen configuration three.}
        \label{fig:td:mes_roc_con}
    \end{minipage}%
    \begin{minipage}{0.46\textwidth}
        \centering
         \captionsetup{width=0.90\linewidth}
        \input{figures/design/measure_rock_single_consept.pdf_t}
        \caption{The figure shows the single rockwool concept. This concept is defined as windscreen configuration four.}
        \label{fig:td:mes_roc_sin_con}
    \end{minipage}
\end{figure}


 

Before the optimal windscreen configuration is founded, an optimality creteria is defined and a test is designed. The optimal creteria for the windscreen is as low wind noise as possible at the microphone position and low sound reflection. To find the windscreen configuration which meets the creteria best, three test is made on the windscreen configuration. First the wind speed attenuation of the windscreen configuration is measured to ensure that the windscreen configuration concept does have an effect on the wind speed. The measurement of the windspeed attenuation can be founded in \autoref{}. Secondly the frequency response of the windscreen have to be founded to ensure that the windscreen configuration does not have a large influence on the frequency measurement response of the speaker. To test this criteria, the frequency response of a speaker is measured in the anechoic chamber without any windscreen configuration and without the original windscreen. This measurement is compared with the frequency response of the speaker with the windscreen configuration. The measurement is founded in \autoref{}. Finally the wind noise is measured. To measure the wind noise two low speed and low noise fan is generating \SI{2.5}{\meter\per\second} at the microphone position. The wind noise is measured without any windscreen configuration and the original windscreen and compared with the wind noise in the microphone position in the windscreen configuration. To ensure that the background noise is identically on the wind noise measurement with and without the windscreen configuration two microphone are used and recorded simultanius. Both the time signal and the frequency content is analysed. The measurement is founded in \autoref{}. The result for all configuration is as following.

\paragraph{Configuration one} is the one with the smallest foam wedge and size of the wedge is measured to have the worst wind attenuation. The wind attenuation shows that the wind speed is lowered from \SI{8}{\meter\per\second} to \SI{2}{\meter\per\second}. But the directional turbulence in the wind is more stabile in this configuration compare the configuration three and above. The frequency response of the windscreen configuration is the one that have the lowest effect. At low frequency upto \Hz{100} the windscreen does not effect the measurement. Frequency above the frequency response gets off with about \dB{2} compare with only the original windscreen. The measured wind noise attenuation is equal zero. In the measurement the wind noise is actualy a bit worse compare to only the original windscreen. The attenuation is both approximatly \dB{10} for both with only the original windscreen and the windscreen configuration in the low frequency below \Hz{10}, but at some frequency the attenuation is lower that \dB{5} for the windscreen configuration. for frequency above \Hz{10} the windscreen configuration have no effect.

\paragraph{Configuration two} is the one with the largest foam wedge and size of the wedge is measured to have one of the best wind attenuation. The wind attenuation shows that the wind speed is lowered from \SI{8}{\meter\per\second} to \SI{1}{\meter\per\second} and have less peek in the wind speed compare to the windscreen with rockwool. The directional turbulence in the wind is more stabile in this configuration compare the configuration three and above but little less stabil compare to configuration one. The frequency response of the windscreen configuration have an amplification in the low frequency range from \Hz{80} to \Hz{600} of  \dB{2}. From \Hz{1000} and above the frequency response is very similar compare to only the original windscreen. At low frequency upto \Hz{80} the windscreen does not effect the measurement much. The windscreen attenuate the wind noise \dB{10} more than only the original windscreen from \Hz{30} and downwards to the measured limit at \Hz{2}. The frequency range between \Hz{30} and \Hz{600} have the same attenuation as the original windscreen and the frequency above have further \dB{10} more attenuation than the original windscreen.

\paragraph{Configuration three} is the one with two rockwool bat formed as an arrow and is measured to have wind attenuation between the small wedge and large wedge. The frequency response of the windscreen configuration is the worst. It alternate between $\pm$\dB{6}. At the low frequency range from \Hz{80} to \Hz{600} the amplification goes from \dB{2} at \Hz{80} to \dB{6.2} at \Hz{250} and then back to  \dB{0} at \Hz{700}. At  \Hz{1000} the attunuation is at \dB{6} and above the frequency response alternate around the frequency response of the original windscreen. The windscreen attenuate the wind noise \dB{10} more than only the original windscreen from \Hz{30} and downwards to the measured limit at \Hz{2}. The frequency range between \Hz{30} and \Hz{600} have the same attenuation as the original windscreen and the frequency above have further \dB{5} to \dB{10} more attenuation than the original windscreen. Based on that the frequency response and the wind wind noise attenuation is worse than configuration two, the configuration is excluded from the rest of the test and is not used.

\paragraph{Configuration four} is the one with only one rockwool bat where the microphone is siturated close to the side of the windscreen and is measured to have one of the best wind attenuation. The wind attenuation shows that the mean wind speed is lowered from \SI{8}{\meter\per\second} to \SI{1}{\meter\per\second}, but the directional and wind speed turbulence is less stabile compare to the configuration the windscreen with foam wedge. The wind speed turbulence circulate from \SI{0}{\meter\per\second} to \SI{2}{\meter\per\second}. The frequency response of the windscreen configuration is does not change more than $\pm$\dB{2} in the low and high frequency range. At frequency from \Hz{600} to \Hz{300} the windscreen have an attenuation of \dB{4}. The noise attenuation is not measured in this configuration, since the mechanical stability is founded to be poor in wind speed above \SI{5}{\meter\per\second}.
