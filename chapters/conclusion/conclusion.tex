


In this chapter, a recapitulation of all results from the master theses is given. This will especially take upon the questions that were stated in \autoref{ch:statement}.

\paragraph{Measurement Design}
A measurement routine, which allows determining syncronised impulse response and weather information in moving inhomogeneous atmospheric condition of a sound source such as a line source array in different angle has been designed and implemented in \matlab \autoref{sec:des:des_mes}. The measurement design was tested before the final measurement is performed. The test outsourced a few difficulties, and the final measurement result is analysed.



\paragraph{Proposal solution}
It has been proposed to rotate the line source array up agents the wind to compensate for the upwards and downwards refraction while the wind is crosswind to the speaker. It has further been proposed to tilt the line source array more downwards to move the shadow zone as far back as possible while parallel wind and upwards refraction is present. It has been founded that the proposal idea is shown to obtain more homogeneous \gls{spl} in the crosswind and downwards tilting propagate more sound energy into the shadow zone. The following to paragraph concludes on the crosswind solution and parallel solution, respectively.   




\paragraph{Crosswind}
It can be concluded that the line source array \gls{spl} coverage can be optimised by rotating the line source array while upwards and downwards refraction is present. The meaning of optimisation is less differences between the measured \gls{spl} between the microphone positions. The optimisation is researched in the octave band from \Hz{1000} to \SI{16}{\kilo\hertz} octave. This frequency range is shown to be the refracting part of the frequency range from \Hz{20} and up to \SI{20}{\kilo\hertz} in the distance of \SI{50}{\meter}. It is observed that the \gls{spl} at the downwards microphone position is up to \dB{17.67} more than at the upwards microphone position in the \Hz{2000} octave band while the line source array is not rotated. While rotating the line source array up against the wind, which means in the upwards direction, the difference is lowered from \SI{17.67}{\decibel}  to \SI{-2.32}{\decibel} while the line source array is rotated \SI{20}{\degree}. This optimisation of the \gls{spl} differences by rotating the line source array is measured in the wind speed interval of $[\SI{5}{\meter\per\second},\, \SI{10}{\meter\per\second}[ $. wind speed above and beneath is not measured or analysed. A linear least square fit is performed on all data in the wind speed interval of $[\SI{5}{\meter\per\second},\, \SI{8}{\meter\per\second}[ $ with wind speed step of $[\SI{1}{\meter\per\second}$ and one linear least square fit on the data in the wind speed interval of $[\SI{8}{\meter\per\second},\, \SI{10}{\meter\per\second}[ $ is performed to predict the optimal angle. The least square fit is performed both as single octave band fit and with all octave band for every wind speed interval. No refraction versus wind speed in the measured wind speed area is observed. The optimal angle in the average single octave band spends from \SI{15.0}{\degree} to \SI{17.3}{\degree} while \dB{1000} octave band linear least square fit is excluded. The execution is based on that the angle is calculated to be negative, which is highly non-realistic based on the refraction theory. The optimal static angle in this interval, while the angle is calculated as an average angle of every linear least square fit, is \SI{16.15}{\degree}. The optimal angle is also calculated based on a linear least square fit on all data for every wind speed interval. This shows a optimal angle from \SI{14.8}{\degree} to \SI{16.2}{\degree} with a average angle of \SI{15.6}{\degree}. In the end, the optimal angle is also calculated from the absolute difference from the centre microphone to the upwards microphone and the downwards microphone as a second order least square fit. This shows an optimal angle from \SI{14.4}{\degree} to \SI{16.3}{\degree} with an average angle of \SI{15.6}{\degree}. 

Three methods of calculating the optimal angle are performed, \SI{15.6}{\degree} angle is predicted to give the lowest absolute difference between the centre microphone to the upwards and downwards microphone and the lowest difference between the upwards and downwards microphone while all octave band data is used to calculate the least square fit.

It is concluded that the optimal angle for the L-Acoustics KODO line source array is \SI{15.6}{\degree} up against the wind, while the wind speed is between $[\SI{5}{\meter\per\second},\, \SI{10}{\meter\per\second}[ $ and the wind direction is \SI{90}{\degree}.


It is acked in the proposal solution is stering more power in the upwards direction also higher the \gls{spl} in the shadow zone. It is founded in this research that the \gls{spl} is raised in the shadow zone by stered the sound energy intro the upwards direction 


\paragraph{Parallel wind}
It can be concluded that the shadow zone distance from the line source array can be optimised by tilting the line source array more downwards. It is shown that the \gls{spl} in octave band at a distances of both \SI{40}{\meter}, \SI{50}{\meter} and \SI{60}{\meter} is raised while the line source array is tilted from \SI{3}{\degree} to \SI{7}{\degree}. While the array is tilted \SI{3}{\degree} the line source array near-field is pointing directly to the microphone where while the line source array is tilted \SI{7}{\degree} the near-field is in front of the microphone. Therefore, if no wind were present in the tilt angle of \SI{3}{\degree}, highest \gls{spl} is predicted. Since the highest \gls{spl} is measured in the tilt angle of \SI{7}{\degree} it can be concluded that tilting the line source array produce higher \gls{spl} in the shadow zone. The decay in \gls{spl} is equally with the tilt angle, and therefore the shadow sone might not be moved back, only the power that enters is higher.  





