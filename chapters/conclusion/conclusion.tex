

\chapter{Conclusions}\label{sec:conclusion}
In this chapter, a recapitulation of all results from the master theses is given. This will especially take up on the questions that were stated in \autoref{ch:statement}.

\paragraph{Measurement Design}
A measurement routine, which allows determining the impulse response and weather information in moving inhomogenius athmosphere condition of a sound sources such as a line source source array in different angle has been designed and implemented in \autoref{sec:des:des_mes}. The measurement design was tested before the final measurement was preformed. The test outsoruced few difficulties, and the final test result was very use full.  


%\paragraph{Ground reflection and wind noise}


\paragraph{Proposal solution}
It has ben proposed to rotate the line source array up agents the wind to compensate for the upwards refraction while the wind is crosswinf to the speaker. It have futher been proposed to tilt the line source array more downwards to move the shadow zone as far back as posible while parallel wind and upwards refraction is present. It has been founded that proposal idea is shown to optain more homogenius \gls{spl} in the crosswind and downwards tilting move the shadow zone or produce more sound energy intro the shadow zone. The following to paragraph conclude on the crosswind soulution and parallel soultion respectavly.   




\paragraph{Crosswind Speaker Angle}
It can be concluded that the line source array \gls{spl} coverage can be optimised by rotating the line source array. The optimization is shown to be possible in octave band from \Hz{1000} to \SI{16}{\kilo\hertz} octave band. This frequency range is shown to be the refracting part of the frequency range from \Hz{20} and above in the distance of \SI{50}{\meter}. It is observed that the \gls{spl} at the upwards microphone position is much less that the \gls{spl} of the downwards microphone position. 



\paragraph{Parallel Speaker Angle}






