In this chapter, a recapitulation of all work from the master theses is given. This will especially take upon the question that is stated in \autoref{ch:statement}.

\paragraph{Observed problem}
It is founded in the problem analysis that sound wave refraction in crosswind condition makes the \gls{spl} coverage over the audience area uneven. In the direction of headwind, the \gls{spl} is lower because of upwards refraction. In the direction of tailwind the \gls{spl} is higher because of downwards refraction.  This problem is researched, and a solution is proposed and tested.


\paragraph{Proposal solution}
It is proposed to rotate the line source array up against the wind to compensate for the upwards and downwards refraction while the wind is crosswind to the line source array. It is further proposed to tilt the line source array more forwards to move the shadow zone as far back as possible while parallel wind and upwards refraction is present. 

\paragraph{Measurement setup}
To be able to test the proposed solution, a measuring setup is designed. The measuring system is a full-scale line source array system from L-Acoustics. The first part analyses the used line source array. The frequency characteristic and frequency directionality of the line source element is measured to be able to decide on the rotation of the line source array versus the sound coverage area. The sound coverage area is researched by a questioner and the typical maximum line source array coverage distance is \SI{50}{\meter}. Based on the coverage area distance of \SI{50}{\meter}, the reference forward tilt angle is \SI{3}{\degree}. The analysis analyses both crosswind and parallel wind effect on sound wave propagation in a refraction atmosphere. To be able to measure the frequency response of the line source array in inhomogeneous weather condition, an area with mown grass and few buildings are chosen. The area is without the audience doing the measurement, and therefore, the condition in the measurement is different concerning the condition while the concert is playing. To be able to reproduce the concert condition doing the measurement, the concert condition is analysed concerning the ground reflection. It is founded that the audience is an absorber in the frequency spectrum above \Hz{250}, and therefore, the head of the audience is the new ground plane for frequency above \Hz{250}. At frequency below the audience absorption decay and therefore, ground reflection might occur. A windscreen is designed to lower the wind noise and ground reflection above \Hz{250} such that the microphone is able to be in the hight of the head. 

To be able to measure the upwards refraction and downwards refraction, the following list explain the equipment.

\begin{itemize}
\item Six L-Acoustics KUDO line source array element with two LAB Gruppen PLM 10000Q amplifier. The line source array is flown in the hight of \SI{5}{\meter} in a ground support truss setup.
\item Three microphone windscreen is designed and build, where all windscreen is positioned in a bow with \SI{50}{\meter} radius from the line source array while crosswind condition. The centre microphone is placed directly in the frontal angle of the line source array, where the two other microphones are placed \SI{25}{\degree} to both sides of the line source array. The centre microphone is in directly crosswind with respect to the wave propagation, where the other is either in upwards and downwards direction.
\item For parallel wind the microphone is placed on a row in front of the line source array with a distance of \SI{40}{\meter}, \SI{50}{\meter} and \SI{60}{\meter}.
\item The rotation is measured with laser attached on the line source array and a circular angle plate, where the rotational angle is given. The laser then lights on the angle plate and show the rotation of the line source array. The line source array is rotated by a long truss piece connected to the top of the line source array. The forward tilting is obtained by a rope connected on the bottom of the line source array.
\item The weather information is measured with two anemometers and one temperature and humidity sensor. One anemometer, the temperature and humidity sensor is positioned close to the line source array, and one anemometer is positioned at the centre microphone. 
\item The sensor is connected to an Arduino UNO and firmware is designed for data transfer between the Arduino and \matlab .
\end{itemize}


\paragraph{Measurement software and hardware}
A measurement routine, which allows determining syncronised impulse response and weather information in moving inhomogeneous atmospheric condition of a sound source such as a line source array in different rotation has been designed and implemented in \matlab. The impulse response measurement is based on deconvolution of measured sine sweep and the played sine sweep in the frequency domain. The weather data is transferred from the Arduino into the serial bus, where \matlab receive weather data in between every sound card buffer transfer. The buffer size which is transferred from \matlab to the sound card is 4096 sample, and the sine sweep is \SI{5}{\second} long with a sampling frequency of 44100 samples per second. With this buffer size, 55 weather sample is measured for every sine sweep measurement. 



\paragraph{Measurement}
The measurement design is tested before the final measurement is performed to outsourced difficulties in the designed measurement. The test showed that the windscreen is succeeding in lowering the wind noise and some ground reflection, but the ground reflection is too high in the frequency above \Hz{250} and therefore the windscreen is decided to be placed on the ground doing the final measurement. Furthermore, a frequency difference between a few measurements is observed while the windscreen is covering the microphone and is not covering the microphone. The frequency response differences are researched, and only \dBr{2} deviation in the frequency response is measured while the windscreen is not tilted and rotated.  The final measurement is performed on a windy day with no rain. The wind is measured to have a wind speed interval between \SI{5}{\meter\per\second} to \SI{10}{\meter\per\second}. The measurement result while crosswind condition and parallel wind condition is analysed and the following two paragraphs explain the result, respectively.



\paragraph{Crosswind}
The line source array \gls{spl} coverage is optimised by rotating the line source array while upwards and downwards refraction is present. The meaning of optimisation is minimising the differences between the measured \gls{spl} between the microphone positions. The optimisation is researched in octave band from \Hz{1000} to \SI{16}{\kilo\hertz} octave band. This frequency range is shown to be the refracting part of the frequency range from \Hz{20} and up to \SI{20}{\kilo\hertz} in the distance of \SI{50}{\meter}. It is observed that the \gls{spl} at the downwards microphone position is up to\textbf{ \dB{17.67} higher} than at the upwards microphone position in the \Hz{2000} octave band while the line source array is not rotated. 

While rotating the line source array \SI{20}{\degree} up against the wind, which means in the upwards direction, the difference is lowered from \SI{17.67}{\decibel}  to \SI{-2.32}{\decibel}. This optimisation of the \gls{spl} differences by rotating the line source array is measured in the wind speed interval of $[\SI{5}{\meter\per\second},\, \SI{10}{\meter\per\second}[ $. Wind speed above and beneath is not measured or analysed. A linear least square fit is performed on all data in the wind speed interval of $[\SI{5}{\meter\per\second},\, \SI{8}{\meter\per\second}[ $ with wind speed step of \SI{1}{\meter\per\second} and one linear least square fit on the data in the wind speed interval of $[\SI{8}{\meter\per\second},\, \SI{10}{\meter\per\second}[ $ is performed to predict the optimal angle. The linear least square fit is performed both as single octave band fit and with all octave band for every wind speed interval. \textbf{No correlation between the optimal line source rotation versus wind speed in the measured wind speed interval is observed} doing the data analysis. Since the wind speed and optimal line source array rotation is uncorrelated, the optimal static rotation in wind speed interval of $[\SI{5}{\meter\per\second},\, \SI{10}{\meter\per\second}[ $ is calculated.

The optimal line source array rotation in the average single octave band spends from \SI{15.0}{\degree} to \SI{17.3}{\degree} while the \Hz{1000} octave band linear least square fit in the wind speed interval  $[\SI{5}{\meter\per\second},\, \SI{6}{\meter\per\second}[ $  is excluded. The execution is based on that the rotation is calculated to be negative, which is against the refraction theory. The average optimal line source rotation for all octave band is \SI{16.15}{\degree}. By the analyse of the single octave band, it is founded that \textbf{the average \gls{spl} differences between the microphone position is correlated with the frequency}. Therefore, as higher frequency, as higher refraction.


The optimal rotation is also calculated based on a linear least square fit on all data for every wind speed interval. This shows a optimal rotation from \SI{14.8}{\degree} to \SI{16.2}{\degree} with a average rotation of \SI{15.6}{\degree}. In the end, the optimal rotation is calculated from the absolute difference between the centre microphone to the upwards microphone and the downwards microphone as a second order least square fit. This shows an optimal rotation from \SI{14.4}{\degree} to \SI{16.3}{\degree} with an average rotation of \SI{15.6}{\degree}. 

Three methods of calculating the optimal angle are performed, \textbf{\SI{15.6}{\degree} rotation is predicted to give the lowest} absolute difference between the centre microphone to the upwards and downwards microphone and the lowest difference between the upwards and downwards microphone while all octave band data is used to calculate the least square fit.

It is further asked in the proposed solution if steering more power in the upwards direction also raises, the \gls{spl} in the shadow zone. It is founded in this research that  \textbf{the \gls{spl} is raised in the shadow zone by steering more sound energy into the upwards direction}.

\textbf{ The optimal rotation for the L-Acoustics KODO line source array is \SI{15.6}{\degree} up against the wind, while the wind speed is between $[\SI{5}{\meter\per\second},\, \SI{10}{\meter\per\second}[ $ and the wind direction is \SI{90}{\degree} with an average directional deviation of $\pm$\SI{20}{\degree}. Furthermore, the \gls{spl} in the shadow zone is raised by pointing more sound energy into the upwards direction. } 





\paragraph{Parallel wind}
The shadow zone attenuation from the line source array is optimised by tilting the line source array more forwards. The microphone is compared by excluding the distance loss and the viscosity influence founded in the analysis. By removing the viscosity and distance loss differences from the microphone, all three microphone position can be compared. If the \gls{spl} is lower at one microphone position than the other, then the microphone is within the shadow zone. It is shown that the \gls{spl} in octave band at a distances of both \SI{40}{\meter}, \SI{50}{\meter} and \SI{60}{\meter} is raised while the line source array is forward tilted from \SI{3}{\degree} to \SI{7}{\degree}. \textbf{At a distance of \SI{50}{\meter} the \gls{spl} is raised with \dB{5.56} in the \Hz{8000} octave band.} In this octave band, the sound wave at a distance of \SI{50}{\meter} is at the border between near-field and far-field. While the array is tilted \SI{3}{\degree} the line source array near-field is pointing directly to the microphone where while the line source array is tilted \SI{7}{\degree} the near-field is in front of the microphone. Therefore, if no wind were present in the forward tilt angle of \SI{3}{\degree}, highest \gls{spl} is predicted. Since the highest \gls{spl} is measured in the tilt angle of \SI{7}{\degree} \textbf{ it is observed that forward tilting of the line source array produce higher \gls{spl} in the shadow zone.} 

The decay in \gls{spl} is equally with the measured forward tilt angle between the front microphone and the centre microphone, and therefore, all microphone is within the shadow zone. The shadow zone might have been moved back, but cannot be supported by this measurement. More microphone closer to the line source array is needed to be able to measure the start position of the shadow zone. The start position of the shadow zone might be seen where one forward tilt angle does not decay the \gls{spl} while the other forward tilt angle does decay the \gls{spl}, where the measuring position is with the same distance to the line source array.

\textbf{The \gls{spl} is raised in the shadow zone by forward tilting the line source array from \SI{3}{\degree} to \SI{7}{\degree} in the octave band from \Hz{1000} to \SI{16}{\kilo\hertz} }