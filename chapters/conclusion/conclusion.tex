


In this chapter, a recapitulation of all results from the master theses is given. This will especially take up on the questions that were stated in \autoref{ch:statement}.

\paragraph{Measurement Design}
A measurement routine, which allows determining the impulse response and weather information in moving inhomogenius athmosphere condition of a sound sources such as a line source source array in different angle has been designed and implemented in \autoref{sec:des:des_mes}. The measurement design was tested before the final measurement was preformed. The test outsoruced few difficulties, and the final test result was very use full.  



\paragraph{Proposal solution}
It has ben proposed to rotate the line source array up agents the wind to compensate for the upwards refraction while the wind is crosswinf to the speaker. It have futher been proposed to tilt the line source array more downwards to move the shadow zone as far back as posible while parallel wind and upwards refraction is present. It has been founded that proposal idea is shown to optain more homogenius \gls{spl} in the crosswind and downwards tilting move the shadow zone or produce more sound energy intro the shadow zone. The following to paragraph conclude on the crosswind soulution and parallel soultion respectavly.   




\paragraph{Crosswind Speaker Angle}
It can be concluded that the line source array \gls{spl} coverage can be optimised by rotating the line source array. The optimization is shown to be possible in octave band from \Hz{1000} to \SI{16}{\kilo\hertz} octave. This frequency range is shown to be the refracting part of the frequency range from \Hz{20} and above in the distance of \SI{50}{\meter}. It is observed that the \gls{spl} at the downwards microphone position is up to \dB{17.67} more than at the upwards microphone position in the \Hz{2000} octave band while the line source array is not notated. While rotating the the line source array up agenst the wind, in the upwards direction \SI{20}{\degree}, the difference is lowered from \SI{17.67}{\decibel} to \SI{-2.32}{\decibel}. This optimisation of the differences by rotating the line source array is measured in all wind speed interval from $[\SI{5}{\meter\per\second},\, \SI{10}{\meter\per\second}[ $. A linear least square fit is preformed on all data from $[\SI{5}{\meter\per\second},\, \SI{7}{\meter\per\second}[ $ with wind speed step of $[\SI{1}{\meter\per\second}$ and one linear least square fit on the data in the interval of $[\SI{8}{\meter\per\second},\, \SI{10}{\meter\per\second}[ $ to predict the optimal angle. The least square fit is performed both as single octave band and all octave band. No refraction versus wind speed in the measured wind speed area is observed. The optimal angle in the average single octave band spends from \SI{15.0}{\degree} to \SI{17.3}{\degree} while \dB{1000} octave band linear least square fit is excluded. The execution is based on that the angle is calculated to be negative which is highly non realistic based on the refraction theory. The optimal static angle in this interval while the angle is calcuælated as a average angle of every linear least square fit is \SI{16.15}{\degree}. The optimal angle is also calculated based on the a linear least square fit on all data for every wind speed interval. This shows a optimal angle from \SI{14.8}{\degree} to \SI{16.2}{\degree} with a average angle of \SI{15.6}{\degree}. In the end the optimal angle is also calculated from the absolute difference from the center microphone to the upwards microphone and the downwards microphone as a second order least square fit. This shows an optimal angle from \SI{14.4}{\degree} to \SI{16.3}{\degree} with an average angle of \SI{15.6}{\degree}. 



Three method of calculating the optimal angle is preformed, \SI{15.6}{\degree} angle is predicted to give the lowest absolute difference between the center microphone to the upwards and downwards microphone and the lowest difference between the upwards and downwards microphone while all octave band data is used to calculate the least square fit.

It is therefore concluded that the optimal angle for the L-Acoustics KODO line source array is \SI{15.6}{\degree} up against the wind while the wind speed is between $[\SI{5}{\meter\per\second},\, \SI{10}{\meter\per\second}[ $.





\paragraph{Parallel Speaker Angle}
It can be concluded that the shadow zone distance form the line source array can be optimised by tilting the line source array more downwards. It is shown that the \gls{spl} in octave band at a distances of both \SI{40}{\meter}, \SI{50}{\meter} and \SI{60}{\meter} is raised while the line source array is tilted from \SI{3}{\degree} to \SI{7}{\degree}. While the array is tilted \SI{3}{\degree} the line source array near-field is pointing directly to the microphone where while the line source array is tilted \SI{7}{\degree} the near-field is in front of the microphone. Therefore is no wind were present the tilt angle of \SI{3}{\degree} should show the highest \gls{spl}. Since the highest \gls{spl} is measured in the tilt angle of \SI{7}{\degree} it can be concluded that the shadow zone can ether be moved back or more energy can be played intro the shadow zone by ttilting the line source array.





