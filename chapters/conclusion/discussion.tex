This chapter discusses the observation and further research on the topic. The first observation, which is discussed is the mechanical line source array tilt angle stability of the line source array doing the measurements. The second observation, which is discussed is the wind interval expansion. The third part gives the authors design suggestion for rotating the line source array in windy weather. The last part gives some idea to research the measurement outliers in new measurements.


\paragraph{Line source array tilt angle position stability in windy weather} 
It is observed doing the measurement that the tilt angle of the line source array is very dynamic along with the wind speed and wind direction. While the wind speed changes the line source array swings along with the wind in the wind direction and oscillates in shallow frequency. It is observed that the tilt angle can change more than $\pm$ \SI{1}{\degree} doing a measurement.  Doing the measurement in this theses, the line source array is stabilised by three rope attached underneath of the line source array and guided down to the ground in three directions. This solution stabilised the line source array to oscillate beneath  $\pm$ \SI{1}{\degree}. $\pm$ \SI{1}{\degree} is an oscillation of a maximum radius of \SI{8}{\centi\meter} of the laser pointer down on the measuring plate from the centre position. The half circle drawn on the angle plate has a radius on \SI{5}{\centi\meter}. In the measurement, the oscillation is within this circle as the circle while imagine that the circle is drawn finished. In further research, a mechanical solution for stability is suggested. 



\paragraph{Measurements expansion in both lower and higher wind speed} 
It is observed in the crosswind measurement design data analysis result in \autoref{res:cross_data_ana} that the optimal line source array rotation is between \SI{14.8}{\degree} to \SI{17.3}{\degree} depending on the optimality criteria. Moreover, it is founded that the highest line source array rotation is not at the highest wind speed. The optimal rotation seems to be low correlated with the wind speed change in the speed interval between $[\SI{5}{\meter\per\second},\, \SI{10}{\meter\per\second}[ $. Therefore it is interesting to analyse the behaviour of the optimal line source rotation in the wind speed interval of $[\SI{0}{\meter\per\second},\, \SI{5}{\meter\per\second}[ $. It is guessed in this wind speed interval that the line source array rotation function is logarithmic. Therefore, in the lower part of the interval, the refraction differences change is highly correlated with the wind speed change, whereas the wind speed raises the correlation decay. The upwards and downwards refraction is due to wind speed differences concerning the hight above ground. This measurement indicates that the speed differences might be stable in the measured wind speed interval. It could furthermore be interesting to measure in higher wind speed interval to research if the measured tendency follows or change. 



\paragraph{Research the outliers}
The measured weather data cannot explain the outliers observed in the measurements in \autoref{res:cross_data_ana}. It might be due to the oscillation of the line source array or that the two wind measuring point is non-representable of those measurements or high wind turbulence. To research outliers in further research, the oscillation of the line source is suggested to be measured. Furthermore, it is suggested to use ultrasonic anemometer for the measurement, to measure more instantaneous weather condition and maybe a grid of anemometer between the line source array and the microphone. With this information, the turbulence in the wind can be measured, the wind direction is more precise in the data, and the line source array forward tilt angle does not change. 




\paragraph{Design suggestion} 
The author suggests that it is the frequency range from \Hz{700} to \SI{20}{\kilo\hertz}  that shall be controlled by the line source array \textbf{rotation} and \textbf{forward tilting} of the line source array for crosswind refraction or parallel wind refraction respectively. This frequency range covers the middle frequency driver and the high frequency driver. The most important frequency range for refraction control might be weighted according to the long term average frequency spectrum in music nowadays. Music today often has a pink spectrum while analysing the long term average frequency spectrum \citep{elowsson2017long}. Moreover, is is founded that the intelligibility frequency range is between \Hz{500} octave band to \Hz{8000} octave band. By this knowledge, the most important frequency range for line source array rotation and forward tilting to compensate for refraction is within \textbf{\Hz{700} to \SI{12}{\kilo\hertz}}. At least two possibilities of controlling this frequency range of rotation the line source array is possible. One solution which can be applied to all existing line source array and one solution for a full redesign of the line source element. The solution to the existing line source array is using an adaptive control of the motorised lifting chain for the line source array. By lifting the line source array in two points where the front lifting point is lifted by one chain, and the back lifting point is lifted by two chains fixed on the truss to the side of the line source array. By this technique, the front chain is the central lifting point, where the two back chain can move the back point from side to side as a power steering solution. The following \autoref{fig:result:dis_design_prop} illustrate the rotational idea.

\xfig{result/design_prop.pdf_t}{The figure shows the measuring setup for parallel wind measuring as a top view.}{fig:result:dis_design_prop}{1}


 For a total redesign of the line source element, the middle frequency driver and high frequency driver shall be build intro one packed as CODA audio does in their line array \citep{coda_ddp}. Furthermore, the drivers shall be fixed such that it can be rotated inside, and the rotation fixpoint is at the mouth of the horn. The following \autoref{fig:result:dis_design_redes} illustrate the design idea.

\xfig{result/total_redesign.pdf_t}{The figure shows the measuring setup for parallel wind measuring as a top view.}{fig:result:dis_design_redes}{1}
