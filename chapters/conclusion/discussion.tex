
The chapter discus the observation and the futher research in the topic. The first observation which is discused is the line source array angle stability doing the measurements. The second observation which is discussed is the wind interval expansion. The third part gives the authors design suggestion for rotating the speaker in windy weather. The last part gives some idea to research the measurement outliers in new measurements.


\paragraph{Line source array position stability in windy weather} 
It is observed doing the measurement that the tilt angle of the speaker is very dynamic along the wind speed and wind direction. While the wind speed change the line source array swings along the wind in the wind direction and oscillate in very low frequency. It is observed that the tilt angle can change more than $\pm$ \SI{1}{\degree} doing a measurement. This oscillation have to be controlled in a futher research of line source array \gls{spl} controle in windy weather. Doing the measurement in this theses, the line source array was stabilised by three rope attached onder neath of the line source array and in down to the ground in three direction. This solution stabilised the line source array to oscillate beneath  $\pm$ \SI{1}{\degree}. $\pm$ \SI{1}{\degree} is an oscilation of a maximum radius of \SI{8}{\centi\meter} of the laser pointer from the center position. The half circle drewen on the angle plate has a radius on \SI{5}{\centi\meter} in. In the measurement the oscilation was within this circle. In further research a mechanical solution for the stability is suggested.


\paragraph{Measurements expansion in both lower and higher wind speed} 
It is observed in the measureing croswind data analysis result in \autoref{res:cross_data_ana} that the optimal angle is between \SI{14.8}{\degree} to \SI{17.3}{\degree} depending on the optimality chreteria. Moreover it is founded that the highest line source array angle is not at the highest wind speed. The optimal angle seems to be low correlated with the speed change in the speed interval between $[\SI{5}{\meter\per\second},\, \SI{10}{\meter\per\second}[ $. Therefore is is interesting to analyse the behaivure of the optimal line source angle in the wind speed interval of $[\SI{0}{\meter\per\second},\, \SI{5}{\meter\per\second}[ $. It is guessed in this wind speed interval that the angle function is logarithmic. So in the lower part of the interval, the refraction dirfferences change is highly correlated with the speed change where as the wind speed raises the correlation drops. The upwards and downwards refraction is due to wind speed differences with respect to the hight, this measurement indicate that the speed differences might be stabile in the measured wind speed interval. It could futher more be intersting to mersure in higher wind speed interval to research if the measured tendency followes or change. 



\paragraph{research the outlier}
The outliers observed in this measurements in \autoref{res:cross_data_ana} can not be explained by the measured weather data. It might be due to the oscillation of the line source array or that the tow wind measuring point was non representable of thoes measurement or some wind turbulence. To research outliers in futher research, the oscilation of the line source is suggested to be measured. Futhermore it is suggested to use ultra sonic anemometer for the measurement, to measure more instantinius weather condition and maybe a grid of anemometer between the line source array and the microphone. With this information the turbolense in the wind can be measured, the wind direction is more precise in the data naalysis and the speaker position does not change. 




\paragraph{Design suggestion} 
It is suggested by the author that it is the frequency range from \Hz{700} and upwards that can be controlled by rotation of crosswind refraction. This frequency range cover the middel frequency driver and the high frequency driver. At least two posibility of controling this frequency range be rotation is posible. One solution which can be applied to all excistion line source array and one solution for a fully redesign of the line source element. The solution to the excistion line source array is using an adaptive control of the motorazied lifting chain for the line source array. By lifting the line source array in two points where the front point is lifted by one chain and the back point is lifted by two change fixes on the truss in both side of the line source array. By this technique, the front chain is the central lifting point and where the two back chain can move the back point from side to side as a power steering solution. For a total redesign of the line source element, the middel frequency driver and high frequency driver shall be build intro one packed as CODA audio do in there line array \citep{coda_ddp}. Futhermore the drivers shall be fixed such that it can be rotated inside and the rotation fix point is at the mouth of the horn.   



