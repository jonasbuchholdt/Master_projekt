\section{Discussion}\label{sec:discussion}

In \autoref{pt:tests} all the requirements were tested to see if they can be approved. It is seen that out the 26 stated requirements, 16 are approved, six are partially approved and four are not approved.
\autoref{req:delay1} is approved, but some additions have to be made to make it function as a usable delay effect. The reason of this is that the time between the echoes is too short. With the implemented buffer size it is only possible to produce a delay time of \SI{41}{\micro\second}, which is too short. 
The six requirements that are partially approved are all related to the missing user interface. The user interface was not implemented due to the time limit of the project and the fact that it was chosen to make the implementation of the effects the main focus. The six partially approved requirements are categorized as such, because it is possible to change the required parameters, but only through changing them directly in the program. 
The first requirement that is not approved is \autoref{req:SNR}. It was measured that the \gls{dsp} has a \gls{snr} of \SI{39.91}{\decibel}, which is lower than the \gls{snr} in the guitar. The low \gls{snr} in the \gls{dsp} means that it will not even be enough to re-design the effects to make them as noiseless as possible.  
The second requirement that is not approved is \autoref{req:power2}, where the reason is that the \gls{dsp} is powered by a \SI{5}{\volt} USB supply. Since the supply voltage that is now used is smaller than the stated \SI{9}{\volt}, this requirement can be approved if a \SI{9}{\volt} supply is used to make a \SI{5}{\volt} supply for the \gls{dsp}.
The third and fourth requirement that are not approved are \autoref{req:flanger1} and \autoref{req:chorus1}. The reason of this is that the implemented \gls{cordic} algorithm is not able to produce sine waves with lower frequencies than \SI{0.4}{\hertz}, because with 32-bit precision sine and cosine values for frequencies less than \SI{0.4}{\hertz} can not be represented. A solution to this would therefore be to use a higher bit precision. \\
%The fifth requirement that is not approved is \autoref{req:preamp3}, because the measured output impedance of the preamp is measured to \SI{9.06}{\kilo\ohm}, which should be smaller than \SI{2}{\kilo\ohm}. The result of the measurement is questionable because the output impedance of an \gls{opamp} normally lies around \SI{50}{\ohm} to \SI{200}{\ohm}. \\

Some choices than were made in the development of the project can be discussed. One of them is the decision of implementing all the effects entirely in assembly. The reason behind the decision was to make the effects run as fast as possible, to get familiar with the assembly language, and to get a better understanding of the \gls{dsp}. It is seen in \autoref{app:effect_run_time}, that each effect uses approximately one tenth of the sampling period to complete its program. This indicates that it might have been possible to implement parts of effects in the slower, but faster implementable language C. This approach might have left time for implementing the user interface for instance. 
Another discussion is whether the resolution of the calculations in the effects should be increased from 16- to 32-bits. This will improve the sound quality of the effects. It was seen in the development of the \gls{cordic} calculations that increasing the precision, made it possible to to produce sine waves of lower frequencies. The resolution could of course always be a parameter that could be improved, but the noticeable difference from going from 16- to 32-bits, might be larger going from 32- to 64-bits. It has to be taken into account that higher resolution would require more memory.  
A subject for discussion is the approach for developing the effects, the \gls{reverb} for example. In this specific case, a Moorer \gls{reverb} unit was chosen. The reason of this decision was to have some guidelines for developing a \gls{reverb} effect. When looking back, it might have been beneficial to design the effect without relying as much on a reference. This might have improved the possibility  of making the effect sound as intended. \\

For further enhancements, several steps could be made towards improving the product. One of them is the user interface, which would not only make it possible to approve more of the requirements, but also make the final product more usable. One would for example not have stop and reprogram the \gls{dsp} to change effect, or even to just change one parameter in an effect. 
Additional developments can also be finishing the implementation of the designed effects but also the missing parts in the already implemented ones in assembly, such as implementing the attenuating part of the equalizer or making the \gls{cordic} algorithm able to produce different oscillations at the same time to finish the chorus effect. 
Moreover, a parameter that can be investigated more is the noise in the system. It could be investigated if it in some way is possible to improve the \gls{snr} in the \gls{dsp}.