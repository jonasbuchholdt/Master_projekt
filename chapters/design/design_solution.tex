\section{Test of measuring design}\label{sec:ds:test_of_mes_des}
This section aims to test the design measurement in a windy day, to outsource problem and error in the measuring design. The test is done in full scale with all six line source array element and in the designed hight. The test is intended to both test the crosswind measuring design and the parallel wind measuring design, but the wind condition only allowed for crosswind test. After the crosswind measuring design was tested the wind speed dropped to beneath \SI{1}{\meter\per\second}.
In the crosswind measuring design, three problems and one code error are observed. The code error is a data save bug, where only the direction of the wind at the line source array tower is saved. The wind direction data at the microphone position is overwritten by the wind direction data at the line source array. The code bug is fixed for the final measurement. The following three \autoref{sec:des:ground_reflection}, \autoref{sec:des:freq_boost} and \autoref{sec:des:measuring_angle} explain the observed difficulties or design failure and propose a solution to them. 
The first difficulty which is discovered is major ground reflections in the measurement. The second problem seems to be frequency response differences while using the windscreen, where the third error is the angle of the line source array while measuring in the hight of the ear.  
The explanations is based on the measurement as seen in \autoref{fig:td:mea_1_0_deg} which shows the frequency response on all three microphones at \SI{0}{\degree} rotation.

\plot{plot/measuring_1_0_deg}{The graph shows the mean frequency response of 10 measurement for all three microphone while the speaker is not rotated. The mean is calculated in the time domain by aligning the impulse response with the help of cross-correlation. The wind speed and wind direction is also the mean of the 10 measurements.}{fig:td:mea_1_0_deg}

To be able to differentiate between the microphone in the explanation, the microphone is named according to the position and the wind direction. In the upwards refraction direction, the microphone is named upwards microphone, in the downwards refraction direction the microphone is named downwards microphone, and the centre microphone is named centre microphone. The following \autoref{fig:ds:position_test} illustrate the microphone position versus the wind direction.

\xfig{design/measurement_one_KUDO.pdf_t}{The figure shows the microphone position versus the position of the line source array, while the line source array is \SI{0}{\degree} rotated}{fig:ds:position_test}{1}

In \autoref{fig:ds:position_test}, the position of the anemometer temperature and humidity sensor is given at its position doing the measurement. Ensure that the measurement is a measure of the transfer function and not the wind noise, a signal to noise measurement is performed as the first part. The signal to noise ration is measured in two ways, one where the wind noise is measured and one where transfer function is measured at high \gls{spl} level whereafter the output level of the speaker is lowered by \dB{10}, and the transfer function is measured again. While the frequency response in the hole frequency range is lowered by \dB{10}, the headroom is at least  \dB{10}. The noise floor is pink with \dB{70} as the maximum at \Hz{2} at all microphone position  with a wind speed of \SI{5}{\meter\per\second}. All measurement is done with signal to noise ration of more than \dB{10}. The signal to noise ratio was visually checked by plotting the measuring result in \matlab. The signal to noise ratio measuring is not saved as a file in the test measurement but is in the final measurement.



\subsection{Ground reflections}\label{sec:des:ground_reflection}
One intended outcome of the windscreen is to block for the ground reflection by the circular wood plate, such that the measuring position can be in the hight of the ear, as explained in \autoref{sub:des:cros_set}. This part of the windscreen fails in blockage all ground reflections in the frequency above \Hz{250} as seen in \autoref{fig:td:mea_1_0_deg}. The measured depth is not the same as in the directionality measurement of the speaker in \autoref{fig:ap:KUDO_25_55_des}. The centre microphone does not have that high peaks and deeps, but the upwards and downwards microphone seems to suffer from a ground reflection in the high frequency, which makes the refraction comparison between the microphone position difficult. One thing that might cause the reflection is the position different from \SI{0}{\degree} vertical of the windscreen with respect to the ground. If the plate is tilted forward, there might be some reflection reaching the microphone. A calculation of the reflection might have helped to justify the ground reflection theory, but since the source is a line source array and not a point source, the ground reflection is not as easy to calculate. There might be thousands of sound path from the line source to the microphone where the sound path length is half the wavelength longer. The following \autoref{fig:td:gro_ref_ilu} illustrate the path calculation difficulties and the forward tilted windscreen. 

\xfig{design/ground_reflection.pdf_t}{The figure shows an illustration of the measured setup and some soundpath}{fig:td:gro_ref_ilu}{1}  


To be able to make a qualified considering to decide if the peaks and depth are due to ground reflection, the measurement is compared to a measurement where the microphone windscreen is situated on the ground and the frequency characteristics in the measuring direction in \autoref{fig:ap:KUDO_25_55_des}.  It has to be noted that the windscreen has a speaker stand connector mounted underneath, which lift the centre of the designed windscreen such that the windscreen cannot lay flat on the ground but is tilted forward. The windscreen was tilted forward by a maximum of \SI{8}{\degree} or less doing all measurement on the ground in all microphone position. The following \autoref{fig:td:mea_1_0_deg_down} shows a frequency response at all three microphones position of the line source array, where the windscreen is placed on the ground and with a non-rotation line source array.

\plot{plot/measuring_1_0_deg_down}{The graph shows the measuring result along with all three microphones, while the microphone is in the windscreen and the windscreen is on the ground. The graph is a mean of three measurements for the upwards and downwards microphone and one measurement for the centre microphone. The mean calculation is done in the time domain where all impulse responses are aligned with cross-correlation}{fig:td:mea_1_0_deg_down}

The frequency depth seen at \Hz{3700} in the downwards direction is due to the directionality characteristics of the line source array and can also be seen in \autoref{fig:ap:KUDO_25_55_des}. The same applies to \Hz{9200} and \SI{10}{\kilo\hertz}

Comparing the measurement where the windscreen is lifted \SI{168}{\centi\meter} from the ground in \autoref{fig:td:mea_1_0_deg} and the measurement in \autoref{fig:td:mea_1_0_deg_down}, it is seen that the first ground reflection comes around \Hz{250} depending on the microphone. This ground reflection is \si{\decibel} wise even for all microphone position and might, therefore, be due to sound wave travels through the windscreen bottom plate. The following arriving ground reflection depends on the microphone position and then might be due to the vertical angle of the designed windscreen. For the upwards microphone, there seems to be highly reflections in the frequency area from \Hz{5500} to \Hz{9500} where comb filtering is present. 
%Comparing the line source array element frequency characteristics, it also shows depth in that frequency range, but not as many and not as low depth as present in the measurement. 
Another common response on all microphone while the windscreen lays on the ground is depth around \SI{10}{\kilo\hertz}. This depth might be due to the lift of the microphone while it is within the modified original windscreen or that the windscreen is tilted forward. Based on the founding in the measurement, ground reflection occurs in the measurement.


\subsection{Frequency differences}\label{sec:des:freq_boost}
Doing the measurement, a frequency differences between \Hz{2000} and \SI{10}{\kilo\hertz} is observed in some measurement and is researched in this section. The frequency differences are observed while the microphone is laying on the ground with and without the windscreen. Furthermore, the wind doing the measurement is less than \SI{1}{\meter\per\second} and therefore, refraction is assumed to be low. A comparison measurement is done, where the windscreen is removed from the centre microphone since the wind stopped at the end of the measurement. The following \autoref{fig:td:mea_1_0_deg_boost} shows the measurement result for all three microphones with windscreen and two measurements where the windscreen is removed from the centre microphone. 

\plot{plot/measuring_1_0_deg_boost}{The graph shows the measuring result along all three microphones, where one of the measurements for the centre microphone is done with designed windscreen setup and two measurements is done without the design windscreen setup. While the microphone is in the designed windscreen, the designed windscreen is on the ground. While the microphone is outside the design windscreen, the microphone lays on a wood plate with the same size as the designed windscreen. }{fig:td:mea_1_0_deg_boost}

It is seen in \autoref{fig:td:mea_1_0_deg_boost} that the depth at \SI{10}{\kilo\hertz} is gone for the centre microphone without the designed windscreen, but the frequency response at the centre and upwards position is generally higher than the other three measurement. The measurement is done with less than \SI{1}{\meter\per\second} of wind speed, and therefore, refraction is assumed to be excluded as a factor of differences. This means that the differences in the windscreen setup might influence the frequency response. 
Three mechanical differences are observed on the windscreen setup doing the measurement. The first is the vertical angle of the windscreen, which was different along with all designed windscreens doing the measurement because the ground is uneven. The ground unevenness is measured afterwards to a maximum of \SI{8}{\degree}. Secondly, the rotational angle of the designed windscreen was also different, along with all designed windscreen. The rotational angle is defined to be \SI{0}{\degree} to the line source array while the line source array points directly into the centre of the windscreen opening, as shown in \autoref{fig:td:win_poi_dir}.

\xfig{design/windscreen_point_direction.pdf_t}{The figure shows an illustration the windscreen vertical \SI{0}{\degree} angle}{fig:td:win_poi_dir}{1}  

As illustrated in \autoref{fig:td:win_poi_dir} no matter the tournament of the speaker, the opening shall point directly to the speaker. In the measurement, the opening was not pointing directly to the speaker. The vertical angle was adjusted with depends on the wind direction, this means that the windscreen at downwards direction is turned left, where the windscreen at upwards direction is turned right. 

The last differences were the placement of the foam on the plate. In the downwards direction the foam is placed more inwards to the centre of the plate while the foam on the other two designed windscreen was placed as shown in \autoref{fig:td:win_poi_dir}


\subsection{Speaker angle}\label{sec:des:measuring_angle}
The angle of the line source array was calculated while the measurement setup was built. In the calculation, the wrong microphone reference point was used. The microphone position which was used in the calculation was while the microphone on the ground and not in the ear hight and therefore, higher tilt angle was calculated.  The measuring angle should have been the given angle in \autoref{sec:pro:test_setup} where the near-field covers both at the ground position and the ear hight position in the distance of \SI{50}{\meter} but the angle was calculated to \SI{5.7}{\degree} and the line source array was tilted to \SI{5}{\degree}. The following \autoref{fig:td:mic_pos_cal_err} shows the microphone positions.

\xfig{design/mic_pos_cal_error.pdf_t}{The figure shows the microphone position doing the calculation of the line source array tilting and the measurement}{fig:td:mic_pos_cal_err}{1}  


While using the \SI{5}{\degree} tilt angle doing the measurement, the hight before the microphone exit above the near-field of the line source array coverage main lobe is calculated to be \SI{63}{\centi\meter}. Therefore, since the microphone was placed \SI{1.68}{\meter} above the ground, the microphone is fare above the near-field. Comparing the \autoref{fig:td:mea_1_0_deg} and \autoref{fig:td:mea_1_0_deg_down} it is clearly seen that the microphone is outside the near-field of the high frequency. Above \Hz{2000}, the \gls{spl} is more than \dB{10} lower in the ear measuring hight compare to the ground position with the same distances to the speaker. 

\subsection{Measuring result}\label{sec:des:measuring_result}
While all error and difficulties are described and is known to disturb the measurement, the measurement indicates that raising the power in the upwards direction, also raising the power in the shadow zone. Comparing the microphone against each other is difficult since the differences in the frequency response between the microphone as described above. Therefore, to extract useful data, the frequency response on the same microphone is compared for \SI{0}{\degree} of tournament, \SI{10}{\degree} of tournament and \SI{20}{\degree} of tournament.   
The first microphone which is compared is the microphone in the upwards direction. The following \autoref{fig:td:mea_1_0_20_deg_up} shows the frequency response for every tournament.

\plot{plot/measuring_1_0_20_upwards}{The graph shows the measuring result for the upwards microphone in three line source array rotation, while the microphone is in the windscreen and the windscreen is in the hight of the ear. The graph is a mean of 10 measurements in all three angles. The mean calculation is done in the time domain where all impulse responses are aligned with cross-correlation}{fig:td:mea_1_0_20_deg_up}

As seen in \autoref{fig:td:mea_1_0_20_deg_up}, while the speaker is turned towards the upwards microphone, the \gls{spl} is raised. The peaks and depth are not at the same frequency, which makes the visually evaluation difficult, but it visually shows that turning the speaker raises the \gls{spl} in some frequency area, especially above \Hz{1000}. The following \autoref{ta:ana:spl_weight_upwards} shows the single number \gls{spl} both non weighted and A-weighted.


\begin{table}[H]
\centering
\caption{The table shows the measured $L_{eq}$ and $L_{A_{eq}}$ \gls{spl} for the upwards microphone}
\begin{tabular}{l|l|l|l}
Speaker angle &  \SI{0}{\degree}  & \SI{10}{\degree}  & \SI{20}{\degree}\\ \hline
       $L_{eq}$       &  \dB{66.64}     &  \dB{67.46} & \dB{68.70} \Tstrut \\
         $L_{A_{eq}}$      &  \dB{63.90}      &  \dB{65.19} & \dB{67.27} \\
\end{tabular}
\label{ta:ana:spl_weight_upwards}
\end{table}


%NON 0 deg = 59.6523, 10 deg = 60.4704, 20 deg = 61.7116
%A 0 deg = 56.9073, 10 deg = 58.2032, 20 deg = 60.2822

The second microphone which is compared is the microphone in the downwards direction. The following \autoref{fig:td:mea_1_0_20_deg_down} shows the frequency response for every tournament.


\plot{plot/measuring_1_0_20_downwards}{The graph shows the measuring result for the downwards microphone in three line source array rotation, while the microphone is in the windscreen and the windscreen is in the hight of the ear. The graph is a mean of 10 measurements in all three angles. The mean calculation is done in the time domain where all impulse responses are aligned with cross-correlation}{fig:td:mea_1_0_20_deg_down}

As seen in \autoref{fig:td:mea_1_0_20_deg_down}, while the speaker is turned, the \gls{spl} is lowered unless the \SI{20}{\degree} above \Hz{2500}. The raise in power comes from the directionality characteristics of the line source array as seen in \autoref{fig:ap:KUDO_25_55_des}. The peaks and depth is ether not at the same frequency which make the visually justment difficult but it is observed i generally that turning the speaker lower the \gls{spl} from \SI{0}{\degree} to \SI{10}{\degree} above \Hz{650}. The following \autoref{ta:ana:spl_weight_downwards} shows the single number \gls{spl} both non weighted and A-weighted.


\begin{table}[H]
\centering
\caption{The table shows the measured $L_{eq}$ and $L_{A_{eq}}$ \gls{spl} for the downwards microphone}
\begin{tabular}{l|l|l|l}
Speaker angle &  \SI{0}{\degree}  & \SI{10}{\degree}  & \SI{20}{\degree}\\ \hline
       $L_{eq}$       &  \dB{66.86}     &  \dB{65.46} & \dB{67.12} \Tstrut \\
         $L_{A_{eq}}$      &  \dB{64.24}      &  \dB{61.59} & \dB{64.36} \\
\end{tabular}
\label{ta:ana:spl_weight_downwards}
\end{table}

%NON 0 deg = 59.8662, 10 deg = 58.4717, 20 deg = 60.1307
%A 0 deg = 57.2538, 10 deg = 54.5998, 20 deg = 57.3711

The third microphone, which is compared, is the microphone in the centre direction. The following \autoref{fig:td:mea_1_0_20_deg_center} shows the frequency response for every tournament.

\plot{plot/measuring_1_0_20_center}{The graph shows the measuring result for the centre microphone in three line source array rotation, while the microphone is in the windscreen and the windscreen is in the hight of the ear. The graph is a mean of 10 measurements in all three angles. The mean calculation is done in the time domain where all impulse responses are aligned with cross-correlation}{fig:td:mea_1_0_20_deg_center}

As seen in \autoref{fig:td:mea_1_0_20_deg_center}, the frequency response does not ether raise or fall markebly while the speaker is turned. The large depth between \Hz{2000} and \Hz{5000} comes from the frequency characteristic of the line source array as seen in \autoref{fig:ap:KUDO_25_55_des}. The following \autoref{ta:ana:spl_weight_center} shows the single number \gls{spl} both non weighted and A-weighted.



\begin{table}[H]
\centering
\caption{The table shows the measured $L_{eq}$ and $L_{A_{eq}}$ \gls{spl} for the center microphone}
\begin{tabular}{l|l|l|l}
Speaker angle &  \SI{0}{\degree}  & \SI{10}{\degree}  & \SI{20}{\degree}\\ \hline
       $L_{eq}$       &  \dB{69.72}     &  \dB{68.79} & \dB{68.77} \Tstrut \\
         $L_{A_{eq}}$      &  \dB{68.64}      &  \dB{67.07} & \dB{67.00} \\
\end{tabular}
\label{ta:ana:spl_weight_center}
\end{table}

%NON 0 deg = 62.7273, 10 deg = 61.8040, 20 deg = 61.7835
%A  0 deg = 61.6521, 10 deg = 60.0828, 20 deg = 60.0100

      
            
\section{research of the problems}
To be able to decide the last few details of the final measurement based on the test and data analysis performed in \autoref{sec:ds:test_of_mes_des}, the frequency response of the designed windscreen is founded in free field condition while rotating and tilting. Secondly, the final windscreen hight is decided. In \autoref{sec:ds:wind_freq_res}, the frequency response of the windscreen is founded where the windscreen is rotated and tilted. 

\subsection{windscreen frequency response}\label{sec:ds:wind_freq_res}
This section aims to research the frequency response of the designed windscreen. It is observed in \autoref{sec:des:freq_boost} that the measurement with the designed windscreen in the centre and downwards direction has a higher frequency response between \Hz{1000} and \SI{10}{\kilo\hertz} compare to two measurements without the designed windscreen at the centre. Therefore it has to be founded if the windscreen by itself have differences in frequency response while the windscreen is rotated or tilted. The windscreen is measured in the anechoic chamber both with tilting and with rotation to research the effect of differences of the windscreen. The measurement is founded in \autoref{app:wind_inf_res}. It is observed in the measurement that the windscreen does not have a frequency differences more than $\pm$\dB{2} while the windscreen is not rotated and not tilted. Furthermore, small forward tilting up to \SI{9}{\degree} only have an attenuation effect above \SI{10}{\kilo\hertz} due to plate reflection. A rotation of \SI{30}{\degree} ether to the left of to the right does attenuate in the frequency range of \Hz{1000} to \Hz{4000}. The frequency response of the windscreen is also measured while removing the foam wedge, but in this configuration produces high reflection in the frequency response. Generality the designed windscreen only change the frequency response up to \dB{2} while the designed windscreen points within $\pm$ \SI{10}{\degree} to the line source array and with tilting beneath \SI{3}{\degree}. The following measurement \autoref{fig:ap:freq_resp_with_foa_til_0_rot_0_dt} shows the different while the microphone only is required with the original modified windscreen and with the designed windscreen
\plot{plot/windscreen_with_foam_tilt_0_rot_0_dt}{The graph shows frequency response of the speaker measured without windscreen and with the designed windscreen with no rotation and no tilting}{fig:ap:freq_resp_with_foa_til_0_rot_0_dt}





\section{Update to the final measurement}
In the final measurement, the designed windscreen is used as it is designed. It is founded that the windscreen does not have any frequency differences more than \dB{2} in the high frequency and small differences in the position make no difference. It is founded that high tilting and high rotation does have an effect of the frequency response and shall be avoided to the final measurement. In the end, the differences between the speaker characteristics with asymmetric make it challenging to compare the side microphone since the \SI{55}{\degree} directionality characteristics have a boost in the higher frequency at \SI{45}{\degree}, as seen is \autoref{fig:ap:KUDO_25_55_des}. Therefore the following points describe the update to the final measurement.

%It is furthermore decided that the ground reflection shall be minimised further that the windscreen does, such that it might be possible to compare between microphone. 

\begin{requirement}\label{req:groundreflection}
    \requirement{The windscreen is positioned at the ground surface hight and not in the ear hight.}
    \argument{This requirement is made according to the founded ground reflection in \autoref{sec:des:ground_reflection} and that the ground reflection shall be minimised such that the microphone might be able to compared among each other. Secondly, it is known from the design of the windscreen that the wind noise is \dB{20} lower near the ground \autoref{sec:ds:wind_noi_att}. Finally, it is assumed in \autoref{sec:ds:wind_noi_att} that the reflection is low in the high frequency while the area is full of the audience which supports that the ground reflection in the high frequency has to be minimised in the measuring point.}
\end{requirement}

\begin{requirement}\label{req:windscreen}
    \requirement{The windscreen shall lay flat on the ground without tilting higher than \SI{6}{\degree} and point \SI{0}{\degree} to the line source array}
    \argument{This requirement is made according to the founded frequency response change in \autoref{sec:ds:wind_freq_res} while the windscreen is tilted and rotated.}
\end{requirement}

\begin{requirement}\label{req:speaker_directionality}
    \requirement{The speaker shall have symmetric directionality characteristics with a beamwidth of \SI{50}{\degree}}
    \argument{This requirement is made according to the founded in the measuring result \autoref{sec:des:measuring_result}, where the differences in directionality characteristics make it difficult to compare the upwards microphone and downwards microphone since the line source array has a \gls{spl} boost in the \SI{45}{\degree} }
\end{requirement}
