\section{Test of measuring design}
The aim of this section is to test the design measurement in a windy day, to outsource problem and error in the measuring design. The test is done in full scale with all six line source array element and in the designed hight. The test was intended to both test the crosswind test design and the parallel wind design, but the weather condition only allowed for crosswind test. After the crosswind test design was tested the wind speed dropped to beneath \SI{1}{\meter\per\second}.
In the crosswind test design three problem and one code error is observed. The code error is a data save bug, where only the direction of the wind at the speaker tower is saved. The wind direction at the microphone position is not saved doing the test measurement. The code bug is fixed for the final measurement. The following three \autoref{}, \autoref{} and \autoref{} explain the observed problem or design failure and propose a solution the the problem. The first problem which is discovered and a solution will be founded is high peaks and deeps in the measurement due to ground reflections. The second problem is non equal frequency response in the designed windscreen, and the third error is the angle of the line source array while measuring in the hight of the ear.  The following three explanation is based on the following \autoref{fig:td:mea_1_0_deg} which shows the frequency response on all three microphone at \SI{0}{\degree} rotation.

\plot{plot/measuring_1_0_deg}{The graph shows the frequency response for all three microphone while the speaker is not rotated}{fig:td:mea_1_0_deg}




\subsection{Ground reflections}
The design of the windscreen is intended to block for the ground reflection of the sound, and lower the wind noise. The noise floor is founded to the lower than the \dB{50} in the frequency of interest and with a wind speed of \SI{2.2}{\meter\per\second}. The measurement is done with \dB{75} which is more than \dB{25} of headroom. The headroom was not measured with higher wind speed due to the weather condition at the measuring day. 

The second effect of the wndscreen is to block for the ground reflection, such that the measuring position is able to be in the hight of the ear as explained in \autoref{}. This part of the windscreen  works, but as the frequency grows the effect seems to worthen since peaks and depth is present in the measurement. The center microphone does not have that high peaks and deeps, but the side microphone seems to suffer from ground reflection, which make the comparison difficult. One thing that might cause the reflection is the position different from \SI{0}{\degree} horizontal of the windscreen in the air. If the plate is tilted forward there might be some reflection reaching the microphone. A calculation of the reflection might have helped to justify the ground reflection theory, but since the source is a line source array and not a point source the ground reflection is not as easy to calculate, since there might be thousands of sound path from the line source to the microphone. the following \autoref{fig:td:gro_ref_ilu} illustrate the path calculation difficulties. 

\xfig{design/ground_reflection.pdf_t}{The figure shows an illustration of the measured setup and some soundpath}{fig:td:gro_ref_ilu}{1}  


To decide if the chosen position is the position which is used in the final test, the measurement is compared to measurement where the microphone windscreen is placed on the ground in stead of calculating the ground reflection frequency. The speaker mount, mounted on the windscreen does that the windscreen cannot lay flat on the ground but is tilted forward. The following \autoref{fig:td:mea_1_0_deg_down} shows a frequency response of the speaker where the windscreen is placed on the ground  

\plot{plot/measuring_1_0_deg_down}{The graph shows }{fig:td:mea_1_0_deg_down}

The depth seen at \Hz{3700} in the downwards direction is due the the directuionality charactica of the speaker and can also be seen in \autoref{fig:ap:KUDO_25_55_des}. The same apples for \Hz{9200} and \SI{10}{\kilo\hertz}

Comparing the measurement where the windscreen is lifted \SI{168}{\centi\meter} from the ground in \autoref{fig:td:mea_1_0_deg} and the measurement in \autoref{fig:td:mea_1_0_deg_down}, it is seen that the first ground reflection comes around \Hz{250} depending on refraction. This ground reflection is \db wise even and might therefore be that the sound wave travels through the plate. The following arriving ground reflection depends highly on the microphone and then might be due to the horizontal angle of the plate. For the upwards refraction microphone there seems to be highly reflection in the frequency area from \Hz{5500} to \Hz{9500} where comb filtering is present, and the general level is lover.

Another common response on all microphone while the windscreen lays on the ground is a depth around \SI{10}{\kilo\hertz}. This might be due to the lift of the microphone while it is within the modified original windscreen or that the windscreen is tilted forward and does not lay flat on the ground. The following section research the frequency boost between \Hz{2000} and \SI{10}{\kilo\hertz} which is present because of the windscreen design. 

\subsection{frequency response}\label{sec:des:freq_boost}
It is discovered doing the measurement that the frequency boost within \Hz{2000} and \SI{10}{\kilo\hertz} produced be the windscreen might be non equal between the microphone. A comparison measurement is done where the windscreen is removed from the center microphone sine the wind stopped in the end of the measurement. The following \autoref{fig:td:mea_1_0_deg_boost} shows the measurement for all three microphone with windscreen and one measurement where the windscreen is removed from the center microphone. The modified original windscreen covering the microphone is also removed.  

\plot{plot/measuring_1_0_deg_boost}{The graph shows }{fig:td:mea_1_0_deg_boost}

It is seen that the depth at \SI{10}{\kilo\hertz} is gone but also that the windscreen amplify the measurement upto \dB{20}. The amplification is not a problem in it self, the arising problem is that the amplification is not even along the windscreen. The measurement is done with less than \SI{1}{\meter\per\second} of wind speed and therefore refraction can be excluded as an factor of differences. This means that differences on the windscreen setup might have a highly influences on the frequency boost. Three differences is observed on the windscreen setup doing the measurement. The first is the horizontal angle of the windscreen, which was different along all windscreen doing the measurement. Secondly the vertical angle of the windscreen was different along all microphone. The explain the vertical angle, the vertical angle is defined to be \SI{0}{\degree} while the speaker points directly into the center of the windscreen opening as shown in \autoref{fig:td:win_poi_dir}.

\xfig{design/windscreen_point_direction.pdf_t}{The figure shows an illustration the windscreen vertical \SI{0}{\degree} angle}{fig:td:win_poi_dir}{1}  

As illustrated in \autoref{fig:td:win_poi_dir} on matter the tournament of the speaker, the opening shall point directly to the speaker. In the measurement the opening was not pointing directly to the speaker as shown in \autoref{fig:td:win_poi_dir}, The vertical angle was adjusted with depends of the wind witch mean that the windscreen at downwards direction was turned left, where the windscreen at upwards direction was turned right. 

The last differences was the placement of the foam on the plate. In the downwards direction the foam was placed more inwards in the plate while the other to foam on the windscreen was placed as shown in \autoref{fig:td:win_poi_dir}


\subsection{Measuring angle}
The angle of the line source array was calculated based from the hight of the ground and not from the hight of the microphone hight.

\xfig{design/mic_pos_cal_error.pdf_t}{The figure shows the microphone position doing the calculation of the line source array tilting and the measurement}{fig:td:mic_pos_cal_err}{1}  


while using the tilt angle doing the measurement, which was \SI{5}{\degree}, the highest hight before the microphone exit above the nearfield of the speaker coverage main lobe is calculated to be \SI{26}{\centi\meter}. Therefore, since the microphone was placed \SI{1.68}{\meter} above the ground the microphone is fare above the nearfield main lobe.

Comparing the \autoref{fig:td:mea_1_0_deg} and \autoref{fig:td:mea_1_0_deg_down} it is clearly seen that the microphone was outside the nearfield main lobe of the high frequency.

\subsection{Measuring result}
While all error and problem is discriped and disturb the measurement, the measurement indicate that raising the power in the main lobe also raising the power in the shadow zone. Comparing the microphone agents each other is difficult sine the differences in the boost between the microphone as discriped in \autoref{sec:des:freq_boost}. Therefore, to extract useful data, the frequency response on the same microphone is compared for \SI{0}{\degree} of turnment, \SI{10}{\degree} of turnment and \SI{20}{\degree} of turnment.   
The first microphone which is compared is the microphone in the upwards direction. The following \autoref{fig:td:mea_1_0_20_deg_up} shows the frequency response for every turnment.
\plot{plot/measuring_1_0_20_upwards}{The graph shows }{fig:td:mea_1_0_20_deg_up}

The graph in \autoref{fig:td:mea_1_0_20_deg_up}

\plot{plot/measuring_1_0_20_downwards}{The graph shows }{fig:td:mea_1_20_0_deg_down}
\plot{plot/measuring_1_0_20_center}{The graph shows }{fig:td:mea_1_0_20_deg_center}




\section{Update to the final test}






