\section{Test of measuring design}
This section aims to test the design measurement in a windy day, to outsource problem and error in the measuring design. The test is done in full scale with all six line source array element and in the designed hight. The test is intended to both test the crosswind test design and the parallel wind design, but the weather condition only allowed for crosswind test. After the crosswind test design was tested the wind speed dropped to beneath \SI{1}{\meter\per\second}.
In the crosswind test design, three problems and one code error are observed. The code error is a data save bug, where only the direction of the wind at the speaker tower is saved. The wind direction at the microphone position is not saved doing the test measurement. The code bug is fixed for the final measurement. The following three \autoref{sec:des:ground_reflection}, \autoref{sec:des:freq_boost} and \autoref{sec:des:measuring_angle} explain the observed difficulties or design failure and propose a solution to them. The first difficulty which is discovered is high peaks and deeps in the measurement due to ground reflections. The second problem is the non-equal frequency response of the designed windscreen, where the third error is the angle of the line source array while measuring in the hight of the ear.  
The explanations is based on the measurement as seen in \autoref{fig:td:mea_1_0_deg} which shows the frequency response on all three microphones at \SI{0}{\degree} rotation.

\plot{plot/measuring_1_0_deg}{The graph shows the frequency response for all three microphone while the speaker is not rotated}{fig:td:mea_1_0_deg}

All three microphone measurement in \autoref{fig:td:mea_1_0_deg} is the mean of 10 measurement. The mean is calculated in time domain by aligning the impulse response with help of cross correlation. The wind speed and wind direction is also the mean of the 10 measurement.


\subsection{Ground reflections}\label{sec:des:ground_reflection}
The design of the windscreen is intended to block for the ground reflection of the sound and lower the wind noise. The noise floor is founded to the lower than \dB{50} in the frequency of interest and with a wind speed of \SI{2.2}{\meter\per\second}. The measurement is done with \dB{75} which is more than \dB{25} of headroom. The headroom is not measured with higher wind speed since the wind speed dropped beneath  \SI{1}{\meter\per\second} at the end of the measurement, and the headroom measurement in the start was not saved. The headroom was only checked visually by some pre-measurement.

The second outcome of the windscreen is to block for the ground reflection, such that the measuring position can be in the hight of the ear as explained in \autoref{}. This part of the windscreen works, but as the frequency grows the effect seems to be worse since higher peaks and depth are present in the measurement compare to the directionality measurement of the speaker in \autoref{fig:ap:KUDO_25_55_des}. The centre microphone does not have that high peaks and deeps, but the side microphone seems to suffer from ground reflection, which makes the comparison between the microphone position difficult. One thing that might cause the reflection is the position different from \SI{0}{\degree} horizontal of the windscreen in the air to the ground. If the plate is tilted forward, there might be some reflection reaching the microphone. A calculation of the reflection might have helped to justify the ground reflection theory, but since the source is a line source array and not a point source the ground reflection is not as easy to calculate since there might be thousands of sound path from the line source to the microphone. the following \autoref{fig:td:gro_ref_ilu} illustrate the path calculation difficulties. 

\xfig{design/ground_reflection.pdf_t}{The figure shows an illustration of the measured setup and some soundpath}{fig:td:gro_ref_ilu}{1}  


To be able to make a qualified considering to decide if the peaks and depth are due to ground reflection, the measurement is compared to a measurement where the microphone windscreen is situated on the ground and the frequency characteristics in the measuring direction in \autoref{fig:ap:KUDO_25_55_des}.  It have to be noted that the speaker mount, mounted on the windscreen does that the windscreen cannot lay flat on the ground but is tilted forward. The following \autoref{fig:td:mea_1_0_deg_down} shows a frequency response on at all three microphones position of the line source array, where the windscreen is placed on the ground and with a non-rotation line source array.

\plot{plot/measuring_1_0_deg_down}{The graph shows the measuring result along all three microphone, while the microphone is in the windscreen and the windscreen is on the ground. The graph is a mean of three measurement for the upwards and downwards microphone and one measurement for the center position. The mean calculation is done in time domain where the all impulse responses is allined with cross-correlation}{fig:td:mea_1_0_deg_down}

The depth seen at \Hz{3700} in the downwards direction is due to the directionality characteristics of the speaker and can also be seen in \autoref{fig:ap:KUDO_25_55_des}. The same applies to \Hz{9200} and \SI{10}{\kilo\hertz}

Comparing the measurement where the windscreen is lifted \SI{168}{\centi\meter} from the ground in \autoref{fig:td:mea_1_0_deg} and the measurement in \autoref{fig:td:mea_1_0_deg_down}, it is seen that the first ground reflection comes around \Hz{250} depending on the microphone. This ground reflection is \db wise even for all microphone position and might, therefore, be due to sound wave travels through the windscreen bottom plate. The following arriving ground reflection depends highly on the microphone position and then might be due to the horizontal angle of the plate. For the upwards refraction microphone, there seems to be highly reflections in the frequency area from \Hz{5500} to \Hz{9500} where comb filtering is present. Comparing the line source frequency characteristics, it also shows depth in that frequency range, but as high depth as present in the measurement. The measurement has more that \dB{15} attenuation where the line source frequency characteristics only have \dB{6} attenuation.

Another common response on all microphone while the windscreen lays on the ground is depth around \SI{10}{\kilo\hertz}. This depth might be due to the lift of the microphone while it is within the modified original windscreen or that the windscreen is tilted forward and does not lay flat on the ground. 

Based on the founding in the measurement, ground reflection occur in the measurement.


\subsection{Frequency boost}\label{sec:des:freq_boost}
It is discovered doing the measurement that the frequency boost within \Hz{2000} and \SI{10}{\kilo\hertz} produced be the windscreen might be non-equal between the microphone. A comparison measurement is done, where the windscreen is removed from the centre microphone since the wind stopped at the end of the measurement. The following \autoref{fig:td:mea_1_0_deg_boost} shows the measurement for all three microphones with windscreen and one measurement where the windscreen is removed from the centre microphone. The original modified windscreen covering the microphone is also removed. 

\plot{plot/measuring_1_0_deg_boost}{The graph shows the measuring result along all three microphone, where one of the measurement for the center microphone is done with designed windscreen setup and one measurement is done without the design windscreen setup but with the modified original windscreen. While the microphone is in the designed windscreen, the windscreen is on the ground. While the microphone is outside the design windscreen, the microphone lays in the grass.  The graph is a mean of three measurement for the upwards and downwards microphone and one measurement for the center position for both center measurement. The mean calculation is done in time domain where the all impulse responses is allined with cross-correlation}{fig:td:mea_1_0_deg_boost}

It is seen in \autoref{fig:td:mea_1_0_deg_boost} that the depth at \SI{10}{\kilo\hertz} is gone but also that the windscreen amplify the measurement up to \dB{20}. The amplification is not a problem in itself, and the arising problem is that the amplification is not even along with the windscreens. The measurement is done with less than \SI{1}{\meter\per\second} of wind speed, and therefore refraction can be excluded as a factor of differences. This means that differences on the windscreen setup might influences highly on the frequency boost. 
Three mechanical differences are observed on the windscreen setup doing the measurement. The first is the horizontal angle of the windscreen, which was different along with all windscreens doing the measurement. Secondly, the vertical angle of the windscreen was also different along with all microphones. The vertical angle is defined to be \SI{0}{\degree} to the speaker while the speaker points directly into the centre of the windscreen opening as shown in \autoref{fig:td:win_poi_dir}.

\xfig{design/windscreen_point_direction.pdf_t}{The figure shows an illustration the windscreen vertical \SI{0}{\degree} angle}{fig:td:win_poi_dir}{1}  

As illustrated in \autoref{fig:td:win_poi_dir} no matter the tournament of the speaker, the opening shall point directly to the speaker. In the measurement the opening was not pointing directly to the speaker as shown in \autoref{fig:td:win_poi_dir}, The vertical angle was adjusted with depends on the wind witch mean that the windscreen at downwards direction was turned left, where the windscreen at upwards direction was turned right. 

The last differences were the placement of the foam on the plate. In the downwards direction the foam was placed more inwards in the plate while the other to foam on the windscreen was placed as shown in \autoref{fig:td:win_poi_dir}


\subsection{Speaker angle}\label{sec:des:measuring_angle}
The angle of the line source array was calculated based on the ground and not from the hight of the microphone hight. The following \autoref{fig:td:mic_pos_cal_err} shows the microphone position calculation.

\xfig{design/mic_pos_cal_error.pdf_t}{The figure shows the microphone position doing the calculation of the line source array tilting and the measurement}{fig:td:mic_pos_cal_err}{1}  


While using the tilt angle doing the measurement, which was \SI{5}{\degree}, the highest hight before the microphone exit above the nearfield of the speaker coverage main lobe is calculated to be \SI{62}{\centi\meter}. Therefore, since the microphone was placed \SI{1.68}{\meter} above the ground the microphone is fare above the nearfield main lobe.Comparing the \autoref{fig:td:mea_1_0_deg} and \autoref{fig:td:mea_1_0_deg_down} it is clearly seen that the microphone was outside the nearfield main lobe of the high frequency. Above \Hz{2000} the \gls{spl} is more than \dB{10} lower in the ear measuring hight compare to the ground position with the same distances to the speaker. 

\subsection{Measuring result}
While all error and difficulties are described and disturb the measurement, the measurement indicates that raising the power in the main lobe also raising the power in the shadow zone. Comparing the microphone agents each other is difficult since the differences in the boost between the microphone as described in \autoref{sec:des:freq_boost}. Therefore, to extract useful data, the frequency response on the same microphone is compared for \SI{0}{\degree} of tournament, \SI{10}{\degree} of tournament and \SI{20}{\degree} of tournament.   
The first microphone which is compared is the microphone in the upwards direction. The following \autoref{fig:td:mea_1_0_20_deg_up} shows the frequency response for every tournament.

\plot{plot/measuring_1_0_20_upwards}{The graph shows the measuring result for the upwards microphone in three speaker angle, while the microphone is in the windscreen and the windscreen is in the hight of the ear. The graph is a mean of 10 measurement in all three angle. The mean calculation is done in time domain where the all impulse responses is allined with cross-correlation}{fig:td:mea_1_0_20_deg_up}

As seen in \autoref{fig:td:mea_1_0_20_deg_up}, while the speaker is turned, the \gls{spl} seems to raise. The peaks and depth are not at the same frequency which makes the visually evaluation difficult, but it visually shows that turning the speaker raises the \gls{spl} in some frequency area especially above \Hz{1000}. The following \autoref{ta:ana:spl_weight_upwards} shows the single number \gls{spl} both non weighted and A-weighted.


\begin{table}[H]
\centering
\caption{The table shows the measured $L_{eq,5}$ and $L_{A_{eq,5}}$ \gls{spl} for the upwards microphone}
\begin{tabular}{l|l|l|l}
Speaker angle &  \SI{0}{\degree}  & \SI{10}{\degree}  & \SI{20}{\degree}\\ \hline
       $L_{eq,5}$   	&  \dB{59.65} 	&  \dB{60.47} & \dB{61.71} \Tstrut \\
         $L_{A_{eq,5}}$  	&  \dB{56.91}  	&  \dB{58.20} & \dB{60.28} \\
\end{tabular}
\label{ta:ana:spl_weight_upwards}
\end{table}


%NON 0 deg = 59.6523, 10 deg = 60.4704, 20 deg = 61.7116
%A 0 deg = 56.9073, 10 deg = 58.2032, 20 deg = 60.2822

The second microphone which is compared is the microphone in the downwards direction. The following \autoref{fig:td:mea_1_0_20_deg_down} shows the frequency response for every tournament.


\plot{plot/measuring_1_0_20_downwards}{The graph shows the measuring result for the downwards microphone in three speaker angle, while the microphone is in the windscreen and the windscreen is in the hight of the ear. The graph is a mean of 10 measurement in all three angle. The mean calculation is done in time domain where the all impulse responses is allined with cross-correlation}{fig:td:mea_1_0_20_deg_down}

As seen in \autoref{fig:td:mea_1_0_20_deg_down}, while the speaker is turned, the \gls{spl} seems to fall unless the \SI{20}{\degree} above \Hz{2500}. The raise in power comes from the directionality charatstics of the line source array as seen in \autoref{fig:ap:KUDO_25_55_des}. The peaks and depth is ether not at the same frequency which make the visually justment difficult but it generally that turning the speaker lower the \gls{spl} from \SI{0}{\degree} to \SI{10}{\degree} above \Hz{650}. The following \autoref{ta:ana:spl_weight_downwards} shows the single number \gls{spl} both non weighted and A-weighted.


\begin{table}[H]
\centering
\caption{The table shows the measured $L_{eq,5}$ and $L_{A_{eq,5}}$ \gls{spl} for the downwards microphone}
\begin{tabular}{l|l|l|l}
Speaker angle &  \SI{0}{\degree}  & \SI{10}{\degree}  & \SI{20}{\degree}\\ \hline
       $L_{eq,5}$   	&  \dB{59.87} 	&  \dB{58.47} & \dB{60.13} \Tstrut \\
         $L_{A_{eq,5}}$  	&  \dB{57.25}  	&  \dB{54.60} & \dB{57.37} \\
\end{tabular}
\label{ta:ana:spl_weight_downwards}
\end{table}

%NON 0 deg = 59.8662, 10 deg = 58.4717, 20 deg = 60.1307
%A 0 deg = 57.2538, 10 deg = 54.5998, 20 deg = 57.3711

The third microphone which is compared is the microphone in the center direction. The following \autoref{fig:td:mea_1_0_20_deg_center} shows the frequency response for every tournament.

\plot{plot/measuring_1_0_20_center}{The graph shows the measuring result for the center microphone in three speaker angle, while the microphone is in the windscreen and the windscreen is in the hight of the ear. The graph is a mean of 10 measurement in all three angle. The mean calculation is done in time domain where the all impulse responses is allined with cross-correlation}{fig:td:mea_1_0_20_deg_center}

As seen in \autoref{fig:td:mea_1_0_20_deg_center}, the frequency response does not ether raise or fall markebly while the speaker is turned. The large depth between \Hz{2000} and \Hz{5000} comes from the frequency charatestic of the line source array as seen in \autoref{fig:ap:KUDO_25_55_des}. The following \autoref{ta:ana:spl_weight_center} shows the single number \gls{spl} both non weighted and A-weighted.



\begin{table}[H]
\centering
\caption{The table shows the measured $L_{eq,5}$ and $L_{A_{eq,5}}$ \gls{spl} for the center microphone}
\begin{tabular}{l|l|l|l}
Speaker angle &  \SI{0}{\degree}  & \SI{10}{\degree}  & \SI{20}{\degree}\\ \hline
       $L_{eq,5}$   	&  \dB{62.73} 	&  \dB{61.80} & \dB{61.78} \Tstrut \\
         $L_{A_{eq,5}}$  	&  \dB{61.65}  	&  \dB{60.08} & \dB{60.01} \\
\end{tabular}
\label{ta:ana:spl_weight_center}
\end{table}

%NON 0 deg = 62.7273, 10 deg = 61.8040, 20 deg = 61.7835
%A  0 deg = 61.6521, 10 deg = 60.0828, 20 deg = 60.0100

      
            
\section{Update to the final test}






