\section{proposal of solution to the cross wind problem}\label{sec:td:pro_sol_pro}

The aim of this section is to propose a solution to the problem founded in the cross wind measurement in \autoref{sec:ana:atm_ref}. The solution is based on the problem statement in \autoref{ch:statement}. The solution is a more homogeneous \gls{spl} coverage in the coverage area of the speaker without wind. In other word, the line source array has a frontal horizontal directional angle defined as the \dB{-6} limit of main pressure lobe. A line source array main lobe is given in the horizontal degree as an addition of the main lobe from the frontal direction to the side  and can both be symmetric and asymmetric, depending on the line source array element. The speaker which will be used to design the solution is a L-Acoustics KUDO line source array where the main lobe coverage can be controlled mechanical. It is both possible to make the main lobe symmetric and asymmetric on this line source element. The following the \autoref{fig:td:kud_cov} shows both the wide and narrow symmetric main lobe option of the KUDO. The asymmetric coverage can be founded in \citep{KUDO_data}

\fig{kudo_coverage.pdf}{The graph shows the symmetric coverage area of the L-Acoustics KUDO line source array \citep{KUDO_data}.}{fig:td:kud_cov}{0.9}

The mechanical coverage solution in the KUDO as well as other line source array element is not made for wind problems but for neighbog desterbines and higher \gls{spl} in the main lobe of the high frequency. All solution used today is only posible to change by hand and is not electrical controlled. 

The proposal solution is to be able to steer the main lobe horizontal direction of the line source array electronical. As beeing able to steer the main lobe horizontal direction of the line source array, the main lobe can be steered more up agents the direction of the coverage area where the wind attenuate.  The crosswind problem is not as drastical close to the speaker, so the line source array which shall be able to be controlled is the coverage area is the element which cover the audience in back. The solution is based on a changeable main lobe which is as narrow as possible to archive as high \gls{spl} and the audience and as low neighbog desterbines as possible. The following \autoref{fig:td:geo_fin_sol} shows a graphical illustrate of the proposal solution to archive a more homogeneous \gls{spl} coverage area in the frontal direction of the speaker without wind.

\xfig{design/geo_find_solution.pdf_t}{The figure shows the wanted direction of the sound coverage area after the effect of crosswind. C is the speed of wind in cross direction of the frontal direction of the speaker. A and B is the angle that needs to be founded. On the figure the angle are equal but that might not be true}{fig:td:geo_fin_sol}{1} 

The gold is then to search A\si{\degree} and B\si{\degree} based on wind speed C \si{\meter\per\second} as shown in \autoref{fig:td:geo_fin_sol} such that the \gls{spl} coverage differences is minimized. The angle of A and B in the figure is equal, this might not be true in for the solution.





\section{Designing the measurement}

The aim of this section is to design a test on a non modified line source array to test the proposal solution from \autoref{sec:td:pro_sol_pro}. 


The speaker is chosen to be adjusted to the narrow main lobe because it is assumed that the distance from the audience to the speaker is so large that the wide angle goes beyond the audience area.

Because of limitation, the speaker is flown in a hight of \SI{6}{\meter}. 

To measure the \gls{spl} coverage of the speaker a flat area with mown grass is chosen to be used. The optimal area area without any building or trees might not be posible, therefore blockage or sound reflaction surface other than the ground is only allowed to be present in the double of distance compare to the distance from the speaker to the microphone. Based on the refraction effect versus distance founded in \autoref{sec:ana:atm_ref}. The distance from the speaker to the microphone array is chosen to be \SI{50}{\meter}. The distance is based on the experience of the author described in \autoref{sec:ana:aut_exp_con} and the founded refraction effect in \autoref{sec:ana:atm_ref}. It was founded that the refraction effect should be minimal at a distance of \SI{50}{\meter} when the speed of wind is \SI{5}{\meter\per\second}. 


To keep the wind speed realistic for measurement and for concert, but still having wind pressent, the wind speed doing the measurement is limited in the range for average \SI{5}{\meter\per\second} to \SI{10}{\meter\per\second}. Less average wind speed than \SI{5}{\meter\per\second} is avoided to ensure measureble effect of the wind on sound propagation. The higher limit of the \SI{10}{\meter\per\second} is chosen to ensure that the speaker tower is safe at the hight at \SI{6}{\meter}. The limited size of the setup makes the setup wind (følsom) because it is not puttet up as a cube but only as a surface. 


where the refraction at \SI{110}{\meter} already starts at \Hz{400}. 


The area is without  The resend to use this  









\section{Technical solution}

The proposal solution is therefore a electronical controlled angle and width of the main based on the strangth of the wind and the coverage distance to the audience. 





