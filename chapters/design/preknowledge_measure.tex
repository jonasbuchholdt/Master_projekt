\section{Measuring in inhomogeneous atmosphere}
The aim of this section is to gain pre knowledge about outdoor measurement, such that effecting condition doing measurement is controlled. Outdoor measurement has two main sources of disturbing which is present at concert area distances. At the concert that might be buildings or other non-horizontal reflection surfaces, those surfaces is excluded from this explanation. The first main sources of measurement disturbing is ground reflection. The ground reflection is covered as the first part of the chapter. The second sources for measurement disturbing is wind noise which have to be controlled doing the measurement to gain measurement of the line source array and not the wind noise. The explanation starts with the former.

\section{Ground reflection}\label{pre:ground_ref}
Ground reflection is a reflection of the sound while the wave reac the ground. The ground reflection is present while the source is above the ground hight or downwards refraction is present. In a measuring system with microphone, the ground reflection is present at the microphone while the microphone is lifted from the ground or downwards refraction is present. A ground reflection of sound gives a time receiving difference at the microphone of the same signal while the direct sound path is different from the ground reflected sound path. The following \autoref{} shows the block diagram of sound path with presents of ground reflection.


\xfig{design/block_gnd_ref.pdf_t}{The figure shows a block diagram of ground reflection from the source to the receiver}{fig:pn:block_gnd_ref}{0.4}  

As shown in the block diagram in \autoref{fig:pn:block_gnd_ref} the delivered \gls{spl} to the receiver depends on the delta time between the sound path and the ground reflection attenuation of the sound. The case where $G=1$ is never true because it require that the reflecting surface is \SI{100}{\percent} reflecting and the source is an infinity high and wide source such that the attenuation with respect to path distances i zero. In the ideal case ground reflection can at a maximum give the double of power or \dB{6} or the sound can fully be cancled. The canceling occore at its maximum when the sound path of the ground reflection is half the wave length longer and one half the wave length and so forth. The maximum amplification occore while the sound path is from the ground reflection is one wave length longer, then two wavelength longer and so forth. Since the wavelength is propertional to the frequency, then if the ground reflect all frequency, the ground reflection gives a comb filter in the frequency response. To calcualate the wavelength extention factor, the following \autoref{pre:fac_wav}

\begin{equation}\label{pre:fac_wav}
\text{fac} = \frac{m \cdot f}{v}
\end{equation}  

\startexplain
\explain{\text{fac}}{is the factor of the wave length it gets longer at the given path distance and frequency}{1}
\explain{f}{is the frequency}{\hertz}
\explain{m}{is the path differences}{\meter}
\explain{v}{is the speed of sound}{\meter\per\second}
\stopexplain


While calculating the factor of the wave length that its get longer by a given sound path differences, it is seen that the frequency dependent maximum attenuation is present for third times the first frequency where the first maximum attenuation is present. Then it is the fifth, then seventh and so forth. This is all the odd dubling of the first frequency. In the maximum amplification it is all the  even dubling of the first frequency.


\section{Wind noise}\label{pre:wind_noise}
Wind noise is noise produced by the pressure fluctuation in turbulence wind flow \citep{doi:10.1121/1.4780400}. The frequency spectrum of the wind noise is pink, therefore the highest frequency component is in the low frequency range. The wind noise therefore might not produces any headroom problem in the frequency where refrection occore, since this is in the middle and high frequency range. The problem with wind noise doing measurement is that the wind noise pressure level in the low frequency can be as high as the microphone or preamp overload. A overload of the microphone or preamp produces distortion in the measurement. Distortion from the preamp is present is all output frequency, since this distortion is produces by maximum rail voltage on the preamp. While the maximum rail voltage is atained, the output is squared. Microphone distortion is as speaker distortion, The membran excites its linear excortion range and the output cutve is squesed. To handle wind noise a windscreen is used. All measurment while wind is present in this thieses is done where the microphone is covered with the original beloning windscreen.  



 