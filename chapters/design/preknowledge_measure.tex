\section{Measuring in inhomogeneous atmosphere}
This section aims to gain pre-knowledge about outdoor measurement, such that the microphone affecting inhomogeneous actors doing measurement is controlled. Outdoor measurement has two primary sources of disturbing, which is present at concert area distances. At a concert, the area might be surrounded by buildings or other non-horizontal reflection surfaces. Those surfaces are excluded in this explanation. The analysis is addressed as following  

\begin{enumerate}
\item The first primary sources of measurement disturbing are ground reflection. The ground reflection is covered in \autoref{pre:ground_ref}.
\item The second sources for measurement disturbing are wind noise, which has to be controlled doing the measurement to gain measurement of the line source array and not the wind noise. The wind noise is covered in \autoref{pre:wind_noise}
\end{enumerate}



\section{Ground reflection}\label{pre:ground_ref}
Ground reflection is a reflection of the sound by the ground surface. The ground reflection is present while the source is above the ground or downwards refraction is present. In a measuring system with a microphone, the ground reflection is present at the microphone while the microphone is lifted from the ground or downwards refraction is present. A ground reflection of sound gives a time receiving difference at the microphone of the same signal while the direct sound path is different from the ground reflected sound path. The following \autoref{fig:pn:block_gnd_ref} shows a block diagram of the sound path with the presence of ground reflection.


\xfig{design/block_gnd_ref.pdf_t}{The figure shows a block diagram of ground reflection from the source to the receiver.}{fig:pn:block_gnd_ref}{0.4}  
\startexplain
\explain{\Delta\text{T}}{is the path difference in time.}{\second}
\explain{G}{is the gain.}{1}
\stopexplain


As shown in the block diagram in \autoref{fig:pn:block_gnd_ref}, the delivered \gls{spl} to the receiver depends on the delta time $\Delta$T between the sound path and the ground reflection attenuation of the sound. The case where $G=1$ is never true because it requires that the reflecting surface is \SI{100}{\percent} reflecting and the source is an infinity high and wide source such that the attenuation concerning the path distances is zero. In the ideal case, ground reflection can at a maximum give the double of power, \dB{6} or the sound is entirely cancelled. The cancelling occurs when the sound path of the ground reflection is half the wavelength longer, one half the wavelength, and so on. The maximum amplification occurs while the sound path from the ground reflection is one wavelength longer, then two wavelengths longer, and so on. Since the wavelength is proportional to the frequency, the ground reflection gives a comb filter in the frequency response. To calculate the wavelength extension $N$ with respect to a given path difference in meter, the following \autoref{pre:fac_wav} is used.

\begin{equation}\label{pre:fac_wav}
N = \frac{m}{\lambda}
\end{equation}  

\startexplain
\explain{N}{is the number of wavelength the ground reflected path gets longer.}{1}
\explain{\lambda}{is the wavelength.}{\meter}
\explain{m}{is the path differences.}{\meter}
\stopexplain


While calculating the wavelength extension factor its get longer by a given sound path differences, it is seen that the frequency dependent maximum attenuation is present for all odd doubling frequency above the first maximum attenuation. In the maximum amplification, it is all the even doubling of the first maximum amplification.


\section{Wind noise}\label{pre:wind_noise}
Wind noise is noise produced by pressure fluctuation in turbulence wind flow \citep{doi:10.1121/1.4780400}. The frequency spectrum of the wind noise is pink. Therefore, the highest frequency component is in the low frequency range. The wind noise, therefore, might not produce any headroom problem in the frequency where refraction occurs, since this is in the middle and high frequency range. The problem with wind noise doing measurement is that the wind noise pressure level in the low frequency can be as high as the microphone or preamp overload. An overload of the microphone or preamp produces distortion in the measurement. Distortion from the preamp is present in all output frequency since the output signal is at the output rail voltage on the preamp. While the maximum rail voltage is attained, the output becomes squared. Microphone distortion is as speaker distortion, The membrane excites its linear excursion range, and the output curve is squeezed. To handle the wind noise, a windscreen is used. All measurement while the wind is present in this thesis is performed where the microphone is covered with the original belonging windscreen.  


 