\section{proposal of solution to the wind problem}\label{sec:td:pro_sol_pro}

This section aims to propose a solution to the problem founded in the crosswind measurement \autoref{sec:ana:atm_ref} and the problem statement in \autoref{ch:statement}. To be able to find a solution to the problem, the optimal condition is defined in \autoref{sec:des:opt_con}. Then a proposal solution to the crosswind is defined in \autoref{sec:des:pro_cross} and lastly the a proposal solution to the parallel wind is defined in \autoref{sec:des:pro_para}


\section{Optimality condition}\label{sec:des:opt_con}

To be able to search for a solution and design a test to research if the proposal solution has the whiched effect on the coverage area, the optimal condition is defined in this section. The optimal condition is as simple as the \gls{spl} coverage in the coverage area of the speaker without wind. In other words, the line source array has a frontal horizontal directional angle defined as the \dB{-6} limit of the main pressure lobe. The line source array main lobe is given in the horizontal degree as an addition of the main lobe from the frontal direction to both side and can both be symmetric and asymmetric, depending on the line source array element. The following \autoref{fig:td:mail_lobe} shows an illustration of the main lobe. 

\xfig{design/main_lobe.pdf_t}{The figure shows the pressure limit which defined the main lobe of the line source array and the coverage area without wind}{fig:td:mail_lobe}{1} 

As illustrated in \autoref{fig:td:mail_lobe} the coverage area is a parabolic surface which is limited as the \dB{-6} coverage limit of the line source array. This is the coverage area with no wind effect and is the coverage area which is defined as the optimal condition. The solution to the crosswind and the parallel wind therefore is a way to be able to adjust the coverage area such that the line source array is able to eliminate the effect of the wind and cover the area as without wind. To be able to eliminate the wind effect at the audience area, audience audience area have to be defined To defind the audience area, a questioner is made among the large sound rental company in Denmark which ask for audience area to concert. The questioner is founded in \autoref{ap:ques}. The gold of the questioner is to find the highest coverage distance from the line source array. The founded maximum distances before delay tower is approximate \SI{60}{\meter}, and furthermore, the wind might be stronger than \SI{5}{\meter\per\second} to a concert but wind above \SI{10}{\meter\per\second} to \SI{12}{\meter\per\second} will stops the concert. Therefore the defined coverage area as shown in \autoref{fig:td:mail_lobe} is \SI{60}{\meter} from the stage front.



\section{Proposal solution to crosswind}\label{sec:des:pro_cross}
The crosswind problem is shown in \autoref{sec:ana:atm_ref} to highly modified the coverage area. Agents the the wind the upwards refraction is shown to attenuate the sound more than \dB{6} A-weighted at a distance of only \SI{25}{\meter} and a mean wind strange of \SI{14}{\meter\per\second}. Furthermore, it is founded that the shadow zone \gls{spl} depends on the \gls{spl} in the sound path, because the wind eddies, eddies the sound intro the shadow zone. It is then research if adding more power intro the upwards refraction direction also ads more power intro the shadow zone by the wind eddies. The following \autoref{fig:td:eddies} illustrate the eddies theory in upwards refraction.

\xfig{design/eddies.pdf_t}{The figure shows the sound path above the zhadow zone and inside the shadow zone produced by the eddies}{fig:td:eddies}{1} 

The proposal solution is then to steer more power intro the direction of upwards refraction and less power intro the front and in the direction of downwards refraction to research the eddies theory shown in \autoref{fig:td:eddies}. The following \autoref{fig:td:geo_fin_sol} shows a graphical illustration of the proposed solution to archive a more homogeneous \gls{spl} in the coverage area of the line source array with wind.

\xfig{design/geo_find_solution.pdf_t}{The figure shows the wanted direction of the sound coverage area after the effect of crosswind. C is the speed of wind in cross direction of the frontal direction of the speaker. A and B is the main lobe angle that needs to be founded. On the figure the angle are equal but that might not be true}{fig:td:geo_fin_sol}{1} 

The gold is then to search A\si{\degree} and B\si{\degree} based on wind speed C \si{\meter\per\second} and the optimal coverage area as shown in \autoref{fig:td:geo_fin_sol} such that the \gls{spl} coverage differences is optimized. As founded in \autoref{sec:ana:atm_ref} the refraction is frequency dependent, therefore, finding the optimal  A\si{\degree} and B\si{\degree} might not be possible for all frequency. Furthermore the refraction in the low frequency is nearly zero for the distance present at the concert. 




%Furthermore, the distance between the line source element to the audience raises with respect to the hight of the line source element. Therefore, the signal to every speaker has to be controlled individually with a different angle and frequency response. 


%The proposed solution is to be able to steer the main lobe horizontal direction of the line source array adaptive by measuring tools and electronical actuators. 
%As being able to steer the main lobe horizontal direction of the line source array, the main lobe can be steered more in the direction of the coverage area where the wind attenuate the \gls{spl} by upwards refraction.  

%Based on the latter fact, the refraction effect below \Hz{1000} in measurement \autoref{sec:ana:atm_ref} and the control possibility of the line source array \autoref{fig:td:kud_cov}, the lower frequency control criteria is chosen be be \Hz{1000}. 

%The frequency above \Hz{1000} is highly affected by refraction and the shadow distances get lower as the frequency rises. 
%The frequency area above \Hz{1000} which is the most important is the frequency area where the highest intelligibility frequency spectra are present. Therefore the optimal frequency range for adaptive main lobe adjustment is chosen to be from \Hz{1000} to \Hz{8000}. The frequency above is not as crucial for conveying information to the audience. 


\section{Proposal solution to parallel wind}\label{sec:des:pro_para}
The above proposal solution deals with the crosswind problem. When the wind direction change such that the wind comes from the back audiences to the stage, or in other words, is parallel with the frontal direction of the speaker, another theory is research than the eddies theory. The resand to search another solution is that using the eddies theory in parallel wind require that the power from the line source array is raised. The proposal solution is then to move the shadow zone instead of reasing the power in the the shadow. To move the shadow zone the idea is to change the vertical angle of the main lobe, such that the upper speaker ether point more downwards or upwards for upwards refraction or downwards refraction respectively. Therefore if the upper speaker points more downwards the energy from the speaker might arrive at the ground where else if the line source element is pointing parallel to the ground, the energy never enters the ground surface. The following \autoref{fig:td:sha_zon_opt} shows the proposal solution to parallel wind refraction.  

\xfig{design/shadow_zone_optimising.pdf_t}{The figure shows the proposal solution to the upwards refraction. The upper part of the figure shows the lower vertical main lobe ray while the array is orthogonal to the ground where the lower part of the figure shows the lower vertical main lobe ray while the line array is angled more downwards}{fig:td:sha_zon_opt}{1} 
 
The shadow zone distance is depending on the hight of the line source array from the ground within the limited hight of flying points on the stage. As higher the line source array is flown, as higher the distance is before the shadow zone is present while the vertical angle is optimised to the audience area. 
  
  
  




%\subsection{The optimal solution}
%This section starts describing the optimality criteria for the crosswind and then the optimality criteria for parallel wind. One thing that the solution have common for both wind situation is that the refraction is frequency dependent and therefore the frequency range that shall be optimised for is the same. It is founded in (ADD SOMETHING ABOUT INTELLIGIBILITY) that the most important frequency range for information dissemination is the frequency range from \Hz{500} to \Hz{8000}. Therefore the optimality criteria frequency range for both solution is weighted as the ineligibility frequency range. 

%\paragraph{Crosswind} The optimality criteria for crosswind is that the \gls{spl} coverage differences within the main lobe coverage after frequency weighting is minimised and symmetric around the frontal direction. Furthermore, the \gls{spl} differences between the frontal direction and the main lobe limit is of a maximum of \dB{6}

%\paragraph{Parallel wind} The optimality criteria for the parallel wind is that the shadow zone is moved thunder back. Therefore the shadow zone is founded and the microphone is placed in the shadow zone. The optimality criteria are then the angle where the \gls{spl} in the shadow zone is highest in the weighted frequency range. 
