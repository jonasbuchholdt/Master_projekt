\section{proposal of solution to the cross wind problem}\label{sec:td:pro_sol_pro}

This section aims to propose a solution to the problem founded in the crosswind measurement \autoref{sec:ana:atm_ref} and the problem statement in \autoref{ch:statement}. 
The solution is a more homogeneous \gls{spl} coverage in the coverage area of the speaker without wind. In other words, the line source array has a frontal horizontal directional angle defined as the \dB{-6} limit of the main pressure lobe. The line source array main lobe is given in the horizontal degree as an addition of the main lobe from the frontal direction to the side and can both be symmetric and asymmetric, depending on the line source array element. The following \autoref{fig:td:mail_lobe} shows an illustration of the main lobe. 

\xfig{design/main_lobe.pdf_t}{The figure shows the}{fig:td:mail_lobe}{1} 

The proposed solution is to be able to steer the main lobe horizontal direction of the line source array adaptive by measuring tools and electronical actuators. 
As being able to steer the main lobe horizontal direction of the line source array, the main lobe can be steered more in the direction of the coverage area where the wind attenuate the \gls{spl} by upwards refraction.  The crosswind problem is not as drastically close to the speaker, so the line source array which shall be able to be controlled is the line source array which covers the audience in back. The solution is therefore based on an adaptive main lobe, which is controlled such that the power of the main lobe is pointing against the audience with lowest \gls{spl}. The following \autoref{fig:td:geo_fin_sol} shows a graphical illustration of the proposed solution to archive a more homogeneous \gls{spl} coverage area in the frontal direction of the speaker with the wind.

\xfig{design/geo_find_solution.pdf_t}{The figure shows the wanted direction of the sound coverage area after the effect of crosswind. C is the speed of wind in cross direction of the frontal direction of the speaker. A and B is the main lobe angle that needs to be founded. On the figure the angle are equal but that might not be true}{fig:td:geo_fin_sol}{1} 

The gold is then to search A\si{\degree} and B\si{\degree} based on wind speed C \si{\meter\per\second} as shown in \autoref{fig:td:geo_fin_sol} such that the \gls{spl} coverage differences is minimised. The angle of A and B in the figure is equal, and this might not be true for the solution. As founded in \autoref{sec:ana:atm_ref} the refraction is frequency dependent, therefore, finding the optimal  A\si{\degree} and B\si{\degree} might not be possible. Therefore an optimal frequency range has to be chosen for the angle search. The refraction in the low frequency is nearly zero for the distance present at the concert, where frequency above \SI{400}{\hertz} is effected at distance above \SI{110}{\meter}.  As found in the questioner in \autoref{ap:ques}, the maximum distances before delay tower is approximate \SI{60}{\meter}, and furthermore, the wind might be stronger than \SI{5}{\meter\per\second} to a concert but wind above \SI{10}{\meter\per\second} to \SI{15}{\meter\per\second} stops the concert. Based on the latter fact, the refraction effect below \Hz{1000} in measurement \autoref{sec:ana:atm_ref} and the control possibility of the line source array \autoref{fig:td:kud_cov}, the lower frequency control criteria is \Hz{1000}. The frequency above \Hz{1000} is highly affected by refraction and the shadow distances get lower as the frequency rises. 
The frequency area above \Hz{1000} which is the most important is the frequency area where the highest intelligibility frequency spectra are present. Therefore the optimal frequency range for adaptive main lobe adjustment is chosen to be from \Hz{1000} to \Hz{8000}. The frequency above is not as crucial for conveying information to the audience. Furthermore, the distance between the line source element to the audience raises with respect to the hight of the line source element. Therefore, the signal to every speaker has to be controlled individually with a different angle and frequency response. 

The above proposal solution deals with the crosswind problem. When the wind change direction to be parallel to the frontal direction of the line source array, the changing in horizontal angle does not solve the problem. At this state, the solution is based on the vertical direction of the main lobe of the line source array. The shadow zone distance is depending on the hight of the line source array from the ground. As higher the line source array is, as higher the distance is before the shadow zone is present while the vertical angle is optimised to the audience area. This is one of the reasons that the delay tower is located as long as \SI{73}{\meter} from the main line source array at Roskilde festival. The proposed solution is then to adjust the vertical angle of the line source array, such that the upper speaker ether point more downwards or upwards for upwards refraction or downwards refraction respectively. 
Therefore if the upper speaker points more downwards the energy from the speaker might arrive at the ground where else if the line source element is pointing parallel to the ground, the energy never enters the ground surface. The following \autoref{fig:td:sha_zon_opt} shows the proposal solution to parallel wind refraction.  

\xfig{design/shadow_zone_optimising.pdf_t}{The figure shows the proposal solution to the upwards refraction, where the hole line source array is angled more downwards}{fig:td:sha_zon_opt}{1} 

The optimal solution, therefore, might be a solution which controls the refraction in the frequency range from \Hz{1000} to \Hz{8000} with live horizontal and vertical adjustment of the line source main lobe. The next section explains the speaker which is used for the solution and how the main lobe is defined and adjusted.


\section{Description of the used line source array}\label{sec:prop:des_of_lin}

This section aims to explain the functionality for the line source array which is used to test the proposed solution. The description starts with a short introduction to the line source element. Then the frequency response of the single element, the horizontal directionally control, and the vertical directionally control is explained.

The line source elements which is used to test the proposed solution is an L-Acoustics KUDO line source array. This line source element is a legacy long throw variable curvature speaker. The speaker is designed as the second option in the K series which today is renewed and renamed to L-acoustics K2. The speaker can be flown as a vertical line with a maximum of 21 elements. The limit is due to the safety limit on the flying tools. One single element have a frequency response from \Hz{50} to \SI{18}{\kilo\hertz} with a maximum deviation of $\pm$ \dB{3} and have a maximum \gls{spl} of \dB{140} at \SI{1}{\meter}. The following \autoref{fig:td:kud_freq_res} shows the average frequency response over \SI{40}{\degree} horizontal angle of a single line source array.

\fig{kudo_frequency_response.pdf}{The graph shows the average frequency response over \SI{40}{\degree} horizontal angle of a single line source array at \SI{1}{\watt} \citep{KUDO_data}.}{fig:td:kud_freq_res}{1}


The line source array coverage can be controlled vertically by the flying tools, where the angle between the speaker is adjusted from \SI{0}{\degree} to \SI{10}{\degree} with \SI{1}{\degree} step size. The horizontal coverage can be controlled individually on every element. The line source element allows both symmetric horizontal coverage and asymmetric coverage. The angle from the frontal direction to the outer main lobe \dB{-6} is ether \SI{25}{\degree} or \SI{55}{\degree}. By this two angle for both side, four coverage angle of the speaker is possible, \SI{110}{\degree}, \SI{50}{\degree} and \SI{80}{\degree} ether to the left or right. The following the \autoref{fig:td:kud_cov} shows both the wide and narrow symmetric main lobe option of the L-acoustics KUDO. The asymmetric coverage can be founded in \citep{KUDO_data}

\fig{kudo_coverage.pdf}{The graph shows the symmetric coverage area of the L-Acoustics KUDO line source array \citep{KUDO_data}.}{fig:td:kud_cov}{1}



The mechanical coverage solution in the L-acoustics KUDO as well as other line source array element is not made for wind problems but for neighbouring determines and higher \gls{spl} in the main lobe of the high frequency. All solution used today is only possible to be changed by hand and is not electrically controlled. The method for changing the horizontal directivity in the L-acoustics KUDO line source element is two plexiglass plate fixed to the front grill. The fixing mechanism can be adjusted sidewise by realising two splits on both plexiglass plates. The plate can then be slid along the grill to change the mouth of the speaker output. The following \autoref{fig:td:kud_dir} illustrate the principle.

\fig{kudo_directivity.pdf}{The figure shows how the horizontal directivity is controlled on a L-Acoustics KUDO line source array element \citep{KUDO_manual}.}{fig:td:kud_dir}{0.5}

To be able to control the vertical main lobe of the line source array, the mechanical solution is the angle between the element. This means that the vertical coverage control cannot be controlled on the individual line source element as the horizontal coverage. To be able to control the vertical coverage, the speaker is trapeze designed such that the high frequency horn throat stays together while the angle between the elements is adjusted in the back of the element for every line source element. The following \autoref{fig:td:kud_dir_ver} shows how the line source element is attached and how they are angled in vertically.

\fig{vertical_coverage_adjust.pdf}{The figure shows how the vertical directivity is controlled on a L-Acoustics KUDO line source array element \citep{KUDO_rig}.}{fig:td:kud_dir_ver}{0.5}

To be able to fix the vertical coverage on the L-acoustics KUDO, the upper left rigging pin shall only be placed into the speaker rig when the angle shown on the metal pease shows the desired vertical coverage between two line source element.  

\section{Test setup}\label{sec:pro:test_setup}
This section aims to design the speaker setup such that the proposed solution can be tested on the limited amount of available line source array. The author only owns six L-acoustics KUDO line source element and four belonging low frequency driver. Therefore the test is split into two tests, one test for crosswind and one test for parallel wind. The reason that the test is separated into two parts is that the amount of line source array is too small to fit both tests of parallel wind and crosswind wind. The next section starts explaining the setup for the latter.

For crosswind the idea is to fly two stacks of line source array beside each other, one stack orthogonal to the wind and one stack rotated some amount of degree agents the wind. The transfer function for both stacks is measured in the coverage area of the frontal pointing line source array. The rotated array is rotated with few degrees for every measurement until the line source array is orthogonal to the frontal line source array. The transfer function which shows the lowest deviation between microphone position is the angle of the solution. The following \autoref{fig:td:speaker_rot} illustrate the line source array set up as a top view. The microphone is situated as far away as that the displacement between the line source array is assumed to be negligible.

\xfig{design/speaker_rotation.pdf_t}{The figure shows the line source setup for the measurement. Every line source consist of three KUDO line source element attached with \SI{0}{\degree} vertical coverage angle}{fig:td:speaker_rot}{1} 

The \autoref{fig:td:speaker_rot} illustrate that the wind makes upwards and downwards refraction and therefore the main lobe is curved. For parallel wind, the idea is to fly one large stack with six L-acoustics KUDO line source array with horizontal symmetric coverage. The array is tilted some degree until the optimal angle is measured. The optimal angle is the angle where the shadow zone is pushed as far away as possible concerning the wind speed. The following \autoref{fig:td:vertical_coverage} illustrate the line source array from the side.

   
\xfig{design/vertical_coverage.pdf_t}{The figure shows the line source setup for the measurement. The line source array consist of six KUDO line source element attached with \SI{0}{\degree} vertical coverage angle}{fig:td:vertical_coverage}{1} 

The \autoref{fig:td:vertical_coverage} illustrate that the wind makes upwards refraction of sound and therefore the main lobe is curved upwards.


\section{Optimality criteria}
This section aims to describe the optimal \gls{spl} coverage. It is founded in \autoref{sec:prop:des_of_lin} that the \dB{-6} describes the angle of the line source element main lobe limit. This means that the frontal direction of the line source array is \dB{6} higher than the outer coverage area in the illustration in \autoref{fig:td:speaker_rot}. Furthermore, the \gls{spl} coverage decay with distance and with respect to the viscosity and frequency dependent refraction. This fact makes the optimality criterion for crosswind different from the parallel wind solution. This section starts describing the optimality criteria for the crosswind and then the optimality criteria for parallel wind. One thing that the solution have common for both wind situation is that the refraction is frequency dependent and therefore the frequency range that shall be optimised for is the same. It is founded in (ADD SOMETHING ABOUT INTELLIGIBILITY) that the most important frequency range for information dissemination is the frequency range from \Hz{500} to \Hz{8000}. Therefore the optimality criteria frequency range for both solution is weighted as the ineligibility frequency range. 


\paragraph{Crosswind} The optimality criteria for crosswind is that the \gls{spl} coverage differences within the main lobe coverage after frequency weighting is minimised and symmetric around the frontal direction. Furthermore, the \gls{spl} differences between the frontal direction and the main lobe limit is of a maximum of \dB{6}

\xfig{design/main_lobe_correction.pdf_t}{The figure shows the}{fig:td:mail_lobe_cor}{1}


The center in the measurement was not measured doing the measurement so the stated value is a prediction based on \citep{review_of_sound} which indicate that the energy addition at short distances because of downwards refraction is small compare to the energy loss with upeards refraction. 

The right \si{\decibel} in the expected result is a gues. This value depend on the sidewise attenuation outside the main lobe of the speaker which is unknown since the produce does not have any given measurement.




\paragraph{Parallel wind} The optimality criteria for the parallel wind is that the shadow zone is moved thunder back. Therefore the shadow zone is founded and the microphone is placed in the shadow zone. The optimality criteria are then the angle where the \gls{spl} in the shadow zone is highest in the weighted frequency range. 
