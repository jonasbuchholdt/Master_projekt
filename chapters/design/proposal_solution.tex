\section{proposal of solution to the wind problem}\label{sec:td:pro_sol_pro}

This section aims to propose a solution to the problem founded in the crosswind measurement \autoref{sec:ana:atm_ref} and the problem statement in \autoref{ch:statement}. 
The solution is a more homogeneous \gls{spl} coverage in the coverage area of the speaker without wind. In other words, the line source array has a frontal horizontal directional angle defined as the \dB{-6} limit of the main pressure lobe. The line source array main lobe is given in the horizontal degree as an addition of the main lobe from the frontal direction to the side and can both be symmetric and asymmetric, depending on the line source array element. The following \autoref{fig:td:mail_lobe} shows an illustration of the main lobe. 

\xfig{design/main_lobe.pdf_t}{The figure shows the}{fig:td:mail_lobe}{1} 

The proposed solution is to be able to steer the main lobe horizontal direction of the line source array adaptive by measuring tools and electronical actuators. 
As being able to steer the main lobe horizontal direction of the line source array, the main lobe can be steered more in the direction of the coverage area where the wind attenuate the \gls{spl} by upwards refraction.  The crosswind problem is not as drastically close to the speaker, so the line source array which shall be able to be controlled is the line source array which covers the audience in back. The solution is therefore based on an adaptive main lobe, which is controlled such that the power of the main lobe is pointing against the audience with lowest \gls{spl}. The following \autoref{fig:td:geo_fin_sol} shows a graphical illustration of the proposed solution to archive a more homogeneous \gls{spl} coverage area in the frontal direction of the speaker with the wind.

\xfig{design/geo_find_solution.pdf_t}{The figure shows the wanted direction of the sound coverage area after the effect of crosswind. C is the speed of wind in cross direction of the frontal direction of the speaker. A and B is the main lobe angle that needs to be founded. On the figure the angle are equal but that might not be true}{fig:td:geo_fin_sol}{1} 

The gold is then to search A\si{\degree} and B\si{\degree} based on wind speed C \si{\meter\per\second} as shown in \autoref{fig:td:geo_fin_sol} such that the \gls{spl} coverage differences is minimised. The angle of A and B in the figure is equal, and this might not be true for the solution. As founded in \autoref{sec:ana:atm_ref} the refraction is frequency dependent, therefore, finding the optimal  A\si{\degree} and B\si{\degree} might not be possible. Therefore an optimal frequency range has to be chosen for the angle search. The refraction in the low frequency is nearly zero for the distance present at the concert, where frequency above \SI{400}{\hertz} is effected at distance above \SI{110}{\meter}.  As found in the questioner in \autoref{ap:ques}, the maximum distances before delay tower is approximate \SI{60}{\meter}, and furthermore, the wind might be stronger than \SI{5}{\meter\per\second} to a concert but wind above \SI{10}{\meter\per\second} to \SI{15}{\meter\per\second} stops the concert. Based on the latter fact, the refraction effect below \Hz{1000} in measurement \autoref{sec:ana:atm_ref} and the control possibility of the line source array \autoref{fig:td:kud_cov}, the lower frequency control criteria is \Hz{1000}. The frequency above \Hz{1000} is highly affected by refraction and the shadow distances get lower as the frequency rises. 
The frequency area above \Hz{1000} which is the most important is the frequency area where the highest intelligibility frequency spectra are present. Therefore the optimal frequency range for adaptive main lobe adjustment is chosen to be from \Hz{1000} to \Hz{8000}. The frequency above is not as crucial for conveying information to the audience. Furthermore, the distance between the line source element to the audience raises with respect to the hight of the line source element. Therefore, the signal to every speaker has to be controlled individually with a different angle and frequency response. The above proposal solution deals with the crosswind problem. When the wind direction change such that the wind comes from the back audiences to the stage, or in other words, is parallel with the frontal direction of the speaker, the adaptive horizontal directional angle is not the only parameter which have to be adjusted to solve the problem. At this state, the proposal solution is based on changcing the vertical angle of the main lobe of the line source array. The shadow zone distance is depending on the hight of the line source array from the ground within the limited hight of flying points on the stage. As higher the line source array is flown, as higher the distance is before the shadow zone is present while the vertical angle is optimised to the audience area. This is one of the reasons that the delay tower is located as long as \SI{73}{\meter} from the main line source array at Roskilde festival. The proposed solution is then to adjust the vertical angle of the line source array, such that the upper speaker ether point more downwards or upwards for upwards refraction or downwards refraction respectively. 
Therefore if the upper speaker points more downwards the energy from the speaker might arrive at the ground where else if the line source element is pointing parallel to the ground, the energy never enters the ground surface. The following \autoref{fig:td:sha_zon_opt} shows the proposal solution to parallel wind refraction.  

\xfig{design/shadow_zone_optimising.pdf_t}{The figure shows the proposal solution to the upwards refraction. The upper part of the figure shows the lower vertical main lobe ray while the array is orthogonal to the ground where the lower part of the figure shows the lower vertical main lobe ray while the line array is angled more downwards}{fig:td:sha_zon_opt}{1} 

The proposal solution to both problem is therefore an adaptive control of both the horizontal and vertical angle of the line source array. The next section explains the speaker which is used for the solution and how the main lobe is defined and adjusted.



\subsection{The optimal solution}
This section starts describing the optimality criteria for the crosswind and then the optimality criteria for parallel wind. One thing that the solution have common for both wind situation is that the refraction is frequency dependent and therefore the frequency range that shall be optimised for is the same. It is founded in (ADD SOMETHING ABOUT INTELLIGIBILITY) that the most important frequency range for information dissemination is the frequency range from \Hz{500} to \Hz{8000}. Therefore the optimality criteria frequency range for both solution is weighted as the ineligibility frequency range. 

\paragraph{Crosswind} The optimality criteria for crosswind is that the \gls{spl} coverage differences within the main lobe coverage after frequency weighting is minimised and symmetric around the frontal direction. Furthermore, the \gls{spl} differences between the frontal direction and the main lobe limit is of a maximum of \dB{6}

\paragraph{Parallel wind} The optimality criteria for the parallel wind is that the shadow zone is moved thunder back. Therefore the shadow zone is founded and the microphone is placed in the shadow zone. The optimality criteria are then the angle where the \gls{spl} in the shadow zone is highest in the weighted frequency range. 
