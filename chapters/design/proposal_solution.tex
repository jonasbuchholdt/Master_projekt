\section{proposal of solution to the cross wind problem}\label{sec:td:pro_sol_pro}

The aim of this section is to propose a solution to the problem founded in the cross wind measurement in \autoref{sec:ana:atm_ref}. The solution is based on the problem statement in \autoref{ch:statement}. The solution is a more homogeneous \gls{spl} coverage in the coverage area of the speaker without wind. In other word, the line source array has a frontal horizontal directional angle defined as the \dB{-6} limit of main pressure lobe. A line source array main lobe is given in the horizontal degree as an addition of the main lobe from the frontal direction to the side  and can both be symmetric and asymmetric, depending on the line source array element.

The proposal solution is to be able to steer the main lobe horizontal direction of the line source array electronical. As beeing able to steer the main lobe horizontal direction of the line source array, the main lobe can be steered more up agents the direction of the coverage area where the wind attenuate.  The crosswind problem is not as drastical close to the speaker, so the line source array which shall be able to be controlled is the coverage area is the element which cover the audience in back. The solution is based on a changeable main lobe which is as narrow as possible to archive as high \gls{spl} and the audience and as low neighbog desterbines as possible. The following \autoref{fig:td:geo_fin_sol} shows a graphical illustrate of the proposal solution to archive a more homogeneous \gls{spl} coverage area in the frontal direction of the speaker without wind.

\xfig{design/geo_find_solution.pdf_t}{The figure shows the wanted direction of the sound coverage area after the effect of crosswind. C is the speed of wind in cross direction of the frontal direction of the speaker. A and B is the angle that needs to be founded. On the figure the angle are equal but that might not be true}{fig:td:geo_fin_sol}{1} 

The gold is then to search A\si{\degree} and B\si{\degree} based on wind speed C \si{\meter\per\second} as shown in \autoref{fig:td:geo_fin_sol} such that the \gls{spl} coverage differences is minimized. The angle of A and B in the figure is equal, this might not be true in for the solution.

A filter is needed since the effect is frequency dependent.


\section{Description of the used line source array}

 The speaker which will be used to design the solution is a L-Acoustics KUDO line source array where the main lobe coverage can be controlled mechanical. It is both possible to make the main lobe symmetric and asymmetric on this line source element. The following the \autoref{fig:td:kud_cov} shows both the wide and narrow symmetric main lobe option of the KUDO. The asymmetric coverage can be founded in \citep{KUDO_data}

\fig{kudo_coverage.pdf}{The graph shows the symmetric coverage area of the L-Acoustics KUDO line source array \citep{KUDO_data}.}{fig:td:kud_cov}{0.9}

The mechanical coverage solution in the KUDO as well as other line source array element is not made for wind problems but for neighbog desterbines and higher \gls{spl} in the main lobe of the high frequency. All solution used today is only posible to change by hand and is not electrical controlled. 

\fig{kudo_directivity.pdf}{The figure shows how the directivity is controlled on a L-Acoustics KUDO line source array element \citep{KUDO_manual}.}{fig:td:kud_dir}{1}




\section{Designing the measurement}

The aim of this section is to design a test on a non modified line source array to test the proposal solution from \autoref{sec:td:pro_sol_pro}. To be able to test the proposal solution, a measurement system have to be designed. To be sure that the wind noise do not effect the measurement the part of the design takes care about finding the perferable microphone wind screen configuration, based on the aviable STOF in acoustics lab. The second part measure the frequency dependent attenuation of the chosen windscreen compare to the microphone response without windscreen. The comparisen is done to cont for the attenuation of the windscreen such that the measured data reflect the \gls{spl} at the point as good as possible. The third part takes care about designing the nesesarly data logging function. The last part designs the measuring signal playback and record method.


\subsection{Design of windscreen}
The aim of this section is to be able find a windscreen to the measuring microphone such that the wind noise is low compare to the measuring signal. There is to aspect in this windscreen configuration, firs the the wind noise cannot be filtered electronical between the microphone and the preamp.Therefore the strength of the wind noise shall be as low as the preamp do not overload. Secondly the measurement shall be measurement of the signal and not the wind noise, therefore the signal shall be sufficiant higher that the wind noise such that the measurement is trustable and represent the \gls{spl} procused of the speaker at the measuring point. 

The idea of an additional wind screen is use the original windscreen to the microphone and then try to stop the wind in just at the microphone with a blocking surface. The surface shall therefore be able to lower the windspeed at the microphone and have as less reflection as possible. The original windscreen is cept on the microphone in the wind stop area to attenuate the wind noise that passes the blockaga and attenuate the turbulence prodused by the wind stopper. 

There is to wind stopping concept that will be tested, a plan surface of rockwool as shown in \autoref{} and a foam wedge solution as shown in \autoref{}
    

There is made a preliminary test test to measure the effect in a fast measuring setup before a field measurement. The measurement was done in acoustics lab with two fan and a wind speed of \SI{2.5}{\meter\per\second}. The result is founded in \autoref{}

\plot{plot/with_ball_wind_compare}{The graph shows one of the time measurement with configuration three}{fig:sec:pop:tim_wit_bal_roc_kil}


As seen in the measurement, the original wind screen does have a huge wind noise attenuation but as shown in ... measurement, the wind noise can be further attenuated by the developed wind stopping concept. 




\subsection{Attenuation of the windscreen} 

\subsection{Data logging system} 

To be able to use the information of the wind spees temparature and humidity, the data logging of the atmospherical condition have to be syncronius with the measurement. 




The speaker is chosen to be adjusted to the narrow main lobe because it is assumed that the distance from the audience to the speaker is so large that the wide angle goes beyond the audience area.

Because of limitation, the speaker is flown in a hight of \SI{6}{\meter}. 

The humidity and temperature have to be measured.

To measure the \gls{spl} coverage of the speaker a flat area with mown grass is chosen to be used. The optimal area area without any building or trees might not be posible, therefore blockage or sound reflaction surface other than the ground is only allowed to be present in the double of distance compare to the distance from the speaker to the microphone. Based on the refraction effect versus distance founded in \autoref{sec:ana:atm_ref}. The distance from the speaker to the microphone array is chosen to be \SI{50}{\meter}. The distance is based on the experience of the author described in \autoref{sec:ana:aut_exp_con} and the founded refraction effect in \autoref{sec:ana:atm_ref}. It was founded that the refraction effect should be minimal at a distance of \SI{50}{\meter} when the speed of wind is \SI{5}{\meter\per\second}. 


To keep the wind speed realistic for measurement and for concert, but still having wind pressent, the wind speed doing the measurement is limited in the range for average \SI{5}{\meter\per\second} to \SI{10}{\meter\per\second}. Less average wind speed than \SI{5}{\meter\per\second} is avoided to ensure measureble effect of the wind on sound propagation. The higher limit of the \SI{10}{\meter\per\second} is chosen to ensure that the speaker tower is safe at the hight at \SI{6}{\meter}. The limited size of the setup makes the setup wind (følsom) because it is not puttet up as a cube but only as a surface. 


\xfig{design/angle_measurement.pdf_t}{The figure shows the measurement setup}{fig:td:ang_mea}{1} 


where the refraction at \SI{110}{\meter} already starts at \Hz{400}. 


The area is without  The resend to use this  

\plot{plot/angle_test}{The graph shows the first transfer function measurement within the high frequency directional angle. The $L_{Aeq,5}$ \gls{spl} different between the microphones is \dB{6.77} (IR_3) The graph is normed to contain the same $L_{Aeq,5}$ \gls{spl} }{fig:ana:ang_tes}



\section{Measuring program}


\section{Result}







