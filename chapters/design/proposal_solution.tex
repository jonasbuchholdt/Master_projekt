\section{Proposal of solution to the wind problem}\label{sec:td:pro_sol_pro}

This section aims to propose a solution to the problem founded in the crosswind measurement \autoref{sec:ana:atm_ref} and the problem statement in \autoref{ch:statement}. 

\begin{enumerate}
\item To be able to find a solution to the problem, the optimal condition is defined in \autoref{sec:des:opt_con}.
\item A proposed solution to the crosswind is defined in \autoref{sec:des:pro_cross}.
\item A proposed solution to the parallel wind is defined in \autoref{sec:des:pro_para}.
\end{enumerate}





\section{Optimality condition}\label{sec:des:opt_con}

To be able to search for a solution and design a test to research if the proposed solution has the optimal effect on the coverage area, the optimal condition is defined in this section. The optimal condition is as simple as the \gls{spl} coverage in the coverage area of the line source array without wind. In other words, the line source array has a frontal horizontal directional angle defined as the \dBr{-6} limit of the main pressure lobe. The line source array main lobe is given in the horizontal degree as an addition of the main lobe from the frontal direction to both side and can both be symmetric and asymmetric, depending on the line source array element. The following \autoref{fig:td:mail_lobe} shows an illustration of the main lobe. 

\xfig{design/main_lobe.pdf_t}{The figure shows the pressure limit which defined the main lobe of the line source array and the coverage area without wind.}{fig:td:mail_lobe}{1} 

As illustrated in \autoref{fig:td:mail_lobe}, the coverage area is a parabolic surface which is limited as the \dBr{-6} coverage limit of the line source array. The illustration illustrates the coverage area without the presence of refraction. The optimality condition is the shown \gls{spl} distribution in the coverage area, as shown in \autoref{fig:td:mail_lobe}, except the difference between the centre and the main lobe limit, is allowed to be lower. The solution to the crosswind is, therefore, a way to be able to adjust the coverage area such that the line source array can eliminate the effect of the wind and cover the area as without wind. To be able to eliminate the wind effect at the audience area, the audience area has to be defined. To defined the audience area, a questioner is made among the large sound rental company in Denmark. The questioner is founded in \autoref{ap:ques}. The gold of the questioner is to find the highest coverage distance from the line source array to the back audience. The founded maximum distances before delay tower are approximate \SI{50}{\meter} for the most companies, and furthermore, the wind speed above \SI{15}{\meter\per\second} might cancel the concert. Therefore the defined coverage area as shown in \autoref{fig:td:mail_lobe} is \SI{50}{\meter} from the stage front and the wind condition is under \SI{15}{\meter\per\second}.



\section{Proposal solution to crosswind}\label{sec:des:pro_cross}
The crosswind problem is shown in \autoref{sec:ana:atm_ref} to highly change the coverage area. Agents the wind, the upwards refraction is shown to attenuate the sound more than \dBr{6} A-weighted at a distance of only \SI{25}{\meter} and an average wind speed of \SI{14}{\meter\per\second}. Furthermore, it is founded that the shadow zone \gls{spl} depends on the \gls{spl} in the sound path, because the wind eddies, eddies the sound energy into the shadow zone. It is then researched if adding more power into the upwards refraction direction also adds more power into the shadow zone by the wind eddies. The following \autoref{fig:td:eddies} illustrate the eddies theory in upwards refraction.

\xfig{design/eddies.pdf_t}{The figure shows the sound path above the shadow zone and inside the shadow zone produced by the eddies.}{fig:td:eddies}{1} 

The proposed solution is then to steer more power into the direction of upwards refraction and less power into the front and in the direction of downwards refraction and then optimise for the optimality condition defined in \autoref{sec:des:opt_con}. The following \autoref{fig:td:geo_fin_sol} shows a graphical illustration of the proposed solution to archive a more homogeneous \gls{spl} in the coverage area of the line source array with the presence of wind.

\xfig{design/geo_find_solution.pdf_t}{The figure shows the directionality of the line source array and the optimised directionality of the line source array after the effect of crosswind. C is the speed of wind in the cross direction. A and B is the main lobe angle change, which needs to be founded. On the figure the angle are equal, but that might not be true.}{fig:td:geo_fin_sol}{1} 

The gold is then to search A\si{\degree} and B\si{\degree} based on wind speed C \si{\meter\per\second} and the optimal coverage area as shown in \autoref{fig:td:geo_fin_sol}. As founded in \autoref{sec:ana:atm_ref} the refraction is frequency dependent, therefore, finding the optimal  A\si{\degree} and B\si{\degree} might not be possible for all frequency. Furthermore the refraction in the low frequency is nearly zero for the distance present at the concert.  


\section{Proposal solution to parallel wind}\label{sec:des:pro_para}
The above proposal solution deals with the crosswind problem. When the wind direction change such that the wind comes from the back audiences to the stage, or in other words, is parallel with the frontal direction of the line source array, another solution is researched than using the eddies theory. The resand to search for another solution is that using the eddies theory in parallel wind require that the power from the line source array is raised. The proposed solution is then to move the shadow zone instead of raising the power in the shadow. To be able to move the shadow zone, the idea is to change the vertical tilt angle of the main lobe, such that the upper line source element ether point more downwards or upwards for upwards refraction or downwards refraction respectively. In the case of upwards refraction, if the upper line source element points more downwards the energy from the line source array might arrive at the ground further back, where else if the line source array is pointing parallel to the ground, the energy might never enter the ground surface. The vertical tilt angle for upwards refraction of the line source array is defined as forwards tilting in the rest of the thesis. The following \autoref{fig:td:sha_zon_opt} shows the proposal solution to parallel wind refraction.  

\xfig{design/vertical_coverage.pdf_t}{The figure shows the proposal solution to the upwards refraction. The non-tilted line source array figure shows the lower vertical main lobe ray while the array is orthogonal to the ground where the tilted line source array shows the lower vertical main lobe ray while the line array is forward tilted.}{fig:td:sha_zon_opt}{1} 


The shadow zone distance might depend on the hight of the line source array from the ground within the limited hight of flying points on the stage. As higher the line source array is flown, as higher the distance might be before the shadow zone is present while the forward tilting is optimised to the audience area. 

  
  
  