\section{proposal of solution to the cross wind problem}\label{sec:td:pro_sol_pro}

The aim of this section is to propose a solution to the problem founded in the cross wind measurement in \autoref{sec:ana:atm_ref}. The solution is based on the problem statement in \autoref{ch:statement}. The solution is a more homogeneous \gls{spl} coverage in the coverage area of the speaker without wind. In other word, the line source array has a frontal horizontal directional angle defined as the \dB{-6} limit of main pressure lobe. A line source array main lobe is given in the horizontal degree as an addition of the main lobe from the frontal direction to the side  and can both be symmetric and asymmetric, depending on the line source array element.

The proposal solution is to be able to steer the main lobe horizontal direction of the line source array electronical. As beeing able to steer the main lobe horizontal direction of the line source array, the main lobe can be steered more up agents the direction of the coverage area where the wind attenuate the \gls{spl} by upwards refraction.  The crosswind problem is not as drastical close to the speaker, so the line source array which shall be able to be controlled is the line source array which cover the audience in back. The solution is based on a changeable main lobe which can be as narrow as possible to archive as high \gls{spl} at the audience and low in the neighbourhood. The following \autoref{fig:td:geo_fin_sol} shows a graphical illustrate of the proposal solution to archive a more homogeneous \gls{spl} coverage area in the frontal direction of the speaker with wind.

\xfig{design/geo_find_solution.pdf_t}{The figure shows the wanted direction of the sound coverage area after the effect of crosswind. C is the speed of wind in cross direction of the frontal direction of the speaker. A and B is the main lobe angle that needs to be founded. On the figure the angle are equal but that might not be true}{fig:td:geo_fin_sol}{1} 

The gold is then to search A\si{\degree} and B\si{\degree} based on wind speed C \si{\meter\per\second} as shown in \autoref{fig:td:geo_fin_sol} such that the \gls{spl} coverage differences is minimized. The angle of A and B in the figure is equal, this might not be true for the solution. As founded in ... the refraction is frequency dependent. Therefore to find the optimal  A\si{\degree} and B\si{\degree} might not be possible. Therefore an optimal frequency range have to be chosen for the angle search. The refraction in the low frequency was nearly zero for the distance of concert, where frequency above \SI{400}{\hertz} is effected at at distance above \SI{110}{\meter}. The concert coverage is often less or upto \SI{110}{\meter} but the wind might be stronger than \SI{5}{\meter\per\second}, but as shown in ... the refraction effect is low under \Hz{1000} with wind speed of \SI{13}{\meter\per\second} and a distance of \SI{25}{\meter}. The solution lower frequency creteria is therefore chosen to be \Hz{1000}. The frequency above \Hz{1000} is the highly effected frequency range and the shadow distances gets lower as the frequency rises. The the frequency area above \Hz{1000} which is the most important is the frequency area where the highest intelligibility frequency spectra is present. Therefore the optimal frequency range for main lobe adjustment is from \Hz{1000} to \Hz{8000}. The frequency above is not important for convey information to the audience. Furthermore the distance between the speaker and the line source element raises with raises hight therefore the signal to every speaker have to be controlled individual with different angle and frequency response. 

The above proposal solution deals with the crosswind problem, when the wind change direction to be parallel to the frontal direction of the speaker the changing in angle will not solve the problem any more. At this state the solution shall be based on the vertical direction of the main lobe of the line source array. The proposal solution here is to adjust the vertical angle of the hole speaker array, such that the upper speaker ether point more downwards or upwards for upwards refraction or downwards refraction respectivly. The shadow zone distance is depending on the hight of the line source array, the higher the source is from the ground the thouther away the shadow zone will be. Therefore if the upper speaker points more downwards the energy from the speaker might arrive to the ground thuder away douing upwards refraction compare to the source element in the middle of the line source array. The following \autoref{fig:td:sha_zon_opt} shows the proposal solution to parallel wind refraction.  

\xfig{design/shadow_zone_optimising.pdf_t}{The figure shows the proposal solution to the upwards refraction, where the hole line source array is angled more downwards}{fig:td:sha_zon_opt}{1} 

The optimal solution therefore might be a solution which controls the refraction in the frequency range from \Hz{1000} to \Hz{8000} with live horizontal and vertical adjustment of the speaker main lobe. The next section explains the speaker which will be used for the solution and how the main lobe is defined and adjusted.


\section{Description of the used line source array}\label{sec:prop:des_of_lin}

 The speaker which will be used to design the solution is a L-Acoustics KUDO line source array where the main lobe coverage can be controlled mechanical. It is both possible to make the main lobe symmetric and asymmetric on this line source element. The following the \autoref{fig:td:kud_cov} shows both the wide and narrow symmetric main lobe option of the KUDO. The asymmetric coverage can be founded in \citep{KUDO_data}

\fig{kudo_coverage.pdf}{The graph shows the symmetric coverage area of the L-Acoustics KUDO line source array \citep{KUDO_data}.}{fig:td:kud_cov}{0.9}

The mechanical coverage solution in the KUDO as well as other line source array element is not made for wind problems but for neighbog desterbines and higher \gls{spl} in the main lobe of the high frequency. All solution used today is only posible to change by hand and is not electrical controlled. 

\fig{kudo_directivity.pdf}{The figure shows how the directivity is controlled on a L-Acoustics KUDO line source array element \citep{KUDO_manual}.}{fig:td:kud_dir}{1}









