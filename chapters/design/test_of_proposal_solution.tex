\section{Test of proposal solution}
This section aims to design a test setup for testing the proposed solution. The test setup will be based on an L-acoustics KUDO line source array which is described in \autoref{sec:prop:des_of_lin} without any modification. This chapter designs the measuring method and the needed windscreen for wind measurement. To be able to design the measuring, the used line source array is briefly described in \autoref{sec:prop:des_of_lin}, then the measuring setup is designed in \autoref{sec:pro:test_setup} and in the end, the measuring method is designed in \autoref{sec:des:des_mes}.



\section{Description of the used line source array}\label{sec:prop:des_of_lin}

This section aims to explain the functionality for the line source array which is used to test the proposed solution. The description starts with a short introduction to the line source element, then the frequency response of the single element, the horizontal directionally control, and the vertical directionally control is explained.

The line source elements which is used to test the proposed solution is an L-Acoustics KUDO line source array. This line source array is a legacy long throw variable curvature speaker. The speaker is designed as the second option in the K series which today is renewed and renamed to L-acoustics K2. The speaker can be flown as a vertical line with a maximum of 21 elements. The maximum number of an element is due to the safety limit on the flying tools. 
One single element have a frequency response from \Hz{50} to \SI{18}{\kilo\hertz} with a maximum deviation of $\pm$ \dB{3} and have a maximum \gls{spl} of \dB{140} at \SI{1}{\meter}. The following \autoref{fig:td:kud_freq_res} shows the average frequency response over \SI{40}{\degree} horizontal angle of a single line source element.

\fig{kudo_frequency_response.pdf}{The graph shows the average frequency response over \SI{40}{\degree} horizontal angle of a single line source array at \SI{1}{\watt} \citep{KUDO_data}.}{fig:td:kud_freq_res}{1}


The horizontal coverage angle can be controlled individually on every line source element. The line source element allows both symmetric horizontal coverage and asymmetric coverage. The angle from the frontal direction to the outer main lobe \dB{-6} is either \SI{25}{\degree} or \SI{55}{\degree}. By this two angle for both sides, four coverage angle of the speaker is possible, \SI{110}{\degree}, \SI{50}{\degree} and \SI{80}{\degree} ether to the left or the right. The following the \autoref{fig:td:kud_cov} shows both the wide and narrow symmetric main lobe option of the L-acoustics KUDO. The asymmetric coverage can be founded in \citep{KUDO_data}

\fig{kudo_coverage.pdf}{The graph shows the symmetric coverage area of the L-Acoustics KUDO line source array \citep{KUDO_data}.}{fig:td:kud_cov}{1}



The mechanical coverage solution in the L-acoustics KUDO as well as other line source array element is not made for wind problems but for neighbouring disruptions and higher \gls{spl} in the main lobe of the high frequency. All solution used today is only possible to be changed by hand and is not electrically controlled. The method for changing the horizontal directivity in the L-acoustics KUDO line source element is two plexiglass plate fixed to the front grill. The fixing mechanism can be adjusted sidewise by realising two splits on both plexiglass plates. The plate can then be slid along the grill to change the mouth of the speaker output. The following \autoref{fig:td:kud_dir} illustrate the principle.

\fig{kudo_directivity.pdf}{The figure shows how the horizontal directivity is controlled on a L-Acoustics KUDO line source array element \citep{KUDO_manual}.}{fig:td:kud_dir}{0.5}



The line source array vertically coverage area can also be controlled from \SI{0}{\degree} to \SI{10}{\degree} with \SI{1}{\degree} step size. To be able to control the vertical main lobe of the line source array, the mechanical solution is the angle between the line source element. This means that the vertical coverage control cannot be controlled on the individual line source element as the horizontal coverage. To be able to control the vertical coverage, the speaker is trapeze designed such that the high frequency horn throat stays together while the angle between the elements is adjusted in the back of the element for every line source element. The following \autoref{fig:td:kud_dir_ver} shows how the line source element are angled vertically.

\fig{vertical_coverage_adjust.pdf}{The figure shows how the vertical directivity is controlled on a L-Acoustics KUDO line source array element \citep{KUDO_rig}.}{fig:td:kud_dir_ver}{0.5}

To be able to fix the vertical coverage on the L-acoustics KUDO, the upper left rigging pin shall only be placed into the line source element rig when the angle shown on the metal pease shows the desired vertical coverage between two line source element.  


\section{Measuring setup}\label{sec:pro:test_setup}
This section aims to design the speaker setup such that the proposed solution can be tested without mechanical change of the line source array. The test setup, therefore, do not change the speaker directionality adaptive but mechanical by hands before every measurement. The amount of available line source array for the test is limited to six line source element and four belonging low frequency driver. The following \autoref{fig:td:tes_set} shows the measuring setup which is used for both measurement.

\xfig{design/test_setup.pdf_t}{The figure shows test setup for both beasurement}{fig:td:tes_set}{1}

Based on the founded maximum distances from the front of the stage to the delay tower, the line source array is rotated vertically as shown in \autoref{fig:td:tes_set}, such that the two upper line source element covers the measurement area without wind. The following two section \autoref{sub:des:par_set} and \autoref{sub:des:cros_set} describe the line source settings which differs from each other with respect to the rotational procedure. 


\subsection{Crosswind line array settings} \label{sub:des:cros_set}
To test the crosswind proposal solution, the idea is to measure the \gls{spl} coverage while playing in the frontal direction and then measure the \gls{spl} while the line source array is horizontally rotated in the wind direction. To be able to ensure that the horizontal rotation of the line source array keeps the \gls{spl} in the downwards refraction direction as much as possible, this section starts designing the directional angle settings of the line source array. It is founded in \autoref{sec:ana:atm_ref} that downwards refraction raises the \gls{spl} by downwards reflection but the amplification is much less than the attenuation in upwards refraction and is therefore assumed negligible for directionally chose. Therefore, to decide on the horizontal directionally settings the pressure versus angle has to be found for the used line source array. The \dB{-6} directionally is already known from the datasheet, but this detail level is not high enough to decide on any prediction of the attenuation in the downwards refraction direction. Therefore a directionality measurement is made on the used line source array for all possible horizontal directionality settings. The measurement can be founded in \autoref{ap:dir_of_kudo} and the \autoref{fig:ap:KUDO_25_25_des} shows the resulting directionality of the L-acoustics KUDO with the \SI{25}{\degree} / \SI{25}{\degree} settings 

\plot{plot/KUDO_25_25_isobar_design}{The graph shows a contour plot with \dB{3} step of the directionally of the L-acoustics KUDO with \SI{25}{\degree} / \SI{25}{\degree} settings. The lower black contour line indicate the \db directionality for the maximum rotation of the speaker}{fig:ap:KUDO_25_25_des}

The following \autoref{fig:ap:KUDO_25_55_des} shows the resulting directionality of the L-acoustics KUDO with the \SI{25}{\degree} / \SI{55}{\degree} settings 

\plot{plot/KUDO_25_55_isobar_design}{The graph shows a contour plot with \dB{3} step of the directionally of the L-acoustics KUDO with \SI{25}{\degree} / \SI{55}{\degree} settings. The lower black contour line indicate the \db directionality for the maximum rotation of the speaker}{fig:ap:KUDO_25_55_des}

As seen in \autoref{fig:ap:KUDO_25_25_des}, when the speaker is rotated \SI{25}{\degree} as the maximum allowed rotation in the proposal solution the \gls{spl} in the downwards direction is lowered from approximately \dB{-6} to approximately \dB{-18} which is an attenuation of \dB{12}. This \dB{12} attenuation might attenuate the \gls{spl} in the downwards refration too much. As seen in \autoref{fig:ap:KUDO_25_55_des}, when the speaker is rotated \SI{25}{\degree} the attenuation \gls{spl} in the downwards direction is lowered from approximately \dB{-6} to approximately \dB{-12} which is an attenuation of \dB{6}. To decide on a mechanical solution an example is calculated based on the measurement above and the measurement in \autoref{sec:ana:atm_ref}. The example is based on an rotation of \SI{20}{\degree}, where the rotational power from a rotation of the L-acoustics KUDO is added to the measurement. The following explains and shows the example.


\paragraph{Example} The example shows four cases from the line array, one case where the data from the datasheet is used, one case where the measurement in \autoref{sec:ana:atm_ref} is used. Then two examples where the differences in \gls{spl} is calculated from an rotation of \SI{20}{\degree} for \SI{25}{\degree} / \SI{55}{\degree} settings and a rotation of  \SI{10}{\degree} for \SI{25}{\degree} / \SI{25}{\degree} settings and added to the measurement. The following \autoref{fig:td:mail_lobe_cor} shows the example.


\xfig{design/main_lobe_correction.pdf_t}{The figure shows the main lobe coverage area without rotation for the to lower and with maximum rotation for the to upper}{fig:td:mail_lobe_cor}{1}

The centre in the measurement was not measured doing the measurement, so the stated value is a prediction based on \citep{review_of_sound} which indicate that the energy addition at short distances because of downwards refraction is small compared to the energy loss with upwards refraction. 

As seen in \autoref{fig:td:mail_lobe_cor} a rotational of \SI{20}{\degree} calculate a more homugenius \gls{spl} while the line source array is in \SI{25}{\degree} / \SI{55}{\degree} settings. The diviation from the frontal direction is approximatly \dB{3}. In the other case while the rotation is only \SI{10}{\degree} and the settings is \SI{25}{\degree} / \SI{25}{\degree} the \gls{spl} is also approximately evently spredt but the divination to the frontal direction is much higher and is at least \dB{-7.9}. Based on the calculated example, the chosen directionally settings is \SI{25}{\degree} / \SI{25}{\degree} for the measurement. The following \autoref{fig:td:speaker_rot} shows the measurement setup as a top wiev. 

\xfig{design/speaker_rotation.pdf_t}{The figure shows the line source setup for the measurement.}{fig:td:speaker_rot}{1} 

The rotated array is rotated with few degrees for every measurement form \SI{0}{\degree} until \SI{25}{\degree}, which mean that the highest possible \gls{spl} points in the maximum angle. 

The transfer function which shows the lowest \gls{spl} deviation between microphone position is the angle of the solution. 

\subsection{Parallel wind line array settings}\label{sub:des:par_set}

For parallel wind, the idea is to have horizontal symmetric coverage while changing the vertical angle for every measurement. The array is tilted some degree until the optimal angle is measured. The optimal angle is the angle where the shadow zone is pushed as far away as possible concerning the wind speed. The following \autoref{fig:td:vertical_coverage} illustrate the line source array from the side.

   
\xfig{design/vertical_coverage.pdf_t}{The figure shows the line source setup for the measurement. The line source array consist of six KUDO line source element attached with \SI{0}{\degree} vertical coverage angle}{fig:td:vertical_coverage}{1} 

The \autoref{fig:td:vertical_coverage} illustrate that the wind makes refraction of sound upwards and therefore the main lobe is curved upwards.


\section{Designing the measurement}\label{sec:des:des_mes}
This section aims to design a test on the non-modified line source array to test the proposed solution from \autoref{sec:td:pro_sol_pro}. To be able to test the proposed solution, a measurement system has to be designed. To be sure that the wind noise does not affect the measurement the part of the design takes care about finding the preferable microphone windscreen configuration, based on the available STOF in acoustics lab. The second part measure the frequency dependent attenuation of the chosen windscreen compare to the microphone response without a windscreen. The comparison is done to count for the attenuation of the windscreen such that the measured data reflect the \gls{spl} at the point as pleasant as possible. The third part takes care of designing the necessary data logging function. The last part designs the measuring signal playback and record method.

\subsection{Microphone position}
This section aims to design the position of the measuring microphone. The position of the microphone depends on the wind condition, for the crosswind the microphone be placed in the sound field were for the parallel wind, the microphone shall be placed in the shadow zone. The description starts with the former.


The microphone position highly depends on the coverage area of the line source element. Usually, the element which is flown highest cover the as far as possible where the line source element which is closest to the ground cover the frontal audience. Therefore, the distance from the speaker to the microphone has to be found based on the knowledge of coverage distance and the minimum distance before refraction. The distances from the stage to the back audience depends on the size of the concert. For a small concert, the main stage covers the full area where for large concert delay tower helps the coverage. Delay tower is often used for a concert where the distances from the stage to the audience is above \SI{75}{\meter} as Roskilde festival. Roskilde festival situate there delay tower at a distance no longer than \SI{73}{\meter} from the main stage \autoref{ap:ques}. The general founded maximum distances from the main stage to the first delay tower was about \SI{73}{\meter} for a huge concert, \SI{50}{\meter} for a large concert and \SI{30}{\meter} for small concert \autoref{ap:ques}. Base on the knowledge of the maximum distances founded in \autoref{ap:ques} and that Roskilde festival is a special case concerning the size and \SI{50}{\meter} to \SI{60}{\meter} is an often used maximum distance for large concert depending on the hight of the main stage line array. The maximum coverage distances are chosen to be \SI{50}{\meter} for the test since the used line array flying tools is not able to fly the line array as high as the asked companies. The flying height of the top line source array element is about from \SI{12}{\meter} to \SI{16}{\meter} where the flying height of the used test setup is only up to \SI{7}{\meter}.  Furthermore it was shown in \autoref{ta:ana:spl_dif} that refraction occur at a distance of \SI{25}{\meter} with \SI{13}{\meter\per\second}.

Another factor which plays a specific role for the microphone distance from the sound source is the hight of the microphone. There are pros and cons for placing the microphone both at the ground or above the ground. The pros of placing the microphone on the ground are that the ground reflection is eliminated, but the cons are that the shadow zone might be closer to the line source array that above the ground, as it can be seen in \autoref{sec:ana:atm_ref}. Relating it to the concert situation, the ground reflection in the high frequency is assumed to be low where the ground reflection in the low frequency range is assumed to be higher. Moreover, the hight of the audience ear is not at the ground but in a hight of approximately \SI{1.70}{\meter} therefore the most realistic scenario without the audience in the high frequency is at the ear height. Since the critical frequency range is in the high ineligibility frequency range, the low frequency reflection has an only sparse effect on the measuring result. 

Since the measuring distance is chosen to be \SI{50}{\meter} the angel of the speaker is chosen to be the lowest possible angle. This choice is taken because the coverage area at that distance is high at the narrow-angle and is a realistic chose from the company and concert area point of view. The following \autoref{fig:td:ang_mea} shows the microphone position with respect to the speaker position.


\xfig{design/angle_measurement.pdf_t}{The figure shows the measurement setup}{fig:td:ang_mea}{1} 


The microphone position of the vertical refraction measurement depends on the shadow zone position, and it is wanted to measure in the shadow zone to explore if it is possible to move the shadow zone backwards by tilting the line array. Therefore the shadow zone has to be found by measurement before the microphone position can be specified.

\xfig{design/angle_measurement_vertical.pdf_t}{The figure shows the measurement setup}{fig:td:ang_mea_ver}{1} 

\subsection{Design of windscreen}
The aim of this section is to be able find a windscreen to the measuring microphone such that the wind noise is low compare to the measuring signal. There is to aspect in this windscreen configuration, firs the the wind noise cannot be filtered electronical between the microphone and the preamp.Therefore the strength of the wind noise shall be as low as the preamp do not overload. Secondly the measurement shall be measurement of the signal and not the wind noise, therefore the signal shall be sufficiant higher that the wind noise such that the measurement is trustable and represent the \gls{spl} procused of the speaker at the measuring point. 

The idea of an additional wind screen is use the original windscreen to the microphone and then try to stop the wind in just at the microphone with a blocking surface. The surface shall therefore be able to lower the windspeed at the microphone and have as less reflection as possible. The original windscreen is cept on the microphone in the wind stop area to attenuate the wind noise that passes the blockaga and attenuate the turbulence prodused by the wind stopper. 

%There is to wind stopping concept that will be tested, a plan surface of rockwool as shown in \autoref{} and a foam wedge solution as shown in \autoref{}
    
    
The first two windscreen concept is very identical but just with difference size of material. The idea for the first windscreen is to seal the microphone with foam all around except at the frontal direction. The frontal direction include both \SI{180}{\degree} angle in the vertical direction and \SI{90}{\degree} in the horizontal direction. The resan to chose that high degree vertical opening is that no sound is from ground reflection is effected and no sound from upwards is stopped by the foam. The reson to have a narrow horizontal  opening is to be able to get sound inside the opening but still have a wind stopping effect. The following \autoref{fig:td:mes_foa_con} illustrate both windscreen configuration one and windscreen configuration two, just with one size foam wedge.
    
\xfig{design/measure_foam_concept.pdf_t}{The figure shows the foam wedge concept. The concept is is covering over to differend foam wedge, ether two small or two large. The small concept is defined as windscreen configuration one, where the large concept is defined as windscreen configuration two.}{fig:td:mes_foa_con}{1} 

The next concept build on the concept in \autoref{fig:td:mes_foa_con} just with plan surfaces rockwool plates. The opening is also \SI{180}{\degree} angle in the vertical direction and \SI{90}{\degree} in the horizontal direction. The concept is defined as windscreen configuration three. The following \autoref{fig:td:mes_roc_con} illustrate the concept.


\xfig{design/measure_rock_consept.pdf_t}{The figure shows the rockwool concept. This concept is defined as windscreen configuration three.}{fig:td:mes_roc_con}{1} 

The next concept build on minimizing the reflection from the additional windscreen by only placing the microphone close agenst one surface, which cover for the wind noise. The concept is definde as windscreen configuration four. The following \autoref{fig:td:mes_roc_sin_con} illustrate the concept.


\xfig{design/measure_rock_single_consept.pdf_t}{The figure shows the single rockwool concept. This concept is defined as windscreen configuration four.}{fig:td:mes_roc_sin_con}{1}    


A combination of windscreen configuration two and windscreen configuration five is also tested. The combination is defined as windscreen configuration five. The following \autoref{fig:td:mes_roc_foa_con} illustrate the concept.

\xfig{design/measure_rock_foam_concept.pdf_t}{The figure shows the single rockwool concept. This concept is defined as windscreen configuration five.}{fig:td:mes_roc_foa_con}{1}  


Before the optimal windscreen configuration is founded, an optimality creteria is defined and a test is designed. The optimal creteria for the windscreen is as low as possible wind noise at the microphone and low reflection and direct sound attenuation from the windscreen. To find the windscreen configuration which meets the creteria best, three test is made on the windscreen configuration. First the wind speed attenuation of the windscreen configuration is measured to ensure that the windscreen configuration concept does have an effect on the wind speed. The measurement of the windspeed attenuation can be founded in \autoref{}. Secondly the frequency response of the windscreen have to be founded to ensure that the windscreen configuration does not have a large influence on the frequency measurement response of the speaker. To test this criteria, the frequency response of a speaker is measured in the anechoic chamber without any windscreen configuration and without the original windscreen. This measurement is compared with the frequency response of the speaker with the windscreen configuration. The measurement is founded in \autoref{}. Finally the wind noise is measured. To measure the wind noise two low speed and low noise fan is generating \SI{2.5}{\meter\per\second} at the microphone position. The wind noise is measured without any windscreen configuration and the original windscreen and compared with the wind noise in the microphone position in the windscreen configuration. To ensure that the background noise is identically on the wind noise measurement with and without the windscreen configuration two microphone are used and recorded simultanius. Both the time signal and the frequency content is analysed. The measurement is founded in \autoref{}. The result for all configuration is as following.

\paragraph{Configuration one} is the one with the smallest foam wedge and size of the wedge is measured to have the worst wind attenuation. The wind attenuation shows that the wind speed is lowered from \SI{8}{\meter\per\second} to \SI{2}{\meter\per\second}. But the directional turbulence in the wind is more stabile in this configuration compare the configuration three and above. The frequency response of the windscreen configuration is the one that have the lowest effect. At low frequency upto \Hz{100} the windscreen does not effect the measurement. Frequency above the frequency response gets off with about \dB{2} compare with only the original windscreen. The measured wind noise attenuation is equal zero. In the measurement the wind noise is actualy a bit worse compare to only the original windscreen. The attenuation is both approximatly \dB{10} for both with only the original windscreen and the windscreen configuration in the low frequency below \Hz{10}, but at some frequency the attenuation is lower that \dB{5} for the windscreen configuration. for frequency above \Hz{10} the windscreen configuration have no effect.

\paragraph{Configuration two} is the one with the largest foam wedge and size of the wedge is measured to have one of the best wind attenuation. The wind attenuation shows that the wind speed is lowered from \SI{8}{\meter\per\second} to \SI{1}{\meter\per\second} and have less peek in the wind speed compare to the windscreen with rockwool. The directional turbulence in the wind is more stabile in this configuration compare the configuration three and above but little less stabil compare to configuration one. The frequency response of the windscreen configuration have an amplification in the low frequency range from \Hz{80} to \Hz{600} of  \dB{2}. From \Hz{1000} and above the frequency response is very similar compare to only the original windscreen. At low frequency upto \Hz{80} the windscreen does not effect the measurement much. The windscreen attenuate the wind noise \dB{10} more than only the original windscreen from \Hz{30} and downwards to the measured limit at \Hz{2}. The frequency range between \Hz{30} and \Hz{600} have the same attenuation as the original windscreen and the frequency above have further \dB{10} more attenuation than the original windscreen.

\paragraph{Configuration three} is the one with two rockwool bat formed as an arrow and is measured to have wind attenuation between the small wedge and large wedge. The frequency response of the windscreen configuration is the worst. It alternate between $\pm$\dB{6}. At the low frequency range from \Hz{80} to \Hz{600} the amplification goes from \dB{2} at \Hz{80} to \dB{6.2} at \Hz{250} and then back to  \dB{0} at \Hz{700}. At  \Hz{1000} the attunuation is at \dB{6} and above the frequency response alternate around the frequency response of the original windscreen. The windscreen attenuate the wind noise \dB{10} more than only the original windscreen from \Hz{30} and downwards to the measured limit at \Hz{2}. The frequency range between \Hz{30} and \Hz{600} have the same attenuation as the original windscreen and the frequency above have further \dB{5} to \dB{10} more attenuation than the original windscreen. Based on that the frequency response and the wind wind noise attenuation is worse than configuration two, the configuration is excluded from the rest of the test and is not used.

\paragraph{Configuration four} is the one with only one rockwool bat where the microphone is siturated close to the side of the windscreen and is measured to have one of the best wind attenuation. The wind attenuation shows that the mean wind speed is lowered from \SI{8}{\meter\per\second} to \SI{1}{\meter\per\second}, but the directional and wind speed turbulence is less stabile compare to the configuration the windscreen with foam wedge. The wind speed turbulence circulate from \SI{0}{\meter\per\second} to \SI{2}{\meter\per\second}. The frequency response of the windscreen configuration is does not change more than $\pm$\dB{2} in the low and high frequency range. At frequency from \Hz{600} to \Hz{300} the windscreen have an attenuation of \dB{4}. MISSING NOISE ATTENUATION


%The windscreen attenuate the wind noise \dB{10} more than only the original windscreen from \Hz{30} and downwards to the measured limit at \Hz{2}. The frequency range between \Hz{30} and \Hz{600} have the same attenuation as the original windscreen and the frequency above have further \dB{5} to \dB{10} more attenuation than the original windscreen. Measure!!


\paragraph{Configuration five} is the one with only one rockwool bat and the two large wedge where the microphone is siturated as in configuration two and is measured to have the best wind attenuation. The wind attenuation shows that the mean wind speed is lowered from \SI{8}{\meter\per\second} to \SI{0.8}{\meter\per\second}, but the directional turbulence is less stabile compare to the configuration the windscreen with only foam wedge. The frequency response of the windscreen configuration is as configuration two but with little closer fit to without windscreen in the high frequency. MISSING NOISE ATTENUATION 




%The windscreen attenuate the wind noise \dB{10} more than only the original windscreen from \Hz{30} and downwards to the measured limit at \Hz{2}. The frequency range between \Hz{30} and \Hz{600} have the same attenuation as the original windscreen and the frequency above have further \dB{5} to \dB{10} more attenuation than the original windscreen. Measure!!





\subsection{Optimization of the chosen windscreen}



\xfig{design/windscreen_optimised.pdf_t}{The figure shows the single rockwool concept}{fig:td:mes_win_opt_ver}{1}  


There is made a preliminary test test to measure the effect in a fast measuring setup before a field measurement. The measurement was done in acoustics lab with two fan and a wind speed of \SI{2.5}{\meter\per\second}. The result is founded in \autoref{}

%\plot{plot/with_ball_wind_compare_large}{The graph shows one of the time measurement with configuration three}{fig:sec:pop:wit_bal_wind_com_lag}


%\plot{plot/with_ball_wind_compare_rock}{The graph shows one of the time measurement with configuration three}{fig:sec:pop:wit_bal_wind_com_roc}

As seen in the measurement, the original wind screen does have a huge wind noise attenuation but as shown in ... measurement, the wind noise can be further attenuated by the developed wind stopping concept. 




\subsection{Attenuation of the windscreen} 
The aim of this section is to analyse the inflyence of the selected windscreen. 

Free field 




\section{Angeling of the line source array}
The aim of this section is to design the turning method for the line source array and ensure that the speaker point in the desired angle.


\xfig{design/measuring_angle_plate.pdf_t}{The figure shows the angle plate}{fig:td:mes_ang_plate}{1}  

\xfig{design/laser_holder.pdf_t}{The figure shows the laser holder}{fig:td:mes_las_hol}{1}  


\section{Data logging system} 
To be able to use the information of the wind spees temparature and humidity, the data logging of the atmospherical condition have to be syncronius with the measurement. 


The speaker is chosen to be adjusted to the narrow main lobe because it is assumed that the distance from the audience to the speaker is so large that the wide angle goes beyond the audience area.

Because of limitation, the speaker is flown in a hight of \SI{6}{\meter}. 

The humidity and temperature have to be measured.

To measure the \gls{spl} coverage of the speaker a flat area with mown grass is chosen to be used. The optimal area area without any building or trees might not be posible, therefore blockage or sound reflaction surface other than the ground is only allowed to be present in the double of distance compare to the distance from the speaker to the microphone. Based on the refraction effect versus distance founded in \autoref{sec:ana:atm_ref}. The distance from the speaker to the microphone array is chosen to be \SI{50}{\meter}. The distance is based on the experience of the author described in \autoref{sec:ana:aut_exp_con} and the founded refraction effect in \autoref{sec:ana:atm_ref}. It was founded that the refraction effect should be minimal at a distance of \SI{50}{\meter} when the speed of wind is \SI{5}{\meter\per\second}. 


To keep the wind speed realistic for measurement and for concert, but still having wind pressent, the wind speed doing the measurement is limited in the range for average \SI{5}{\meter\per\second} to \SI{10}{\meter\per\second}. Less average wind speed than \SI{5}{\meter\per\second} is avoided to ensure measureble effect of the wind on sound propagation. The higher limit of the \SI{10}{\meter\per\second} is chosen to ensure that the speaker tower is safe at the hight at \SI{6}{\meter}. The limited size of the setup makes the setup wind (følsom) because it is not puttet up as a cube but only as a surface. 




where the refraction at \SI{110}{\meter} already starts at \Hz{400}. 


The area is without  The resend to use this  

\plot{plot/angle_test}{The graph shows the first transfer function measurement within the high frequency directional angle. The $L_{Aeq,5}$ \gls{spl} different between the microphones is \dB{6.77} (IR_3) The graph is normed to contain the same $L_{Aeq,5}$ \gls{spl} }{fig:ana:ang_tes}



\section{Measuring program}



