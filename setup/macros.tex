

%%%%%%%%%%%%%%%%%%%%%%%%%%%%%%%%%%%%%%%%%%%%%%%%
% Loads user defined variables
%%%%%%%%%%%%%%%%%%%%%%%%%%%%%%%%%%%%%%%%%%%%%%%%

\newcommand{\noSIunit}{$1$}% Definerer hvad der skal skrives hvis symbolet ikke har nogen enhed.
\newcommand{\obcTransducerAddress}{0xB1}% CAN adresse til tranducer modulet i OBC.

\def\circuitScale{.5}%

%%%%%%%%%%%%%%%%%%%%%%%%%%%%%%%%%%%%%%%%%%%%%%%%
% Macros for the titlepage
%%%%%%%%%%%%%%%%%%%%%%%%%%%%%%%%%%%%%%%%%%%%%%%%
%Creates the aau titlepage
\newcommand{\aautitlepage}[3]{%
  {
    %set up various length
    \ifx\titlepageleftcolumnwidth\undefined
      \newlength{\titlepageleftcolumnwidth}
      \newlength{\titlepagerightcolumnwidth}
    \fi
    \setlength{\titlepageleftcolumnwidth}{0.5\textwidth-\tabcolsep}
    \setlength{\titlepagerightcolumnwidth}{\textwidth-2\tabcolsep-\titlepageleftcolumnwidth}
    %create title page
    \thispagestyle{empty}
    \noindent%
    \begin{tabular}{@{}ll@{}}
      \parbox{\titlepageleftcolumnwidth}{
        \iflanguage{danish}{%
          \includegraphics[page=1,width=\titlepageleftcolumnwidth]{figures/aau_logo}
        }{%
          \includegraphics[page=2,width=\titlepageleftcolumnwidth]{figures/aau_logo}
        }
      } &
      \parbox{\titlepagerightcolumnwidth}{\raggedleft\sf\small
        #2
      }\bigskip\\
       #1 &
      \parbox[t]{\titlepagerightcolumnwidth}{%
        \iflanguage{danish}{%
          \textbf{Synopsis:}\smallskip\par
        }{%
          \textbf{Abstract:}\smallskip\par
        }
        \fbox{\parbox{\titlepagerightcolumnwidth-2\fboxsep-2\fboxrule}{%
          #3
        }}
      }\\
    \end{tabular}
    \vfill
    \iflanguage{danish}{%
      \noindent{\footnotesize\emph{Rapportens indhold er frit tilgængeligt, men offentliggørelse (med kildeangivelse) må kun ske efter aftale med forfatterne.}}
    }{%
      \noindent{\footnotesize\emph{The content of this report is freely available, but publication may only be pursued with reference.}}
    }
    \cleardoublepage
  }
}

%Create english project info
\newcommand{\englishprojectinfo}[8]{%
  \parbox[t]{\titlepageleftcolumnwidth}{
    \textbf{Title:}\\ #1\bigskip\par
    \textbf{Theme:}\\ #2\bigskip\par
    \textbf{Project Period:}\\ #3\bigskip\par
    \textbf{Project Group:}\\ #4\bigskip\par
    \textbf{Participants:}\\ #5\bigskip\par
    \textbf{Supervisor:}\\ #6\bigskip\par
    \textbf{Number of Pages:} \pageref{LastPage}\bigskip\par
    \textbf{Date of Completion:}\\ #8
  }
}

%Create danish project info
\newcommand{\danishprojectinfo}[8]{%
  \parbox[t]{\titlepageleftcolumnwidth}{
    \textbf{Titel:}\\ #1\bigskip\par
    \textbf{Tema:}\\ #2\bigskip\par
    \textbf{Projektperiode:}\\ #3\bigskip\par
    \textbf{Projektgruppe:}\\ #4\bigskip\par
    \textbf{Deltagere:}\\ #5\bigskip\par
    \textbf{Vejleder:}\\ #6\bigskip\par
    \textbf{Oplagstal:} #7\bigskip\par
    \textbf{Sidetal:} \pageref{LastPage}\bigskip\par
    \textbf{Afleveringsdato:}\\ #8
  }
}

%%%%%%%%%%%%%%%%%%%%%%%%%%%%%%%%%%%%%%%%%%%%%%%%
% An example environment
%%%%%%%%%%%%%%%%%%%%%%%%%%%%%%%%%%%%%%%%%%%%%%%%
\theoremheaderfont{\normalfont\bfseries}
\theorembodyfont{\normalfont}
\theoremstyle{break}
\def\theoremframecommand{{\color{aaublue!50}\vrule width 5pt \hspace{5pt}}}
\newshadedtheorem{exa}{Example}[chapter]
\newenvironment{example}[1]{%
		\begin{exa}[#1]
}{%
		\end{exa}
}


%%%%%%%%%%%%%%%%%%%%%%%%%%%%%%%%%%%%%%%%%%%%%%%%%
% Exponential function defined as upright e, \exp
%%%%%%%%%%%%%%%%%%%%%%%%%%%%%%%%%%%%%%%%%%%%%%%%%
\renewcommand{\exp}{\text{e}}


%%%%%%%%%%%%%%%%%%%%%%%%%%%%%%%%%%%%%%%%%%%%%%%%%%%
% Command \clearevenpage start chapter on even page
%%%%%%%%%%%%%%%%%%%%%%%%%%%%%%%%%%%%%%%%%%%%%%%%%%%
\makeatletter
\newcommand*{\clearevenpage}{%
  \clearpage
  \if@twoside
    \ifodd\c@page
      \hbox{}%
      \newpage
      \if@twocolumn
        \hbox{}%
        \newpage
      \fi
    \fi
  \fi
}
\makeatother

%%%%%%%%%%%%%%%%%%%%%%%%%%%%%%%%%%%%%%%%%%%%%%%%%%%%%%%
% Command \addunit. Use it to add si units to equations
%%%%%%%%%%%%%%%%%%%%%%%%%%%%%%%%%%%%%%%%%%%%%%%%%%%%%%%
\makeatletter
\providecommand\add@text{}
\newcommand\addunit[1]{%
  \gdef\add@text{\text{[}\ifthenelse{\equal{#1}{}}{\noSIunit}{\si{#1}}\text{]}\gdef\add@text{}}}% 
\renewcommand\tagform@[1]{%
  \maketag@@@{\llap{\add@text\quad}(\ignorespaces#1\unskip\@@italiccorr)}%
}
\makeatother

%%%%%%%%%%%%%%%%%%%%%%%%%%%%%%%%%%%%%%%%%%%%%
% Oversættelser af ord ved brug af \autoref{}
%%%%%%%%%%%%%%%%%%%%%%%%%%%%%%%%%%%%%%%%%%%%%
\addto\extrasdanish{%
  \def\figureautorefname{Figur}%
  \def\subfigureautorefname{Figur}%
  \def\tableautorefname{Tabel}%
  \def\partautorefname{Del}%
  \def\appendixautorefname{Bilag}%
  \def\equationautorefname{Ligning}%
  \def\Itemautorefname{Punkt}%
  \def\chapterautorefname{Kapitel}%
  \def\sectionautorefname{Afsnit}%
  \def\subsectionautorefname{Afsnit}%
  \def\subsubsectionautorefname{Underafsnit}%
  \def\paragraphautorefname{Delafsnit}%
  \def\Hfootnoteautorefname{Fodnote}%
  \def\AMSautorefname{Ligning}%
  \def\theoremautorefname{Sætning}%
  \def\pageautorefname{Side}%
  \def\requirementautorefname{Krav}%
}

\addto\extrasenglish{%
  \def\sectionautorefname{section}%
  \def\subsectionautorefname{section}%
  \def\subsubsectionautorefname{section}%
  \def\requirementautorefname{Requirement}%
  \def\algorithmautorefname{Algorithm}%
}

%%%%%%%%%%%%%%%%%%%%%%%%%%%%%%%%%%%%%%%%%%%%%%%%%%%%%%%%%%%%%%%%%%%%%%
% Tilføjer kommando \fullref{} for både at referere til nummer og navn
%%%%%%%%%%%%%%%%%%%%%%%%%%%%%%%%%%%%%%%%%%%%%%%%%%%%%%%%%%%%%%%%%%%%%%
\renewcommand*{\fullref}[1]{\hyperref[{#1}]{\autoref*{#1} \nameref*{#1}}} % One single link


%%%%%%%%%%%%%%%%%%%%%%%%%%%%%%%%%%%%%%%%%%%%%%%%%%%%%%%%%%%%%%%%%%%%%%
% Tilføjer forkortelser af ord.
%%%%%%%%%%%%%%%%%%%%%%%%%%%%%%%%%%%%%%%%%%%%%%%%%%%%%%%%%%%%%%%%%%%%%%
\newcommand{\AAU}{%
\iflanguage{english}{%
Aalborg University%
}{%
Aalborg Universitet%
}}

\newcommand{\opamp}{%
\iflanguage{english}{%
operational amplifier%
}{%
operationsforstærker%
}}

\newcommand{\hifi}{%
\iflanguage{english}{%
hi-fi amplifier%
}{%
hi-fi-forstærker%
}}

%%%%%%%%%%%%%%%%%%%%%%%%%%%%%%%%%%%%%%%%%%%%%%%%%%%%%%%%%%%%%%%%%%%%%%
% Command \DefVar{VARIBLE_NAME}
% Is used to create a variable.
% Set varible: \VARIBLE_NAME{VALUE}
% Get varible: \getVARIBLE_NAME
%%%%%%%%%%%%%%%%%%%%%%%%%%%%%%%%%%%%%%%%%%%%%%%%%%%%%%%%%%%%%%%%%%%%%%

\makeatletter%
\newcommand{\DefVar}[1]{\@namedef{#1}##1{\global\@namedef{get#1}{##1}}\@nameuse{#1}{}}%
\makeatother%

%%%%%%%%%%%%%%%%%%%%%%%%%%%%%%%%%%%%%%%%%%%%%%%%%%%%%%%%%%%%%%%%%%%%%%
% Requirements environment:
%
% \begin{requirement}
%   \requirement{}
%   \argument{}
%   \fullfilment{}
% \end{requirement}
%%%%%%%%%%%%%%%%%%%%%%%%%%%%%%%%%%%%%%%%%%%%%%%%%%%%%%%%%%%%%%%%%%%%%%

\def\reqPrefixName{??}% Individual prefix for the requirements
\newcounter{reqIDCounter}% Requirement counter for the subsections
\newcounter{requirement}% Absolute requirement counter. Used to trigger label/reference target.

\newlength{\reqBoxWidth}% Create length varible to define width of requirement box
\setlength{\reqBoxWidth}{5cm}% Actual width of requirement box

\newcommand{\reqPrefix}[1]{% Command to define requirement prefix
\ifthenelse{\equal{#1}{}}{\reqPrefixName}{\setcounter{reqIDCounter}{0}\def\reqPrefixName{#1}}%
}

\makeatletter% Define requirementID for references
\newcommand{\reqLabel}{%
\protected@edef\@currentlabel{\reqPrefixName\thereqIDCounter}%
\protected@edef\@currentlabelname{Requirement}%
}\makeatother%

\newenvironment{requirement}%
{\par\vspace{\baselineskip}\noindent\ignorespaces%
\refstepcounter{requirement}%
\stepcounter{reqIDCounter}%
\reqLabel%
\DefVar{requirement}\DefVar{argument}%\DefVar{fullfilment}%
%
\tabularx{1\textwidth}{p{\reqBoxWidth} !{\color{white}{\vrule width 2pt}} X}%
\greyrow \parbox[t]{\reqBoxWidth}{\textsb{Update}\\\getrequirement\vskip1mm}% ~\reqPrefixName\thereqIDCounter
&%
\parbox[t]{\textwidth-\reqBoxWidth-4\tabcolsep-2pt}{\textsb{Argumentation}\\\getargument\vskip1mm}
%\addlinespace[2pt]%
%\greyrow \multicolumn{2}{l}{\parbox{\textwidth-2\tabcolsep}{\vskip1mm\textsb{Conditions of fulfilment}\\\getfullfilment\vskip1mm}}%
}%
{\endtabularx\par\ignorespacesafterend}


%%%%%%%%%%%%%%%%%%%%%%%%%%%%%%%%%%%%%%%%%%%%%%%%%%%%%%%%%%%%%%%%%%%%%%
% Tilføjer kommando \numnameref{labelnavn}
% Kommandoen tilføjer en reference til et afsnit med både nummer og navn.
%%%%%%%%%%%%%%%%%%%%%%%%%%%%%%%%%%%%%%%%%%%%%%%%%%%%%%%%%%%%%%%%%%%%%%
\newcommand{\numnameref}[1]{%
\hyperref[#1]{\autoref{#1}: \nameref{#1}}%
}

%%%%%%%%%%%%%%%%%%%%%%%%%%%%%%%%%%%%%%%%%%%%%%%%%%%%%%%%%%%%%%%%%%%%%%
% Defines subscript as upright text if it contains more than one character.
% 
%%%%%%%%%%%%%%%%%%%%%%%%%%%%%%%%%%%%%%%%%%%%%%%%%%%%%%%%%%%%%%%%%%%%%%

\catcode`\_=12% Makes underscore an inactive character
\begingroup\lccode`~=`\_% Loads underscore character for redefinition
\lowercase{\endgroup\def~}#1{% Start definition of underscore
\StrLen{#1}[\subscriptstringlen]% Determine length of subscript string
\ifthenelse{\subscriptstringlen=1}% Test if subscript string contain one character
{\sb{#1}}% Prints subscript as italic if only one character is present
{\sb{\mathrm{#1}}}}% Prints subscript as upright text if there are more than one character
\AtBeginDocument{\mathcode`\_=\string"8000 }% Makes underscore active only in math mode


%%%%%%%%%%%%%%%%%%%%%%%%%%%%%%%%%%%%%%%%%%%%%%%%%%%%%%%%%%%%%%%%%%%%%%
% Defind counter
% Bruges til forklaringer af ligninger og equpment
%%%%%%%%%%%%%%%%%%%%%%%%%%%%%%%%%%%%%%%%%%%%%%%%%%%%%%%%%%%%%%%%%%%%%%
\newcounter{firstexplain}% Holder styr på om det er den første symbolforklaring

%%%%%%%%%%%%%%%%%%%%%%%%%%%%%%%%%%%%%%%%%%%%%%%%%%%%%%%%%%%%%%%%%%%%%%
% Defind counter
% Bruges til forklaringer af ligninger og equpment
%%%%%%%%%%%%%%%%%%%%%%%%%%%%%%%%%%%%%%%%%%%%%%%%%%%%%%%%%%%%%%%%%%%%%%

%%%%%%%%%%%%%%%%%%%%%%%%%%%%%%%%%%%%%%%%%%%%%%%%%%%%%%%%%%%%%%%%%%%%%%
% Tilføjer kommando \startexplain, \stopexplain og \explain{variable}{test}{is unit}
% Bruges til forklaringer af ligninger.
%%%%%%%%%%%%%%%%%%%%%%%%%%%%%%%%%%%%%%%%%%%%%%%%%%%%%%%%%%%%%%%%%%%%%%
%\newcounter{firstexplain}% Holder styr på om det er den første symbolforklaring
%
%\def\startexplain{%
%	\setcounter{firstexplain}{1}%
%	{\noindent}%
%	{\par\noindent}
%	\begin{tabular}{@{}p{.06\columnwidth}p{.76\columnwidth}@{\hskip.04\columnwidth}p{.02\columnwidth}@{}}}%
%\def\stopexplain{\end{tabular}\\[10pt]}%
%
%\newcommand{\explain}[2]{%
%	\ifthenelse{\thefirstexplain=1}{%
%	Where:\\
%	&#1&[\ifthenelse{\equal{#2}{}}{\noSIunit}{#2}]\\\setcounter{firstexplain}{0}
%	}{%
%	&#1&[\ifthenelse{\equal{#2}{}}{\noSIunit}{#2}]\\%
%	}%
%}%




\def\startexplain{\setcounter{firstexplain}{1}{\noindent}{\par\noindent}\renewcommand{\arraystretch}{1}\begin{tabular}{@{}p{.06\columnwidth}p{.04\columnwidth}@{}p{.72\columnwidth}@{\hskip.04\columnwidth}p{.02\columnwidth}@{}}}%

\def\stopexplain{\end{tabular}\\[10pt]}%

\newcommand{\explain}[3]{\ifthenelse{\thefirstexplain=1}{Where:\\
	&$#1$&#2&[\si{#3}]\\\setcounter{firstexplain}{0}
	}{%	
	&$#1$&#2&[\si{#3}]\\%
	}%
}%

%%%%%%%%%%%%%%%%%%%%%%%%%%%%%%%%%%%%%%%%%%%%%%%%%%%%%%%%%%%%%%%%%%%%%%
% Tilføjer kommando \startexplain, \stopexplain og \explain{}{}{}
% Bruges til forklaringer af ligninger.
%%%%%%%%%%%%%%%%%%%%%%%%%%%%%%%%%%%%%%%%%%%%%%%%%%%%%%%%%%%%%%%%%%%%%%

%%%%%%%%%%%%%%%%%%%%%%%%%%%%%%%%%%%%%%%%%%%%%%%%%%%%%%%%%%%%%%%%%%%%%%
% Tilføjer kommando \starteq, \stopequp og \equp{variable}{test}{is unit}{}
% Bruges til forklaringer af ligninger.
%%%%%%%%%%%%%%%%%%%%%%%%%%%%%%%%%%%%%%%%%%%%%%%%%%%%%%%%%%%%%%%%%%%%%%


\def\startequipment{\setcounter{firstexplain}{1}{\noindent}{\par\noindent}
\begin{table}[!ht]
\caption{Equipment list}
\begin{tabular}{@{}p{.23\columnwidth}|p{.40\columnwidth}@{}|p{.20\columnwidth}@{}|p{.10\columnwidth}@{}}}%

\def\stopequipment{\end{tabular}
\end{table}\\[10pt]}%

\newcommand{\equipment}[4]{\ifthenelse{\thefirstexplain=1}{
Description & Model & Serial-no & AAU-no \\ \hline
	#1 & #2 & #3 & #4 \Tstrut \\ \setcounter{firstexplain}{0}
	}{%	
	#1 & #2 & #3 & #4\\ %
	}%
}%


%%%%%%%%%%%%%%%%%%%%%%%%%%%%%%%%%%%%%%%%%%%%%%%%%%%%%%%%%%%%%%%%%%%%%%
% Tilføjer kommando \starteq, \stopequp og \equp{variable}{test}{is unit}{}
% Bruges til forklaringer af ligninger.
%%%%%%%%%%%%%%%%%%%%%%%%%%%%%%%%%%%%%%%%%%%%%%%%%%%%%%%%%%%%%%%%%%%%%%


%%%%%%%%%%%%%%%%%%%%%%%%%%%%%%%%%%%%%%%%%%%%%%%%%%%%%%%%%%%%%%%%%%%%%%
% Tilføjer kommando spørge skema
%%%%%%%%%%%%%%%%%%%%%%%%%%%%%%%%%%%%%%%%%%%%%%%%%%%%%%%%%%%%%%%%%%%%%%


\def\startask{\setcounter{firstexplain}{1}{\noindent}{\par\noindent}\begin{tabular}{@{\vspace{0.1cm}}p{.03\columnwidth}p{.60\columnwidth}@{\hskip.04\columnwidth}p{.15\columnwidth}@{\hskip.04\columnwidth}p{.12\columnwidth}@{}}}%

\def\stopask{\end{tabular}\\[10pt]}%

\newcommand{\ask}[3]{\ifthenelse{\thefirstexplain=1}{
&Question & Answer & Unit  \\ \cline{2-4} 
	&\textit{#1}&#2&#3 \TBstrut \\\setcounter{firstexplain}{0}
	}{%	
	&\textit{#1}&#2&#3 \TBstrut \\%
	}%
}%


%%%%%%%%%%%%%%%%%%%%%%%%%%%%%%%%%%%%%%%%%%%%%%%%%%%%%%%%%%%%%%%%%%%%%%
% Tilføjer kommando spørge skema
%%%%%%%%%%%%%%%%%%%%%%%%%%%%%%%%%%%%%%%%%%%%%%%%%%%%%%%%%%%%%%%%%%%%%%

%%%%%%%%%%%%%%%%%%%%%%%%%%%%%%%%%%%%%%%%%%%%%%%%%%%%%%%%%%%%%%%%%%%%%%
% Tilføjer figur
%%%%%%%%%%%%%%%%%%%%%%%%%%%%%%%%%%%%%%%%%%%%%%%%%%%%%%%%%%%%%%%%%%%%%%

\newcommand{\dfig}[8]{
\begin{figure}[H]
\centering	  
 \captionsetup{width=1\linewidth}
     \begin{subfigure}[t]{#7\textwidth}
        \centering
        \captionsetup{width=0.90\linewidth}
        \includegraphics[width=0.90\linewidth]{#1}
        \caption{#2}
        \label{#6_1}
     \end{subfigure}%
     \begin{subfigure}[t]{#8\textwidth}
        \centering
        \captionsetup{width=0.90\linewidth}
        \includegraphics[width=0.90\linewidth]{#3}
        \caption{#4}
        \label{#6_2}
     \end{subfigure}%
\caption{#5}
\label{#6}
\end{figure}}%

\newcommand{\fig}[4]{
\begin{figure}[H]
\centering
	  \captionsetup{width=#4\linewidth}
\includegraphics[width=#4\textwidth]{#1}
\caption{#2}
\label{#3}
\end{figure}}%

\newcommand{\xfig}[4]{
\begin{figure}[H]
\centering
	  \captionsetup{width=#4\linewidth}
	  \input{figures/#1}
%\includegraphics[width=#4\textwidth]{#1}
\caption{#2}
\label{#3}
\end{figure}}%


\newcommand{\plot}[3]{
 \begin{figure}[H]
	\centering
	\include{#1}
		\caption{#2}
		\label{#3}
\end{figure}}

\newcommand{\plotsettings}[0]{
width=120mm, 
height=65mm, 
at={(5mm,5mm)}, 
scale only axis,
grid=both,
grid style={line width=.1pt, draw=gray!10},
major grid style={line width=.2pt,draw=gray!50}, 
minor tick num=4,
axis x line*=bottom,
axis y line*=left,
axis background/.style={fill=white},
}




\newcommand{\plotsettingslow}[0]{
width=120mm, 
height=45mm, 
at={(5mm,5mm)}, 
scale only axis,
grid=both,
grid style={line width=.1pt, draw=gray!10},
major grid style={line width=.2pt,draw=gray!50}, 
minor tick num=4,
axis x line*=bottom,
axis y line*=left,
axis background/.style={fill=white},
}


%%%%%%%%%%%%%%%%%%%%%%%%%%%%%%%%%%%%%%%%%%%%%%%%%%%%%%%%%%%%%%%%%%%%%%
% Tilføjer figur
%%%%%%%%%%%%%%%%%%%%%%%%%%%%%%%%%%%%%%%%%%%%%%%%%%%%%%%%%%%%%%%%%%%%%%


%%%%%%%%%%%%%%%%%%%%%%%%%%%%%%%%%%%%%%%%%%%%%%%%%%%%%%%%%%%%%%%%%%%%%%
% Tilføjer kommando \hyph
% Bruges til bindestreger i ord så de stadig kan deles korrekt af Latex.
%%%%%%%%%%%%%%%%%%%%%%%%%%%%%%%%%%%%%%%%%%%%%%%%%%%%%%%%%%%%%%%%%%%%%%
\def\hyph{-\penalty0\hskip0pt\relax}


%%%%%%%%%%%%%%%%%%%%%%%%%%%%%%%%%%%%%%%%%%%%%%%%%%%%%%%%%%%%%%%%%%%%%%
% Tilføjer kommando \hex og \bin \decibel
% Bruges til at skrive hexidecimal og binære tal.
%%%%%%%%%%%%%%%%%%%%%%%%%%%%%%%%%%%%%%%%%%%%%%%%%%%%%%%%%%%%%%%%%%%%%%
\newcommand{\hex}[1]{%
	\texttt{#1$_{16}$}
}%

\newcommand{\bin}[1]{%
	\texttt{#1$_{2}$}
}%

\newcommand{\dB}[1]{%
	\SI{#1}{\decibel} \gls{spl}}%



\newcommand{\db}[0]{%
	\si{\decibel}
}%

\newcommand{\octave}[1]{%
	${\scriptsize \sfrac{1}{#1}}$ octave
}%

%\newcommand{\Hz}[1]{%
%	\SI{#1}{\hertz}
%}%

\newcommand{\dif}[0]{%
	$\Delta$
}%

\newcommand{\greyrow}{%
	\rowcolor{lightGrey}
}%


\newcommand{\Hz}[1]{%
\def\variable{5}
\ifnum#1>999   
\pgfmathparse{#1/1000}
    \SI{\pgfmathresult}{\kilo\hertz}%
  \else
    \SI{#1}{\hertz}\fi}



%%%%%%%%%%%%%%%%%%%%%%%%%%%%%%%%%%%%%%%%%%%%%%%%%%%%%%%%%%%%%%%%%%%%%%
% Tilføjer kommando \citeref
% Bruges til at referere til egen rapport/bilag.
%%%%%%%%%%%%%%%%%%%%%%%%%%%%%%%%%%%%%%%%%%%%%%%%%%%%%%%%%%%%%%%%%%%%%%
\newcommand{\citeref}[1]{%
	[\autoref{#1}]%
}%


%%%%%%%%%%%%%%%%%%%%%%%%%%%%%%%%%%%%%%%%%%%%%%%%%%%%%%%%%%%%%%%%%%%%%%
% Tilføjer kommando \rot{}
% Bruges fx til at rotere kollonneoverskrifter i en tabel.
%%%%%%%%%%%%%%%%%%%%%%%%%%%%%%%%%%%%%%%%%%%%%%%%%%%%%%%%%%%%%%%%%%%%%%
\newcommand*\rot{\rotatebox{60}}


%%%%%%%%%%%%%%%%%%%%%%%%%%%%%%%%%%%%%%%%%%%%%%%%%%%%%%%%%%%%%%%%%%%%%%
% Tilføjer kommando \file{}
% Bruges til at indsætte et filnavn i teksten.
%%%%%%%%%%%%%%%%%%%%%%%%%%%%%%%%%%%%%%%%%%%%%%%%%%%%%%%%%%%%%%%%%%%%%%
\newcommand{\file}[1]{\texttt{#1}}

\definecolor{javared}{rgb}{0.6,0,0} % for strings
\definecolor{javagreen}{rgb}{0.25,0.5,0.35} % comments
\definecolor{javapurple}{rgb}{0.5,0,0.35} % keywords
\definecolor{javadocblue}{rgb}{0.25,0.35,0.75} % javadoc
\definecolor{gray}{rgb}{0.4,0.4,0.4}
\definecolor{darkblue}{rgb}{0.0,0.0,0.6}
\definecolor{cyan}{rgb}{0.0,0.6,0.6}
\definecolor{lightblue}{rgb}{0.0,0.3,0.7}
\definecolor{orange}{rgb}{0.8,0.3,0.0}
\definecolor{matlabstring}{rgb}{.627,.126,.941}
\definecolor{arduinoBlue}{rgb}{0.01, 0.73, 0.75}
\definecolor{arduinoGreen}{rgb}{0.17, 0.43, 0.01}


%Hvordan XML kode skal se ud
\lstdefinestyle{customXML}{
  belowcaptionskip=1\baselineskip,
  breaklines=true,
  frame=L,
  xleftmargin=\parindent,
  language=XML,
  showstringspaces=false,
  basicstyle=\footnotesize\ttfamily,
  morestring=[b]",
  morestring=[s]{>}{<},
  morecomment=[s]{<?}{?>},
  stringstyle=\color{black},
  identifierstyle=\color{darkblue},
  keywordstyle=\color{cyan},
  morekeywords={xmlns,version,type},
 commentstyle=\color{gray}\upshape,
}


%Hvordan java kode skal se ud
\lstdefinestyle{customjava}{
  belowcaptionskip=1\baselineskip,
  breaklines=true,
  frame=L,
  xleftmargin=\parindent,
  language=java,
  showstringspaces=false,
  basicstyle=\footnotesize\ttfamily,
  keywordstyle=\color{javapurple}\bfseries,
stringstyle=\color{javared},
commentstyle=\color{javagreen},
morecomment=[s][\color{javadocblue}]{/**}{*/},
}

%Hvordan C kode skal se ud
\lstdefinestyle{customc}{
  belowcaptionskip=1\baselineskip,
  breaklines=true,
  frame=L,
  xleftmargin=\parindent,
  language=C,
  showstringspaces=false,
  basicstyle=\footnotesize\ttfamily,
  keywordstyle=\bfseries\color{arduinoBlue},
  commentstyle=\itshape\color{gray},
  identifierstyle=\color{black},
  stringstyle=\color{arduinoBlue},
  emph=[1]{Serial,begin,attachInterrupt,digitalPinToInterrupt,Timer1,initialize,attachInterrupt,Serial,print,DHT,analogRead,delay,millis},emphstyle=[1]\color{orange}, 
emph=[2]{\#define,setup,loop,for,if,else},emphstyle=[2]\color{arduinoGreen}, 
emph=[3]{true,false,INPUT,FALLING},emphstyle=[3]\color{arduinoBlue}, 
}

%Hvordan ASM kode skal se ud
\lstdefinestyle{customassembly}{
  belowcaptionskip=1\baselineskip,
  breaklines=true,
  frame=L,
  xleftmargin=\parindent,
  %language=assembly,
  showstringspaces=false,
  basicstyle=\footnotesize\ttfamily,
  keywordstyle=\bfseries\color{green!40!black},
%where there is a space  
  numbers=left,%
  numberstyle={\tiny \color{black}},% size of the numbers
  numbersep=9pt, % this defines how far the numbers are from the text
  commentstyle=\itshape\color{javagreen},
  %assembly structure
  emph=[1]{equ,text,data,word,set},emphstyle=[1]\color{matlabstring}, 
  %commands
  emph=[2]{MOV,mov,BTST,BCC,NOP,RET},emphstyle=[2]\color{gray}, 
  %label
  emph=[3]{in,recieve_flag,out,transmit_flag,buffer,positive,negative,final,case_one,case_two},emphstyle=[3]\color{blue},
  identifierstyle=\color{black},
  stringstyle=\color{blue},
    comment=[l];,                              % comments
}

%Hvordan Matlab kode skal se ud
\lstdefinestyle{custommatlab}{
basicstyle=\footnotesize\ttfamily,
    breaklines=true,%
    morekeywords={matlab2tikz},
    keywordstyle=\color{blue},
    %morekeywords=[2]{1}, 
    keywordstyle=[2]{\color{black}},
    identifierstyle=\color{black},
    morestring=[m]', % defines that strings are enclosed in double quotes
    stringstyle=\color{matlabstring},
    showstringspaces=false,%without this there will be a symbol in the places where there is a space
    numbers=left,%
    numberstyle={\tiny \color{black}},% size of the numbers
    numbersep=9pt, % this defines how far the numbers are from the text
    emph=[1]{if,for,end,break,while,else,function},emphstyle=[1]\color{blue}, %some words to emphasise
    emph=[2]{all,on},emphstyle=[2]\color{matlabstring}, 
% matlat languate definition
  comment=[l]\%,                              % comments
  morecomment=[l]...,                         % comments
  morecomment=[s]{\%\{}{\%\}},                % block comments
  commentstyle=\color{Green},
}

%Hvordan VHDL kode skal se ud
\lstdefinestyle{customVHDL}{
  belowcaptionskip=1\baselineskip,
  breaklines=true,
  frame=L,
  xleftmargin=\parindent,
  language=VHDL,
  showstringspaces=false,
  basicstyle=\footnotesize\ttfamily,
  keywordstyle=\bfseries\color{blue!100!black!80},
  commentstyle=\itshape\color{green!90!black!90},
  identifierstyle=\color{black},
  stringstyle=\color{orange},
}

%Hvordan PHP kode skal se ud
\lstdefinestyle{customPHP}{
  belowcaptionskip=1\baselineskip,
  breaklines=true,
  frame=L,
  xleftmargin=\parindent,
  language=PHP,
  showstringspaces=false,
  basicstyle=\footnotesize\ttfamily,
 keywordstyle    = \color{blue},
  stringstyle     = \color{gray},
  identifierstyle = \color{lightblue},
  commentstyle    = \color{green},
  emph            =[1]{php},
  emphstyle       =[1]\color{black},
}
%Commando til at indsætte kode i latex
\newcommand{\includeCode}[7]{\lstinputlisting[caption=#5 | #1, style=custom#2, numbers=left, firstnumber=#3, firstline=#3, lastline=#4, label=#6]{#7#1}}

%Caption navn foran nummeret i kode
\renewcommand{\lstlistingname}{Code snippet}
\def\lstlistingautorefname{Code snippet}


\makeatletter
%\newcommand*{\getlength}[1]{\strip@pt#1}
% Or rounded back to `mm` (there will be some rounding errors!)
\newcommand*{\getlength}[1]{\strip@pt\dimexpr0.35146\dimexpr#1\relax\relax}
%
\makeatother

%%%%%%%%%%%%%%%%%%%%%%%%%%%%%%%%%%%%%%%%%%%%%%%%%%%%%%%%%%%%%%%%%%%%%%
% Tilføjer MATALB
% Bruges i stedet for gls
%%%%%%%%%%%%%%%%%%%%%%%%%%%%%%%%%%%%%%%%%%%%%%%%%%%%%%%%%%%%%%%%%%%%%%
\newcommand{\matlab}{MATLAB\textsuperscript{\textregistered}}






%% measurement
\newcommand{\measurement}[2]{%
 		\hyperref[#1]{Measurement #2} \autoref{#1} 
}%



%\newcommand{\meas}[2]{%
%  \expandafter\newcommand\csname 1#1\endcsname{#2}%
%}